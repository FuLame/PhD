\documentclass[titlepage,a4paper]{book}
%\documentclass[12pt]{report}
\usepackage[english]{babel}
%\usepackage[polski]{babel}
\usepackage[utf8]{inputenc} 
\usepackage{wrapfig}
\usepackage{graphicx}
\usepackage{color}
\usepackage{colortbl}
\usepackage{amsfonts}
\usepackage{amssymb}
\usepackage{amsmath}
\usepackage{amsthm}
\usepackage{subfigure}
\usepackage[papersize={21cm,29.7cm},text={15.5cm,24.5cm},left=2.7cm,top=2.5cm]{geometry}
\usepackage{enumerate}
\usepackage{rotfloat}
\usepackage[T1]{fontenc}
\usepackage{braket}
\usepackage{siunitx}
\usepackage{hyperref}

\newcommand{\wciecie}{\quad\phantom{v}}
\newcommand{\refe}[1]{(\ref{#1})}
\newcommand{\myparagraph}[1]{\paragraph{#1}\mbox{}\\}
\renewcommand{\baselinestretch}{1.3}
\renewcommand{\arraystretch}{1.4}


\hypersetup{
    colorlinks,
    citecolor=black,
    filecolor=black,
    linkcolor=black,
    urlcolor=black
}

%\fancyhf{}
%\fancyhead[R]{\small\bfseries\thepage}
%\fancyhead[L]{\small\bfseries\rightmark}
%\fancyfoot[C]{\small\bfseries\thepage}
%\renewcommand{\headrulewidth}{0.5pt}
%\renewcommand{\footrulewidth}{0pt}
%\renewcommand*{\tablename}{Tab.}
%\addtolength{\headheight}{0.5pt}
%\fancypagestyle{plain}

\begin{document}
\newcommand{\HRule}{\rule{\linewidth}{0.5mm}}

%\begin{titlepage}
%\begin{center}


\newpage
\large
\thispagestyle{empty}
\tableofcontents


\newpage

\chapter{Introduction}
\wciecie
Since the dawn of science, people always wanted to encounter new frontiers and cross them, expanding the boundaries of the universal knowledge. Even in the modern times, it is still our duty to explore the matter around us and the laws governing it. Although throughout the 19th  and 20th century, chemistry succeed in finding and classifying most of the elements -- building blocks of matter, there is still plenty of work to do in discovery and classification of the distinct states of matter, called phases. Matter in the quantum approach can form different phases, such as crystalline solids, magnets and superconductors. All of them are subjects of study of a discipline called condensed matter physics. 

Condensed matter physics is a branch of physics that deals with the physical properties of condensed phases of matter. The goal of the discipline is to understand the behavior of these phases and describe them in mathematical laws. In particular -- the laws of quantum mechanics, electromagnetism and statistical mechanics.

Nowadays, the interest of condensed matter physics lies in a discovery and characterization of novel particles and phases, which are present in solid state materials. These particles often behave in the most uncanny manner like the Dirac fermions, which behavior mimics the behavior of relativistic particles, or the exotic Majorana fermions, which are their own antiparticles. These kind of excitations and more can be realized in condensed matter systems.

The most familiar condensed phases are solids and liquids. However, there are more uncommon condensed phases, like the superconducting phase exhibited by certain materials at low temperature, the ferromagnetic and antiferromagnetic phases of spins on atomic lattices, and the Bose–Einstein condensate found in cold atomic systems.

\section{Topological states of matter}
\wciecie
One of the most remarkable achievements of condensed matter physics in the recent times is the classification of quantum states of matter by the principle of spontaneous symmetry breaking \cite{Anderson_Topology}. The pattern of symmetry breaking led to a unique concept of order parameter, which can be understood in terms of the famous work of Landau-Ginzburg \cite{Landau_Topology}, where the notion of the effective field theory is described. The effective field theory is determined by the general properties like dimensionality and symmetry of the order parameter, and can be used to give a universal description of quantum states of matter.

The following examples can be given to better understand the concept of states of matter and symmetry breaking: a crystalline solid breaks translation symmetry, despite the fact that the interaction between its atomic cells is translationally invariant. A rotational symmetry in magnetic systems is spontaneously broken, even though the fundamental interactions are isotropic. A superconductor breaks the more subtle gauge symmetry, which as a consequence leads to phenomena such as flux quantization and Josephson effect \cite{Josephson_Topology}.

The concept of symmetry breaking and local order parameter used to describe the phase transition is well accepted as in general. However, it fails to explain some interesting phenomena such as the integer Quantum Hall Effect (QHE) \cite{Klitzing_Topology} and many body phases of the fractional QHE \cite{Tsui_Topology}. The study of these effects ultimately led to a new paradigm in the classification of condensed matter systems -- the concept of topological order \cite{Laughlin_Topology} \cite{Thouless_Topology}.

\subsection{The Quantum Hall Effect}
\wciecie
The electronic band structure at high magnetic field forms a set of distinct Landau levels. The way how levels are occupied determines whether the current can flow or not through the system. According to the standard band theory, which successfully explains most of the electrical properties of solids, there should be no current flowing through the sample, if the number of occupied levels is an integer. On the other hand, if there is some number of unoccupied states at given level, the current can flow. 

Yet a system in the quantum Hall state behaves in a different manner. Even if its bulk is insulating (there is an energy gap between the highest occupied band and the lowest empty band), the electrical conductivity is nonzero. The electric current is still carried along the edges of the system, forming discrete channels of conductance. Those edge states are chiral -- the direction of current propagation depends on the direction of magnetic field. The current in the channels avoids dissipation and has a very precise value of resistance, giving rise to a quantized Hall effect. The basic concept of QHE is presented on Figure \ref{fig:QHE}. 
\begin{figure}[ht]
	\centering
	\includegraphics[width=8cm]{gfx/QHE.jpg}
	\vspace{-10pt}
	\caption{\textit{asd}}
	\label{fig:QHE}
\end{figure} 

The quantum Hall state turned out to be the first example of a quantum state being topologically different from all other states of matter known before. The state in which the quantum Hall occurs, does not break any symmetries but it defines a topological phase, meaning that particular fundamental properties are insensitive to smooth changes in general parameters of the system. The very fundamental reason for such a quantization is the existence of topological invariants, in case of the QHE such an invariant is the conductance. The conductance takes values only of integer units of $e^2/h$, and is independent of the type of material investigated and does not change (is invariant) for smooth variations of material parameters – it can be considered as non-local order parameter of the system.  

\subsection{Topological Invariant}
\wciecie
This invariant, also called \textit{Chern number}, was connected for the first time with the quantized value of Hall resistance by Thouless \textit{et al.} \cite{Thouless_Topology}. The idea is related to a phase that is acquired by the Bloch wave functions of bulk electrons, when the wave-vector \textbf{\textit{k}} varies over the boundary of the Brillouin zone. This phase is called Berry phase, after Sir Michael Berry, who described it in 1984 \cite{Berry_phase}. The Chern number is defined as the sum of the Berry phases over all occupied \textbf{bulk} bands, and is strongly related to the number of conducting \textbf{edge} modes defining the Hall conductivity in the quantum Hall regime. This means that the topological properties of the edge channels are directly related to the bulk. This relation is called bulk-boundary correspondence and is necessary for understanding of the topological insulators.

Topological invariant was introduced as a mathematical concept used to classify different geometrical objects into broad classes. One of such invariants can be the number of holes on the geometrical surface. The most famous example is a coffee mug and a torus. Both of them classify as the same topological object because both of them posses exactly one hole and one can be deformed, via a smooth transformation, into the other, and vice-versa. A sphere can be smoothly deformed into an ellipsoid, because they share the same number of holes (zero), etc. Topology tends to disregard the small differences of objects and focus on their general properties. In this context, QHE and quantized conductance, which can be found in a variety of materials of different shape, remains unchanged -- it is an invariant. 
 
\subsection{Topology in condensed matter}
\wciecie
The connection between geometrical classes of objects and condensed matter physics is not direct -- topology in solid state physics has very little to do with shape of considered material, and definitely the subject is much broader than just the QHE. 

Those two disciplines were probably first linked by the works of Berezinski \cite{Berezinski1}\cite{Berezinski2}, and Kosterlitz and Thouless \cite{Thouless_nobel}. It is worth to highlight that the Nobel Prize in Physics in the year 2016 was divided -- one half awarded to David J. Thouless, the other half jointly to F. Duncan M. Haldane and J. Michael Kosterlitz \textit{for theoretical discoveries of topological phase transitions and topological phases of matter}.

In topological geometry there are surfaces, holes and smooth transformations of surfaces, which does not require tearing the surface or making holes in it. In condensed matter Hamiltonians are used to describe a given system, providing information about energy gaps (which play a role of holes in geometry) and its band structure. There is a possibility to transform given Hamiltonian into a different one, by changing some parameters that the Hamiltonian depends on. If the transformation is smooth in a sense that it does not require closing of a gap (an equivalent of creating a hole in geometry) an any point, then the transformation preserves the topology of the system. This concept, where number of energy gap closing is a topological invariant, obviously can be applied only to systems, which posses the band gap -- semiconductors and insulators, it cannot be applied to metals nor superconductors. 

The name and the idea of topological insulators can be traced back to the works of Kane and Mele, where an universal concept of identifying another topological index (invariant) was described \cite{Kane_Topology2}. They proposed a realization of a system in graphene \cite{Kane_Topology}, where the Spin-Orbit Coupling (SOC) opens a band gap, rendering the bulk of the sample insulating. The same way as magnetic field suppresses bulk conductivity in QHE, however, SOC does not require application of external magnetic field. Nevertheless, at the boundary of this system a set of topologically protected edge states should emerge. These states are analogous to the states of the QHE. The difference is that these states are protected by a time-reversal symmetry, thus they are not chiral, as chirality is forbidden, but helical (spin-polarized). That is why the effect of emergence of spin-polarized quantum Hall edge states without magnetic field was called quantum spin Hall effect (QSHE), and a 2D topological insulator, where QSHE occurs -- quantum spin Hall (QSH) insulator.

Unfortunately, the proposed realisation of the QSHE in graphene turned out to be unrealistic, because the gap opened by SOC is extremaly small, of the order of 1 $\mu$eV \cite{Yao_Topology}\cite{Min_Topology}. One year later, in 2006 QSHE was predicted by Bernevig \cite{Bernevig_Topology2}\cite{Bernevig_Topology1} and experimetally observed in 2007 by König \cite{Konig_Topology} in HgTe/CdTe quantum well system (Figure \ref{fig:QSHE_HgTe}). After the prediction of topological insulating phase in HgTe/CdTe quantum well,  a similar phase was predicted by Liu \cite{Liu_Topology} in a InAs/GaSb, and strained GaAs \cite{Bernevig_Topology1}. 

\begin{figure}[ht]
	\centering
	\includegraphics[width=12cm]{gfx/QSHE_HgTe.png}
	\vspace{-10pt}
	\caption{\textit{\textbf{Panel a)} Schematic of the spin-polarized egde channels in a quantum spin Hall insulator. \textbf{Panel b)}. The longitudinal resistance of various normal (I) and inverted (II, III, and IV) QW structures as a function of gate voltage measured for $B = 0$ T at $T = 30$ mK. Image comes from the work of König \cite{Konig_Topology}.}}
	\label{fig:QSHE_HgTe}
\end{figure} 

After the initial findings of topological insulators in 2D, the 3D topological insulating phase was predicted in the Bi$_{1-x}$Sb$_x$ alloy for a specific compositions $x$ \cite{Fu_Topology}, and shortly after the topologically nontrivial surface states were observed by angle-resolved photoemission spectroscopy (ARPES) by Hsieh \cite{Hsieh}. Similarly, the topological insulators in 3D were predicted in Bi$_2$Te$_3$, Sb$_2$Te$_3$ \cite{Zhang_Topology} and Bi$_2$Se$_3$ \cite{Zhang_Topology}\cite{Xia_Topology}. There compounds exhibit a large bulk band gap and a gapless surface states consisting of a single Dirac cone. Xia \cite{Xia_Topology} and Chen \cite{Chen_Topology} observed a linear dispersion relation of this states using ARPES (Figure \ref{fig:ARPES}). In 2011, the Molenkamp's group proved that the thick strained layers of HgTe belong to 3D topological insulators as well \cite{Brune_State2}.

\begin{figure}[H]
	\centering
	\includegraphics[width=12cm]{gfx/ARPES.png}
	\vspace{-10pt}
	\caption{\textit{ARPES measurements of Bi$_2$Se$_3$ (111) evidencing the surface states with linear energy dispersion, forming a unique Dirac cone in the bulk near the $\Gamma$ point. Image comes from the work of Xia \cite{Xia_Topology}.}}
	\label{fig:ARPES}
\end{figure}

%\clearpage
\subsection{The properties of topological states}
\wciecie
The interesting phenomena take place at the boundary of two materials characterized by a different Chern number. The best description can be found in \cite{Bernevig_book} which explains that the presence of the edge states is a fundamental aspect of many topological insulators. The argument revolves around a direct observable manifestation of a Chern number -- Hall conductance.

Given a set of two insulators, each with a different value of Hall conductance, in a close proximity, so they have a common boundary, and away from the boundary, the two insulators extend to infinity. The Chern number is always an integer, defined separately on both sides of the interface between the two insulators. It cannot be changed unless the bulk gap closes and reopens again with a different Chern number on the other side. This means that the boundary region connecting two insulators with different values of Hall conductance must posses a gap-closing and gap-reopening point somewhere on it -- which is precisely an edge mode. Otherwise, the whole space would be gapped, which by definition means that the Hall conductance in the whole space would be the same, which does not fulfill the assumptions. This kind of considerations can be applied to any boundary region between two topologically different insulators, as long as the boundary holds the symmetry that protects the bulk-insulating states. 

These gapless states, existing at the bonduary (egde in 2D, surface in 3D) of the topological insulator, lead to the existence of conducting states with predicted properties unlike any other electronic systems, like a vanishing effective mass and a relativistic (linear) dispersion relation. Effective mass in electronic band in inverted regime is negative. It is nothing unusual, considering that the effective hole mass in a normal semiconductor is negative as well. This is a consequence of the shape of a band, but can be also understood in a relativistic approach. Einstein, in his famous equation stated, that energy is proportional to mass, thus in systems with a negative energy gap the mass should be negative. When the bands tend to continously join with a positive gap insulator at the boundary, the energy gap and the effective mass switches to positivie. The transition has to be smooth, so going from negative to positive value at some point the system has to have a gap closure, when the effective mass collapses as well. At this point the particles have to be described by the relativistic equation with a linear dispersion relation \cite{Zawadzki_Topology}.


\section{This work}
\wciecie
At the beginning this section a history of studies and discoveries related to selected areas of topological insulators will be presented, starting where the previous section finished -- at the first discoveries of 2D and 3D topological insulators. The areas covered will deal with the most prominent features of topological insulators and the development of related technology and experimental techniques. 

The first paragraph will describe the experiments confirming the properties of the edge channels, like conductance, helicity, lack of dissipation, and spin polarization. The second paragraph will focus on the efforts concerning the amplification of the bulk band gap in order to lower its conductivity and to highlight the influence of the edge channels on the total transport properties of a given material. This is related but not limited to the strain engineering, which allows to open the bulk band gap.

Finally, a brief description of phase transitions will be given, which is directly related to the subject of this thesis. 

\subsection{State of the art}
\myparagraph{Edge Channels}
\wciecie
The discovery of the QHE \cite{Klitzing_Topology}, in which the conductance is quantized, was a surprise to the physics community. This effect occurs in layered metallic structures at high magnetic fields. As a result, conducting one-dimensional channels develop at the edges of the sample. In each of the channels the current flows only in one direction and its conductance is quantized, which is a sign of one-dimensional transport \cite{Halperin_State}. Moreover, the current flowing through these edge states is resistant to scattering. The value of the quantum Hall conductance is strictly connected to the number of edge channels in the sample. Before the discovery of QSH insulators, the existence of a state exhibiting the quantum Hall conductance was limited to low temperatures and high magnetic field, which was a formidable obstacle to overcome in terms of possible applications.
   
In QSH phase the conductance of the edge channels is quantized. The time reversal symmetry requires the edge channels to be helical, which means that electrons with spin up and spin down propagate in opposite directions along the edge of the sample with conserved helicity. As a consequence, carriers on time-reversed paths around a non-magnetic impurity in the helical edge interfere destructively, which results in a zero probability of backscattering. This property was predicted by Murakami \cite{Murakami_State}, Kane \cite{Kane_Topology}, and Bernevig \cite{Bernevig_Topology1}. 

A detection of the edge states is an experimentally difficult task. In an ideal QSH phase the current is carried only via the edge states while the bulk is fully resistant. In practice however, the band gap is small, usually a few meV (4 meV for InAs/GaSb QW \cite{Altarelli_BandStructure}\cite{Yang_BandStructure}, 15 meV for HgTe QW). Because of that, assuring the low temperature of measurements is a necessity to prevent thermal excitations of electrons from occurring. Moreover, a processing of samples is required -- only gated structures have the possibility to tune the Fermi level with enough accuracy into the band gap. Growing the structures where the Fermi level intrinsically lies inside the band gap is a virtually impossible task.

König \cite{Konig_Topology} for the first time observed quantized conductance in HgTe QW in nontrivial regime, which was the first indication of the existence of the edge channels. Later, in 2009, the helicity and dissipationless of the channels was confirmed by the study of nonlocal transport on multiterminal devices carried out by Roth \cite{Roth_State} and Büttiker \cite{Buttiker_State}. However, the direct evidence of the spin polarization of helical states was still missing. 

The existence of the spin polarization was confirmed for the first time by Brüne \cite{Brune_State}. By using specially-designed "H-shaped" HgTe QW-based structures, it was possible to detect the spin polarization of the QSH edge states via the inverse spin Hall effect \cite{Hankiewicz_State}\cite{Valenzuela_State}. The investigated structures were in a two-gate configuration, and could be tuned locally from metallic state into QSH state, as the carrier concentration in two legs could be adjusted separately. This configuration allowed the metallic state to act as a source of spin-polarized carriers and the QSH state as a detector, and vice-versa. 

In the case of InAs/GaSb QWs the situation is more complicated. Except of the quantized channel conductance, there are evidences of residual bulk conductivity, even at very low (20 mK) temperatures. Knez, having studied a set of InAs/GaSb samples with various dimensions and length/width ratios, was able to identify the contribution of the edge channel transport with conductance comparable with the expected value. However, the the highest observed resistance was 2-3 times smaller than $\hbar/2e^2$, which can be attributed to the conductivity of the bulk of the order of $10 e^2/ \hbar$ \cite{Knez1_State}. Theoretical investigations of Naveh and Laikhthman \cite{Naveh_State} concluded that even a finite-level broadening due to the carrier scattering could result in non-zero conductivity, even at $T = 0$.  

Later, in 2011, Knez remarked that edge modes persist alongside the conductive bulk and show only weak magnetic field dependence. This decoupling of the edge from the bulk is a direct result of the gap opening, which takes place away from the Brillouin zone center and, as a consequence, there is a large disparity in Fermi vectors between bulk and edge states. This leads to a qualitatively different QSHI phase than in HgTe/CdTe QW, where the gap opens at the zone center \cite{Knez2_State}. By performing magnetotransport measurements, Knez concluded that despite the fact that conductive bulk allows edge electrons to tunnel from one side to another, the probability of this effect to occur is reduced by a large Fermi wave vector mismatch. The probability of scattering of electrons between the edges is increased if a weak disorder or scattering interactions are taken into account. In a theoretical work Zhou \cite{Zhou_State} found that the edge states on the two sides can couple together to produce a gap in the spectrum. As a result, the single electron elastic backscattering of the edge states is no longer forbidden, and the edge states are not protected completely by time-reversal symmetry. 

Up to this moment, because of the strong bulk influence in transport, all the evidence of the quantized edge channels were indirect and unclear. The solution to this problem was to change the transport properties of the bulk, while preserving the conductance of the edge states.
There have been several methods applied so far.  
Suzuki \cite{Suzuki_State} performed a systematic study of a set of specially designed six-terminal small Hall devices with a doping layer of beryllium in the QW barrier. Doping allowed to lower the carrier concentration and place the Fermi level closer to the energy gap. As a result, it was possible to tailor the structure to exhibit conducting edge channels while maintaining the gap in bulk region. Du \cite{Du_State} implemented a Si doping directly in the QW, at the interface of InAs and GaSb layer. Silicon acts as a donor in InAs and acceptor in GaSb, inducing a disorder in the structure. Generally, disorder reduces the transport properties in the structure. However, the edge states are topologically protected in nature, the disorder has a small impact on their existence and transport properties. As a result, the carrier mobility in the bulk is greatly reduced. This idea was followed by Knez \cite{Knez3_State}, who studied InAs/GaSb QW in the disordered regime. A similar concept to suppress the bulk conductivity was implemented by Charpentier \cite{Charpentier_State}. However, his idea was to use a gallium source with charge-neutral impurities, which have a direct influence on the transport properties of the structure. He compared two sources of gallium and obtained a drastically different results in terms of mobility. The sample grown using "low mobility" gallium had more than one order of magnitude lower mobility that the sample grown using "high mobility" gallium. 

A remarkable evidence of the edge channels was provided in 2013 at Stanford by usage of a micrometer SQUID (Superconducting Quantum Interference Device) loop \cite{Huber_State} to directly image the current density in HgTe QW \cite{Nowack_State} and in 2014 InAs/GaSb BGQW \cite{Spanton_State}. It was shown that the current in the sample flows via the edge states only when the structure is in inverted regime, which is presented on Figure \ref{fig:Spanton_State} for the case of InAs/GaSb. The edge conductivity persisted despite the fact that the sample was much bigger than the ballistic limit (around 2 $\mu$m \cite{Knez2_State}), even at higher temperatures (up to 30 K). 

\begin{figure}[ht]
	\centering
	\includegraphics[width=8cm]{gfx/Introduction/State/Spanton.png}
	\vspace{-10pt}
	\caption{\textit{Flux and current maps in a four-terminal device made from a Si-doped InAs/GaSb quantum well. (a) Schematic of the device. Si doping (shown in orange) suppresses residual bulk conductance in the gap. (b) Schematic of the measurement. Alternating current (orange arrows) flows from left to right on the positive part of the cycle. A voltage ($V_g$) applied to the front gate (yellow box) tunes the Fermi level. The SQUID’s pickup loop (red circle) scans
across the sample surface, with lock-in detection of the flux through
the pickup loop from the out of plane magnetic field produced by the
applied current. (c) Four-terminal resistance $R_{14,23} = V_{23}/I_{14}$ as a
function of $V_g$, showing both the upwards (black) and downwards
(gray) gate sweeps. $R_{14,23}$ is maximized when the chemical
potential is tuned into the gap. \cite{Spanton_State}}}
	\label{fig:Spanton_State}
\end{figure} 

\myparagraph{Strain Engineering}
\wciecie
Despite the numerous efforts in increasing the bulk resistivity, the residual conductivity persists even at low temperatures. This limits a possibility to observe the special properties of the edge channels only to cryogenic temperatures for both HgTe/CdTe and InAs/GaSb systems. The energy gap in the case of HgTe/CdTe is around three times larger than in InAs/GaSb, which puts it in the privileged position, as the influence of the bulk on the sample conductance can be neglected at low temperatures \cite{Konig_Topology}. 

In order to decrease the bulk conductivity (by enlarging the band gap) an implementation of strain engineering was proposed. The band gap in InAs/GaSb originates from the hybridization of electron and hole levels. However, the strength of this effect strongly depends on the overlapping of electron and hole wavefunction. Electrons and holes exist in separate layers, thus the spatial separation reduces the overlapping and the hybridization gap in a consequence. The overlap can be improved by making thinner layers, however it can be done only up to a point where the structure has an inverted band ordering. Further decrease of thickness of the layers results in a normal band ordering.

One of the possible solutions to this issue was proposed \cite{Smith_State} by Smith and Maihiot in 1987, as there were studying inverted InAs/GaSb superlattices for infrared detectors. They implemented strain to the structure by alloying GaSb with InSb (which has lattice constant about 6.4 Å). Strain in the growth plane shifts the energy of the conduction band in InAs downwards, while the energy of the valence band in InGaSb splits into heavy hole level and light hole level. The energy of heavy hole level is higher than the original top of the valence band in pure GaSb. As a consequence, the InAs/InGaSb structures can be grown thinner to achieve the stronger overlapping of the bands, while maintaining the same (inverted) energy structure.

This idea was implemented and further developed by Du \cite{Du_State2}. An investigation of samples based on InAs/In$_x$Ga$_{1-x}$Sb with $x = 0.25$ allowed to induce an enchanced interacton of wavefunction of electrons and holes, which resulted in an increase of the hybridization gap from around 4 meV up to even 12 meV. The crystalline stucture remained coherent across the heterostructure interfaces despite the 1.2 \% in-plane strain. As a consequence, transport measurements revealed for the first time an existence of a truly insulating hybridization gap at low temperature \cite{Du_State2}. A similar investigation was carried out by Akiho \cite{Akiho_State}, who used structure with $x = 0.25$ indium content as well but different thicknesses of the layers, and obtained a similar increase of an hybridization gap, estimated to be equal to 10.8 meV. Akiho also presented calculations for different indium contents, and claimed that it is possible to increase the hybridization gap even more, up to 25 meV, for $x = 0.25$ \cite{Akiho_State}, which induces a 2.45 \% strain.

In HgCdTe based systems the intrinsic energy gap is larger than in the case of InAs/GaSb. However, it is still relatively small (usually less than 15 meV), which makes it difficult to tune the Fermi level homogeneously into the gap over the entire structure. Irregularities in the structure composition or layer thickness can result in an appearance of conductive channels, if the Fermi level locally slips out the gap. Moreover, an elevation of temperature can make the observation of quantum effects impossible, due to the thermal excitations. 

The strain engineering technology can be applied to increase the band gap in HgCdTe systems, in both cases of bulk systems and QWs as well. In 2011 Brüne reported \cite{Brune_State} a study of an energy structure dependence on strain in bulk HgTe. Three-dimensional HgTe is expected to have Dirac-like surface states \cite{Pankratov_State}, but because of the fact that the material itself is semimetallic, sufrace states are always coupled to metallic bulk states. The lattice constant of HgTe grown on CdTe is 0.3 \% larger than that of bulk HgTe. The critical thickness required for lattice relaxation for this mismatch is around 200 nm, which means that thinner layers are subjected to a substrate-induced strain. For 70 nm thick layer of HgTe \cite{Brune_State} the application of strain opened up a gap between the light-hole and heavy-hole, implying that strained 3D HgTe is expected to be a 3D TI.

A standard approach to the strain engineering of HgTe involve the usage of MBE grown substrates based on pure CdTe and Cd$_{0.96}$Zn$_{0.04}$Te. Both of those materials have the lattice constant larger than HgTe, which results in a tensile strain in the epilayers. Under such conditions, the largest gap achieved are $E_g = 17$ meV and 25 meV for QWs grown on CdTe and Cd$_{0.96}$Zn$_{0.04}$Te, respectively \cite{Pfeuffer_State}. In order to obtain a larger gap a stronger strain is required. However, highly strained structures often suffer from low crystal quality. The idea, implemented by Leubner \cite{Leubner_State}, was to use strained-layer superlattices based on CdTe-Cd$_{0.5}$Zn$_{0.5}$Te. These superlattices, grown on GaAs substrate, provided a straigthforward control of the strain in the HgTe layers. This allowed to apply a tensile or compressive strain at will. Out of three samples studied by Leubner, two had almost the same thickness ($\approx$ 15 nm, which is well within the inverted regime), but due to the strain their properties differ significantly. The third one was grown thinner ($\approx$ 7.5 nm) and with a different composition of barriers. The comparison of the samples with similar thickness revealed that the strain in the layers primarily affects the shape of the valence band. The tensile strain induced an overlap of the valence band and conduction band, which resulted in a phase transition from topological insulator to topological semimetal, while the sample with compressive strain remained a topological insulator with enlarged band gap up to 17 meV. The band gap in third sample, characterized by even stronger compressive strain, was enlarged up to 55 meV, which is well above the thermal energy at the room temperature ($\approx$ 26 meV at 300 K) \cite{Leubner_State}. 

\myparagraph{Topological phase transitions}
\wciecie
Phase transitions give a special possibility to study topological states of matter, as they grant a direct access to the physics occurring in the both phases and give a clear evidence of the differences between them. The study of phase transitions in topological insulators began with the theoretical work of Bernevig \cite{Bernevig_Topology2}, as he proposed that HgTe/CdTe QW can be tuned via a topological transition from a nontrivial phase to a trivial one just by varying the QW width. The idea is presented on Figure \ref{fig:HgTe_QW_width_PT}. The edge channels with well defined conductivity appear only if the system is in inverted band gap regime, which takes place for a specific range of QW widths. At the critical thickness the phase transition occurs, and the system changes its phase to semiconducting with a positive band gap. This transition causes the edge channels to vanish.

\begin{figure}[ht]
	\centering
	\includegraphics[width=11cm]{gfx/HgTe_QW_width_PT.jpg}
	\vspace{-10pt}
	\caption{\textit{\textbf{Panel A)} Experimental setup on a six-terminal Hall bar showing pairs of edge states, with spin-up states in green and spin-down states in purple. \textbf{Panel B)} A two-terminal measurement on a Hall bar would give $G_{LR}$ close to $2e^2/h$ contact conductance on the QSH side of the transition and zero on the insulating side. In a six-terminal measurement, the longitudinal voltage drops $\mu_2$ -- $\mu_1$ and $\mu_4$ -- $\mu_3$ vanish on the QSH side with a power law as the zero temperature limit is approached. The spin Hall conductance has a plateau with the value close to $2e^2/h$. The image comes from the work of Bernevig, Hughes, and Zhang \cite{Bernevig_Topology2}.}}
	\label{fig:HgTe_QW_width_PT}
\end{figure} 

However, long before that, in the early sixties, there were many works devoted to studies of inverted band structure of bulk HgCdTe systems and its properties. Even back then it was realized that by varying the chemical composition of the structure, mainly the cadmium content, it is possible to change the band order of the system from semimetallic (for low Cd content) to semiconductor (for high Cd content). This subject will be broader discussed in this work. There are more systems, which phase can exhibit a phase transition driven by a variation of chemical composition, like topological crystalline insulator Pb$_{1-x}$Sn$_{x}$Te, which was studied by ARPES measurements by Xu \textit{et al.} \cite{Xu_CTI}.

Phase transition can be induced not only by strain and layer thickness and chemical composition. There are far more factors like electric and magnetic field, temperature, pressure, etc. Some of them will be described in this thesis. Recently, there have been multiple ideas to induce and investigate a topological phase transition. One of the most important is to use temperature as an external parameter driving the HgTe/CdTe QW from topological insulator phase to semiconductor phase by studying electrical properties in both phases \cite{Wiedmann_State}. 

\subsection{Scope of this thesis}
\wciecie
The principle idea behind this thesis is to demonstrate the possibility of investigating the topological insulators and other narrow-gap semiconductors/semimetals by the means of THz spectroscopy. 

The systems under the scope of this work are three kinds of mercury-cadmium-telluride heterostructures, which were chosen accordingly to expose some of the topological properties described before.

Chapter 2 is divided into two parts. The first part introduces the actual methods of investigation. It starts with an overview of the interaction of matter with light. Later general principles of the Fourier spectroscopy are briefly explained with an insight how to read and interpret the results. The second part is where the experimental set-up is described. 

Chapter 3 gives the details about the first investigated systems - a set of two HgTe/CdTe Quantum Wells. These QWs were chosen to investigate the possibility to observe a temperature induced topological insulator-to-semiconductor phase transition.

Chapter 4 overviews the second of investigated systems - a set of four HgTe thick layers with different thicknesses. These systems were designed and grown with a strain, which opened the bulk band gap. This way a semimetallic structure becomes a topological insulator. The thickness of the samples is between 15 nm and 100 nm, which means that they cannot really be described as bulk nor as QW. These systems exhibit a set of surface states... 

Chapter 5 presents the third and the last system - a set of two genuine HgCdTe bulk samples with thickness of few microns. One of the samples had a positive gap, thus is a regular semiconductor. The other is semimetallic at low temperatures and undergoes a temperature induced semimetal-to-semiconductor phase transition. In this chapter the evolution of the band structures of the both samples was investigated and presented.

At the end there is a brief summary with a conclusion and some perspectives and ideas for future work are presented.

\chapter{Optical properties of matter}
\label{chpt:optical}
\wciecie
In this chapter the principal properties of matter, related to optical phenomena are presented. From the point of view of spectroscopy, the interaction between light and solid state systems is crucial. This chapter explains the mechanism of absorption of light inside a solid state system. The absorption mechanism in solids is based on two pieces. The first piece depends on a probability of transition, while the second is based on an optical joint density of states, and its dependence on external conditions like temperature and magnetic field. 

The chapter starts with a derivation of absorption coefficient from the basic laws of electrodynamics (Maxwell's equations) via an introduction of the complex dielectric function and complex optical conductivity. Later, the probability (also called rate or strength) of optical transition is explained via a perturbation Hamiltonian. This probability, related to the Fermi Golden rule, is based on an interaction of quantum states taking part in transitions. 

????Further, the density of states for all dimensional systems is presented for linear and parabolic electronic dispersion relations, and how does it change while magnetic field is applied. All the concept in this chapter are presented in rather qualitative way, under two assumptions. The first assumption is that all the transitions are direct -- they take place at \textbf{\textit{k}} = 0, which is valid since the momentum of a photon is negligible in comparison to the momentum of an electron. A phonon is needed in order to execute a indirect transition, making it a three-body process, thus of much lower probability.???? 

The second assumption is that the fields of an electromagnetic wave are small so the calculations require only terms linear with $\vec{E}$ and $\vec{B}$. This assumption is valid, because the only sources of light intense enough to be considered in terms of nonlinear optics are the strongest lasers, usually working in a pulse mode, which are not in the scope of this work. 
 
\section{Review of Fundamental Relations for Optical Phenomena}
\wciecie
Electromagnetic radiation allows to investigate the energy band structure, impurity levels, lattice vibrations, excitons, localized defects and many more. The studies are based on measuring certain quantities, which manifest themselves via interaction of light with matter. The most important quantities are the dielectric function $\varepsilon (\omega)$ and the optical conductivity $\sigma (\omega)$, which is directly related to the energy structure of solids. 
\subsection{The dielectric function and optical conductivity}
\label{section:Maxwell_equations}
\wciecie
The (complex) dielectric function and the (complex) optical conductivity can be found in the basic equations of electromagnetism -- the Maxwell's equations, given here in CGS units and assuming no charge density:
\begin{eqnarray}
\label{eq:Maxwell_equations}
\begin{aligned}
\nabla \times \vec{H} - \frac{1}{c}\frac{\partial D}{\partial} - \frac{4\pi}{c}\vec{j} = 0 \\
\nabla \times \vec{E} + \frac{1}{c}\frac{\partial B}{\partial t}= 0 \\
\nabla \cdot \vec{D} = 0 \\
\nabla \cdot \vec{B} = 0.
\end{aligned}
\end{eqnarray}
The quantities which represent concepts of dielectric function and optical conductivity can be found in a set of complementary equations:
\begin{eqnarray}
\begin{aligned}
\label{eq:Maxwell_equations2}
\vec{D} = \varepsilon \vec{E} \\
\vec{B} = \mu \vec{H} \\
\vec{j} = \sigma \vec{E}. \\
\end{aligned}
\end{eqnarray}
The wave equations for fields $\vec{E}$ and $\vec{B}$ can be derived from the Maxwell's equations (\ref{eq:Maxwell_equations}) and Equations \ref{eq:Maxwell_equations2}, stating:
\begin{equation}
\label{eq:Wave_E}
\nabla^2 \vec{E} = \frac{\varepsilon \mu}{c^2} \frac{\partial^2 \vec{E}}{\partial t^2} + \frac{4\pi \sigma \mu}{c^2} \frac{\partial \vec{E}}{\partial t}
\end{equation}
and
\begin{equation}
\label{eq:Wave_H}
\nabla^2 \vec{H} = \frac{\varepsilon \mu}{c^2} \frac{\partial^2 \vec{H}}{\partial t^2} + \frac{4\pi \sigma \mu}{c^2} \frac{\partial \vec{H}}{\partial t}.
\end{equation}
The solutions of a wave equations (\ref{eq:Wave_E} and \ref{eq:Wave_H}) should resemble a sinusoidal functions. For the case of $E$ field (for $H$ field is similar):
\begin{equation}
\label{eq:Solution_E}
\vec{E} = \vec{E_0}e^{i(\vec{K}\cdot\vec{r} - \omega t)},
\end{equation}
where $K$ is a complex propagation constant and $\omega$ is the frequency of light. The real part of $K$ represent a wave vector. The imaginary part however, is responsible for the attenuation of the wave by matter. Substitution of a sinusoidal function from Equation \ref{eq:Solution_E} into Equation \ref{eq:Wave_E} gives the relation for $K$:
\begin{equation}
\label{eq:Solution_K}
K^2 = \frac{\varepsilon \mu \omega^2}{c^2}  + \frac{4\pi i \sigma \mu \omega}{c^2},
\end{equation}
which can be rewritten as
\begin{equation}
\label{eq:Solution_K2}
K^2 = \frac{\omega}{c} \sqrt{\varepsilon_c \mu},
\end{equation}
where $\varepsilon_c$ is the actual complex dielectric function.
\begin{equation}
\label{eq:Solution_K3}
\varepsilon_c = \varepsilon + \frac{4\pi i \sigma}{\omega}.
\end{equation}

The relation \ref{eq:Solution_K3} contains an imaginary term with $\sigma$. This expression can be written in terms of the complex optical conductivity $\sigma_c$ as:
\begin{equation}
\label{eq:Solution_K4}
\varepsilon_c = \frac{4\pi i}{\omega}\left( \sigma + \frac{\varepsilon \omega}{4\pi i} \right) = \frac{4\pi i}{\omega} \sigma_c,
\end{equation}
where $\sigma_c$ was defined by
\begin{equation}
\label{eq:Solution_K5}
\sigma_c = \sigma + \frac{\varepsilon \omega}{4\pi i}.
\end{equation}
The complex dielectric function and complex optical conduction defined above can be related to 
\begin{itemize}
\item the observables measured in experiments, like transmission,
\item the properties of the solids like carrier density, relaxation time, energy gaps, effective masses, etc.
\end{itemize}

In vacuum, where $\mu = 1$, $\varepsilon = 1$, and $\sigma = 0$, the electromagnetic wave propagates freely. However, if the electrical conductivity of the medium is finite, the behavior of the wave is governed by the coefficient of refraction
\begin{equation}
\label{eq:Solution_K6}
N = \sqrt{\mu\varepsilon_c} = \sqrt{\mu\varepsilon \left(1 + \frac{4\pi i \sigma}{\varepsilon \omega} \right)} = \tilde{n}(\omega) + i\tilde{k}(\omega),
\end{equation}
where $\tilde{n}$ and $\tilde{k}$ are real and imaginary parts of the coefficient of refraction. The coefficient $\tilde{k}$ is responsible for an exponential decay of an amplitude of a wave, thus it is often called the extinction coefficient. It can be related with the absorption coefficient via a relation
\begin{equation}
\label{eq:Solution_K7}
\alpha (\omega) = \frac{2\omega}{c} \tilde{k}(\omega).
\end{equation}

The absorption coefficient depends on the frequency of the light. It is responsible for energy dissipation of the wave inside a solid. The rate of decay is strongly related to the properties of medium $\mu$, $\varepsilon$, and $\sigma$, which is reflected in the band structure, carrier density, etc.

\subsection{Probability of absorption}
\label{section:oscillator_strength}
\wciecie
The absorption of light by a semiconductor can be described classically in the terms of Beer-Lambert's law. It states that if a beam of light of given intensity $I_0$ penetrates the surface of a solid, the intensity of the light decreases with the penetration depth $z$ as
\begin{equation}
\label{eq:L-B_law}
I(z) = I_0 e^{-\alpha (\omega) z},
\end{equation}
where $\alpha$ is an absorption coefficient of a solid. The absorption coefficient is closely related to a quantum-mechanical transition rate $W_{if}$ given by the Fermi Golden Rule. The Fermi Golden Rule expresses the probability per unit time that a photon of energy $\hbar\omega$ excites an electron from the initial state $\bra{\psi_i}$ to the final state $\ket{\psi_f}$:
\begin{equation}
W_{if} = \frac{2\pi}{\hbar} |\bra{\psi_f}\mathcal{H}_{if}'\ket{\psi_i}|^2 \rho_j(\hbar\omega).
\label{eq:FermiGolderRule}
\end{equation}
%W_{if} = \frac{2\pi}{\hbar} |\bra{\psi_f}\mathcal{H}_{if}'\ket{\psi_i}|^2 \delta [E_f(\vec{k}) - E_i(\vec{k})-\hbar\omega].
In this expression, the matrix element $\mathcal{H}_{if}'$ corresponds to the external optical perturbation, and $\rho_j(\hbar\omega)$ expresses the the joint density of states (DOS) function describing the density of states associated with an energy of the excitation photon ($E = \hbar \omega$). 

The derivation of perturbation Hamiltonian $\mathcal{H}_{if}'$ starts from a standard one-electron Hamiltonian without magnetic field, which takes form of:
\begin{equation}
\label{eq:Perturbed_Hamiltonian}
\mathcal{H}_0 = \frac{p^2}{2m} + V(\vec{r}),
\end{equation}  
where $V(\vec{r})$ is a scalar periodic potential. The momentum is replaced by a term $i\vec{\nabla}$, and in the effective mass approximation the periodic potential is replaced by the effective mass $m^*$. 

The single electron Hamiltonian in the presence of magnetic field changes via a substitution $\vec{p} \rightarrow \vec{p} - (e/c)\vec{A}$, thus the full form of Hamiltonian is the following
\begin{equation}
\label{eq:Perturbed_Hamiltonian2}
\mathcal{H}_{B} = \frac{1}{2m}\left( \vec{p} - \frac{e}{c}\vec{A}\right)^2 + V(\vec{r}) = \underbrace{\frac{p^2}{2m} + V(\vec{r})}_{\mathcal{H}_0} \underbrace{-\frac{e}{mc}\vec{A}\cdot\vec{p}+\frac{e^2 A^2}{2mc^2}}_{\text{perturbation}},
\end{equation}
where the first part resembles the Hamiltonian $\mathcal{H}_0$ from Equation \ref{eq:Perturbed_Hamiltonian}, and the second is a perturbation. The optical fields are usually very weak in comparison with fields inside a crystal, thus only the term linear with $\vec{A}$ remains in a good approximation. The expression for the perturbation Hamiltonian is
\begin{equation}
\label{eq:Perturbed_Hamiltonian3}
\mathcal{H}' = -\frac{e}{mc}\vec{A}\cdot\vec{p}+\frac{e^2 A^2}{2mc^2} \approx -\frac{e}{mc}\vec{A}\cdot\vec{p}.
\end{equation}

The matrix element, $\bra{\psi_f}\mathcal{H}'\ket{\psi_i}$ coupling the initial and final states through the optical fields, determines the strength of optical transitions, which depends on the electromagnetic field perturbation $\mathcal{H}'$.


\section{Density of states and optical transitions}
\label{section:DOS}
\wciecie
The standard density of states can be quickly derived assuming Born-Karman's periodic boundary conditions for the Bloch functions describing electronic states of a finite periodic crystal lattice. Considering that the states in reciprocal space are evenly distributed, and neglecting spin of electrons the \textbf{\textit{k}}-space density of states takes form of
\begin{equation}
\label{eq:DOS_k}
\rho_{D} (k) = \frac{1}{(2\pi)^{D}},
\end{equation}
where $D$ is the dimensionality of the system. Due to the Fermi exclusion principle one state can be occupied only by a pair of electrons with opposite spins. This means that the appropriate DOS functions in three) and two dimensions take look like:
\begin{equation}
\label{eq:DOS_2}
\rho_{3D} (k) = \frac{2}{(2\pi)^{3}}, \rho_{2D} (k) = \frac{2}{(2\pi)^{2}}.
\end{equation}
This function describes the momentum-depended density of states per volume (surface) unit of a reciprocal space of a finite crystal. Knowing that the DOS function is related to the amount of energy states per unit of energy $\rho (E) = dN/dE$, a relation to the reciprocal space can be found by
\begin{equation}
\label{eq:DOS_3}
\rho (E) = \frac{dN}{dE} = \frac{dN}{dk}\frac{dk}{dE},
\end{equation}
where $dN = \rho_D (k) dV_k$. The DOS function $\rho_D (k)$ and an element of volume in \textbf{\textit{k}}-space $dV_k$ have to take forms of a proper dimension. Now only a dispersion relation $E(k)$ is needed to calculate the energy-depended DOS function. Two cases of energy dispersion relation will be described -- parabolic and linear.
 
\subsection{DOS in parabolic band}
\wciecie
In the case of parabolic bands and effective mass approximation the dispersion relation takes form of:
\begin{equation}
\label{eq:DOS_4}
E(k) = \frac{\hbar^2k^2}{2m^*},
\end{equation}
thus
\begin{equation}
\label{eq:DOS_5}
\frac{dk}{dE} = \frac{m^*}{\hbar^2} \frac{1}{k}.
\end{equation}

If the element of volume $dV_k$ is equal to the difference of volumes of two balls (in 3D) or two discs (in 2D) with radii $k+dk$ and $k$, then the amount of states in the element of volume can be expressed in as  
\begin{eqnarray}
\label{eq:DOS_6}
\begin{aligned}
dN_{3D} = \rho_{3D} (k) dV_k = \frac{2}{(2\pi)^3} \cdot 4\pi k^2 dk,\\
dN_{2D} = \rho_{2D} (k) dV_k = \frac{2}{(2\pi)^2} \cdot 2\pi k dk.
\end{aligned}
\end{eqnarray}

Substituting Equations \ref{eq:DOS_5} and \ref{eq:DOS_6} into Equation \ref{eq:DOS_4} gives the DOS function per unit energy for parabolic bands for 3, 2, 1, and 0 dimensions:
\begin{eqnarray}
\label{eq:DOS_7}
\rho_{3D}(E) = \frac{1}{2\pi^2} \left(\frac{2m^*}{\hbar^2} \right)^{\frac{3}{2}} E^\frac{1}{2}, \\
\rho_{2D}(E) = \frac{m^*}{\pi\hbar^2}\sum_i \Theta(E-E_i),\\
\rho_{1D}(E) = \frac{1}{\pi}\left(\frac{m^*}{\hbar^2}\right)^{\frac{1}{2}} \sum_i \frac{\Theta(E-E_i)}{(E-E_i)^{-\frac{1}{2}}} ,\\ 
\rho_{0}(E) = 2\sum_i \delta(E-E_i),
\end{eqnarray}
where $\Theta(E-E_i)$ is the Heaviside function, equals 1 if $E > E_i$, and the summation is over $i$ states. In a 0D all the available states exist only at a set of discrete energies -- the DOS of a 0D system is described by a delta function. 

\subsection{DOS in linear band}
\wciecie
In the case of linear band the dispersion relation takes form of:
\begin{equation}
\label{eq:DOS_lin1}
E(\vec{k}) = \hbar v_f \vec{k},
\end{equation}
thus
\begin{equation}
\label{eq:DOS_lin2}
\frac{dk}{dE} = \frac{1}{\hbar v_f}.
\end{equation}

The dispersion relation does not influence the element of volume $dV_k$, thus the DOS function per unit energy for linear bands for 3, 2, 1, and 0 dimensions:
\begin{eqnarray}
\label{eq:DOS_lin4}
\rho_{3D}(E) = \frac{E^2}{\hbar^3 v_f^3 \pi^2}, \\
\rho_{2D}(E) = \frac{E}{\hbar^2 v_f^2 \pi},\\
\rho_{1D}(E) = \frac{2}{\hbar v_f \pi},
\end{eqnarray}

\myparagraph{Optical Joint Density of States}
\wciecie
Not all states take part in an optical transition. The conservation of energy requires that a difference of energies of available states has to be equal to an energy of an exciting photon $\hbar\omega$. This requires a modification of the calculated DOS functions \ref{eq:DOS_7} by substituting $E$ by a term $\hbar\omega-\Delta E$, where $\Delta E$ is a difference of energy between the final and initial states. Because of that, the function $\rho_j (E)$ was named \textit{optical joint density of states}, and for the 3D case it takes form of:
\begin{equation}
\label{eq:3D_jointDOS}
\rho_j (\omega) = \frac{1}{2\pi^2} \left(\frac{2m^*}{\hbar^2} \right)^{(3/2)} (\hbar\omega-E_g)^{\frac{1}{2}}.
\end{equation}
The only energy range, where there exist physical solutions of Equation \ref{eq:3D_jointDOS}, requires that $\hbar\omega \geq \Delta E$.

\subsection{Interband transitions}
\wciecie
The process called interband optical transition is based on an absorption of a photon by an electron, which results in an excitation of electron into a different energy band. 

There are few rules, which optical transitions have to obey:
\begin{itemize}
\item There is a threshold energy, related to the difference of energies of the initial and final states. Photons carrying lower energies than this threshold are not absorbed as there are no final states available for the electrons to be excited to. The photons carrying higher energy may or may not be absorbed. This depends on the internal band structure and allowed relaxation processes in the system.
\item The transitions are either direct or indirect. The conservation of momentum yields that $\vec{k_v} = \vec{k_c} \pm \vec{k_{\hbar\omega}}$. The momentum of a photon $\vec{k_{\hbar\omega}}$ is no reason to be concerned as its value is few orders of magnitude smaller than the dimensions of the Brillouin zone, thus $\vec{k_{\hbar\omega}}$ can be neglected, rendering  $\vec{k_v} = \vec{k_c}$. However, the transition can still be indirect if a phonon is involved. Nevertheless this process is a three-body process thus its probability is much lower than the probability of a direct transition.
\item A transition occurs from an occupied inital state into an empty final state. If the final state is occupied, due to the Pauli Exclusion Principle, the process cannot take place. Similarly, if the initial state is empty, there is no electron to absorb a photon and the transition does not occur.
\item The optical properties of a system are found by an integration over $k$-space, and the joint density of states is important. Photons are more effective if there are many places in the Brillouin zone, where the energy separation of the initial and final states is equal to the energy of a photon, which is usually not the case. However, the places where the energy difference is right and it does not vary much as a function of \textbf{\textit{k}}, are still good candidates for the transition to take place, as there is a lot of joint states. This usually happens near the band extrema, where $dE/dk = 0$, and the states with $\vec{k} = 0$ make the largest contribution per unit bandwidth.  
\end{itemize}

An example of interband transition is presented schematically in Figure \ref{fig:Absorption1}. In an undoped semiconductor the valence band is fully occupied and the conduction band completely empty. A photon carrying energy $h\nu$ gets absorbed if its energy $\hbar\omega \geq E_g$, where $E_g = E_c - E_v$ is energy gap -- the difference between the energy of the bottom of conduction band $E_c$ and the top of the valence band $E_v$. 

\begin{figure}[ht]
	\centering
	\includegraphics[width=10cm]{gfx/Spectro/Absorption1.png}
	\vspace{-10pt}
	\caption{\textit{Direct absorption process in a typical undoped semiconductor with parabolic bands. \textbf{Panel (a)}: The valence band is fully occupied (full red points) and the conduction band is empty (open blue points). An incident photon has energy $h\mu$ equal to the value of band gap $E_g$. \textbf{Panel (b)}: The photon is absorbed by an electron of the top of the valence band and its energy is used to excite the electron into a state at the bottom of the conduction band. An occupied excited state (full blue point), and a hole (empty red point) is left behind.}}
	\label{fig:Absorption1}
\end{figure} 

In the process of absorption the energy of the photon is transferred into an occupied electronic state in the valence band (schematically shown as a full red point), which results in its excitation into an unoccupied state in the conduction band (open blue point). After the absorption process is complete, the system remains with a hole (the empty state in the valence band) and an excited electron in the conduction band (Panel (b) of Figure \ref{fig:Absorption1}). Later, the electron may or may not relax to a ground state through a recombination process (with the hole in the valence band).

\subsection{Magnetic field}
\label{section:DOS_at_magneticField}
\wciecie
The presence of magnetic field causes changes in the movement of electrons in a solid, forcing the energy bands to split into a set of levels. If the magnetic field is applied in the $z$-direction, only the states in $x,y$ plane are affected. The two-dimensional Hamiltonian takes form:
\begin{equation}
\label{eq:LL0}
\mathcal{H} = \frac{\hat{\mathbf{P}}^2}{2m}.  
\end{equation}
The canonical momentum $\mathbf{\hat{P}}$ is 
\begin{equation}
\label{eq:LL1}
\hat{\mathbf{P}} = \frac{\hbar}{i}\nabla + \frac{e}{c}\mathbf{A}.  
\end{equation}
The magnetic potential is, assuming the Landau gauge, expressed by $\mathbf{A} = (0, Bx, 0)$. The Schrödinger equation takes form:
\begin{equation}
\label{eq:LL2}
\mathcal{H}\psi(\vec{r}) = \frac{\hbar^2}{2m} \left[ -\nabla_x^2 + \left( \frac{1}{i}\nabla_y - \frac{eB}{\hbar c}x \right)^2 \right] \psi(\vec{r}) = E\psi(\vec{r}).
\end{equation}
The choice of gauge grants that the Hamiltonian is independent of $y$, thus $x, y$ wavefunctions can be separated
\begin{equation}
\label{eq:LL2.5}
\psi (x,y) = e^{iky}\phi (x). 
\end{equation}
Moreover, by separating the wavefunction, the result is a one-dimensional Schrödiger equation $\mathcal{H}_x \phi(x) = E \phi(x)$, with the effective Hamiltonian:
\begin{equation}
\label{eq:LL3}
\mathcal{H}_x = \frac{\hbar^2}{2m} \left[ -\nabla_x^2 +(x-x_k)^2 \right].
\end{equation}
This Hamiltonian expresses a one-dimensional harmonic oscillator centered at $x_k = l_B^2 k$, where $l_B$ is the magnetic length. Its solution is a set of energy levels, called Landau Levels (LLs), which energy can be described by
\begin{equation}
\label{eq:LL4}
E_n = \hbar\omega_c\left( n+\frac{1}{2} \right),
\end{equation}
where $\omega_c = \frac{eB_z}{m_c^*}$ is the cyclotron frequency, and $n$ = 0, 1, 2... is an integer quantum number corresponding to different LLs. The Equation \ref{eq:LL4} is valid for a 2D system. In a 3D case the energy of LLs takes form $E_{n,z} = E_n + E(z) = \hbar\omega_c\left( n+\frac{1}{2} \right)+ E(z)$.

The amount of independent states in a system of dimensions $L_x \times L_y$ can be estimated using the boundary conditions for the function $\psi$ (Equation \ref{eq:LL2.5}) in the $y$-direction:
\begin{equation}
\label{eq:LL5}
k = \frac{2\pi}{L_y}m_y,
\end{equation}
for any integer $m_y$. This means that the allowed values of $x_k$ are separated by $\Delta x = l_B^2 \Delta k = 2\pi l_B^2/L_y$. If $L_y \gg l_B$, which is not the case only for 1D systems, then $\Delta x \ll l_B$ so the energy separation of successive states is much smaller than the width of each. The total number of states is equal to $(L_x L_y)/2\pi l_B^2$ and each LL has $1/(2\pi l_B^2)$ states per unit area.

\subsubsection{DOS in magnetic field}
\wciecie
As seen from Equation \ref{eq:LL1}, magnetic field influences the 3D allowed states only in a plane perpendicular ($k_x,k_y$) to the applied magnetic field direction ($z$). This results in a collapse of DOS function of $k$-space into a set of concentric tubes parallel to $\vec{B}$. It can be interpreted as a "reduction" of available DOS by two dimensions -- from 3D to 1D, which is presented in Figure \ref{fig:DOS1}. 

In 2D the quantization in $z$-direction is given by the quantum well, so the allowed energy states are fully quantized, as in a 0D quantum dot system. This results in an appearance of a distinct LL ladder of states in form of delta functions. However, in real-life systems those levels are broaden due to the scattering effects.
\begin{figure}[H]
	\centering
	\includegraphics[width=12cm]{gfx/Spectro/DOS1.png}
	\vspace{-10pt}
	\caption{\textit{\textbf{Panel (a)} Electron energy bands for a 3D solid as a function of the $z$-direction wave vector for different Landau levels ($n$ = 0, 1, 2...). \textbf{Panel (b)} density of states function for the Landau levels compared with the free electron gas for the case $B$ = 0. Image comes from the work \cite{Martinez_Nanotechnology}.}}
	\label{fig:DOS1}
\end{figure}



\subsection{Fermi level}
\label{section:Fermi_level}
\wciecie
%The Fermi level is the total electrochemical potential for electrons in a solid. It describes the thermodynamic work necessary to add one electron to the system. In the framework of band structure theory the Fermi level can be understood as an energy level of an electron, which, at the state of thermodynamic equilibrium, would have a 0.5 probability of being occupied at given time.

From the optical properties of a solid point of view, the Fermi Golden Rule (Equation \ref{eq:FermiGolderRule}) only describes the strength (or probability) of transitions between two states. However, if a transition is to be executed, the initial state has to be occupied by an electron, which can be excited by a photon. Moreover, the final state has to be unoccupied, as the Pauli exclusion principle forbids two electron to occupy the same quantum state. 

The total concentration of electron in a solid has to be distributed on available states. This distribution is governed by two factors -- the DOS function (Equation \ref{eq:DOS_3}) and, as the electrons are fermions, the Fermi-Dirac distribution:
\begin{equation}
\label{eq:Fermi_distribution}
f(E) = \frac{1}{1+exp\left(\frac{E-\mu}{k_B T}\right)},
\end{equation}
where $\mu$ is the chemical potential, and $k_B$ is Boltzmann's constant. The chemical potential is equal to the Fermi energy at $T = 0$ K.

\subsubsection{Temperature effects}
\wciecie

As stated in Equation \ref{eq:Fermi_distribution}, the Fermi distribution depends on temperature. For $T$ = 0 K it resembles a step function -- all states below the Fermi level are occupied with probability 1, and all states above are completely empty, as presented in Figure \ref{fig:Fermi_Distribution} in black curve. As the temperature increases, the Fermi distribution gets smoother -- the population of electron states of energy above $E_f$ increases at the expense of electron states of energy below $E_f$. This effect has a profound consequences in the optical properties of a solid, as they vary with a change of temperature.

\begin{figure}[ht]
	\centering
	\includegraphics[width=10cm]{gfx/Spectro/Fermi_Distribution.png}
	\vspace{-10pt}
	\caption{\textit{Fermi distribution for various temperatures in range $0 K \leq T \leq 300 K$. The Fermi level $E_f$ = 0 meV. The horizontal grey dashed line marks the probability 0.5.}}
	\label{fig:Fermi_Distribution}
\end{figure}

At the beginning of this section (\ref{section:Fermi_level}) a concept of Fermi level and electron distribution was given. The way how the electrons are distributed on the bands is crucial for the shape of an absorption spectra. This means that even if the given transition's strength is high, and there are sufficient electron states related to the transition (which is governed by the joint density of states), but the electron distribution is not right, the transition does not occur. The right electron distribution requires the initial state to be occupied while the final state must be empty. 

Figure \ref{fig:Transitions_vs_Ef} illustrates this concept. It shows an example of LL structure (two levels) and an potentially observed absorption line due to the transitions between those LLs. Panel (a) shows the Fermi distribution of electrons at $T$ = 0 K with the position of chemical potential is marked with a purple line, which also extends into Panel (b). This means that only the electron states with $E \leq E_f$ (lower than the purple line) are occupied, which is represented as thick lines. The unoccupied states are represented by a thin line. The whole magnetic field range is divided into three regions A, B, and C. Panel (c) shows the peak of expected transition line. At given magnetic field its value is equal to the difference of energies of LLs from panel (b) at the same field. The shape of the transition is linear, as the difference of linear LLs is linear as well. In a case of LLs with a square-root relation $E \propto \sqrt{B}$, the transition shape would be square-root as well. 

\begin{figure}[ht]
	\centering
	\includegraphics[width=12cm]{gfx/Spectro/Transitions_vs_Ef.png}
	\vspace{-10pt}
	\caption{\textit{\textbf{Panel (a)}: The Fermi distribution at $T$ = 0 K. \textbf{Panel (b)}: Plot of two Landau levels divided into three regions, depending if a transition between those Landau levels can occur. The chemical potential is marked with a horizontal purple line. Region 1 -- the transition is forbidden due to the lack of empty final states (red arrow). Region 2 -- the transition is allowed (black arrow). Region 3 -- the transition in not possible due to the lack of carriers in initial states (red arrow). \textbf{Panel (c)}: Position of expected transition in an absorption spectra.}}
	\label{fig:Transitions_vs_Ef}
\end{figure}

The assumption is made that the transition strength is high at the whole range and there are enough states in both initial (blue line) and final (green line) states. 

\myparagraph{Region A:}
\wciecie
The transition is forbidden by the Pauli principle, which is symbolized by a red arrow. Despite the fact that initial states are occupied, there are no empty states at the final level, as all states are occupied as well. This is why there is no observed transition in this region (Panel (c)).

\myparagraph{Region B:}
\wciecie
At the beginning of region B the higher in energy LL (green) crosses the chemical potential, which means its electronic states are no longer occupied (thin line). This renders the transition possible, which is represented by a black arrow. The position of the transition is plotted on Panel (c). The intensity of transition is the same in the whole region.

\myparagraph{Region C:}
\wciecie
At the beginning of region C the lower in energy LL (blue) crosses the chemical potential, which means that both LLs are not occupied. If there are no occupied states in initial states there are no electrons which can interact with a photon and be excited to higher level. The transition is not possible, which is marked with a red arrow. Also, there is no absorption marked on Panel (c).
 
The description above explains the requirements which have to be met in order the transition to be observed. An appearance of a transition and its disappearance give an important information about the position of chemical position, which usually has to be confirmed via transport measurements.

The situation slightly changes if temperature is elevated. The Fermi distribution changes shape, as presented in Figure \ref{fig:Transitions_vs_Ef_vsT} (Panel (a)). This means that not all states with $E < E_f$ are occupied and not all states with $E > E_f$ are empty, especially those close to $E_f$. Instead, the probability of occupation decreases smoothly with increasing energy (and with magnetic field -- especially in the case of LL structure as in Figure \ref{fig:Transitions_vs_Ef_vsT} or other LL structures, where $E(B)$ is a monotonic function). This effect is schematically represented with horizontal solid grey lines, which mark the region of energy, where both LLs are partially occupied. Because of this the Region B is broader in comparison to the case of $T = 0$ K.
\begin{figure}[ht]
	\centering
	\includegraphics[width=12cm]{gfx/Spectro/Transitions_vs_Ef_vsT.png}
	\vspace{-10pt}
	\caption{\textit{Obsadz}}
	\label{fig:Transitions_vs_Ef_vsT}
\end{figure}

\myparagraph{Region A:}
\wciecie
Region A is similar to the $T = 0$ K case. All states on both LLs are occupied so the transition is forbidden due to the Pauli exclusion principle. 

\myparagraph{Region B':}
\wciecie
As soon as the higher (green) LL passes the bottom grey line, the probability of occupation decreases and some empty states emerge. Electrons from lower (blue) LL can be excited to those empty states, which results in an appearance of absorption line. The amount of absorbed light, thus the value of the maximum of absorption line, is lower that in the case of $T = 0$ K, as the amount of available electrons is lower. As the energy increases the electron distribution shifts -- more empty states appear in higher LL and the transition gains intensity. 

As the magnetic field increases, the lower LL passes the bottom grey line. This means that some empty states start to appear. This potentially would decrease the intensity of the transition, because there is less electrons to excite. However, at the same time, there is still more occupied states than the empty states. The amount of empty states increases thus the intensity of the transition is more or less the same.

In the middle of region B all of the states in higher LL are empty, and with increasing magnetic field the amount of occupied states in lower LL decreases -- the intensity of the transition decreases as well.

At the magnetic field of value close to the end of region B, there are only few occupied states in the lower LL, to the transition is very weak. Despite this, the range where the transition was observed is still broader than in the case of $T = 0$ K.

\myparagraph{Region C:}
\wciecie
In the region C of the magnetic field, as both LLs are higher in energy than the top grey line, and the amount of occupied states in lower LL is negligible, there are no electrons to absorb the light. This means that the transition vanishes (Panel (c)).

\section{FIR Spectroscopy}
\wciecie

%Spectroscopy can be performed in a broad range of electromagnetic spectrum, because it is based on absorption of radiation due to the quantization of energy levels, which takes place at a broad scale. For example, matter can interact with radiation of mid-infrared and far-infrared range differently. Many kinds of molecules, both organic and inorganic, can potentially absorb a far-infrared photon and convert its energy into a vibration mode. However, absorbing an energetically lower photon of microwave range, can increase rotational energy of the molecule, which is commonly used in, for example, a microwave oven. Visible light can be used to determine the band structure of regular semiconductors, as their band gap usually lies in near-infrared up to UV ranges.

%The same principle applies to the spectroscopy of topological insulators (TI) and narrow-gap semiconductors (NGS). However, the energy of the light has to be adapted to the band structure. This means that only infrared and THz light can probe the band gap of TIs and NGS. Moreover, as magnetic field is applied, an additional quantization appears in a form of Landau levels. At reasonably low magnetic field the energy difference between LLs is comparable with the band gap, and can be also probed using THz light.
Spectroscopy is a branch of physics that deals with the investigation of the interactions between radiation and matter. These interactions include absorption, reflection, emission, or scattering. The term \textit{radiation} does not refer only to photons -- electromagnetic (EM) radiation, it also involves other types of particles, like neutrons, electrons, and protons, which also can be used to study matter. From the point of view of this thesis however, only EM radiation is relevant, or rather a small part of EM spectrum -- infrared/terahertz radiation.

The whole EM spectrum includes all the frequencies -- from low energetic radio waves ($\approx 10^0$ Hz) to high energy gamma rays (up to $10^22$ Hz). There is no a single spectroscopic method which allows to study light and matter at the whole EM spectrum. As the properties of matter differ with the frequency,  different spectroscopic techniques have to be used to investigate specific physical phenomena in given spectral region. For example in solid state physics the electronic energy levels span over a wide energies in a range from few meV  for narrow-gap semiconductors (NGS) and topological insulators (TI) to few eV for regular semiconductors and insulators. This corresponds to energies of photons from far-infrared (FIR), through visible, to ultraviolet range.

The most interesting region of the whole electromagnetic spectrum, considering this thesis, is the THz region. In general the THz region spans over the frequency domain between 0.3 THz to 300 THz. According to generally accepted convention, THz range overlaps with both FIR and mid-infrared regions. An adjacent region of radiation, with lower energies than FIR, is called microwaves. Consequently, an adjacent region of radiation with higher energies is called near-infrared. 

Covering of the whole THz range is a difficult task mostly because of the properties of optical systems like beamsplitters, windows, and, most importantly, radiation sources. This is why the experiments done in our set-up were focused on a more narrow range of 0.3 -- 30 THz, which translates to 10 -- 1000 cm$^{-1}$, or $\approx 1.25$ -- 125 meV. This energy range covers most of the possible inter- and intra-LL transitions in NGS and TI, which allows to investigate their band structure in a precise way.   

\subsection{Spectrum}
\wciecie
A spectrum is a plot of measured light intensity versus wavelength. Usually the $y$-axis of a spectrum represents absorbance or transmittance. Absorbance is a measure of the amount of light absorbed by a sample. It is usually characterized by a positive peaks in a spectrum. Transmittance, on the other hand, is a measure of the amount of light transmitted through a sample. It is characterized by negative minima or "dips" in a spectrum. Absorbance and transmittance are directly related to each other via the energy conservation.

Spectra are used to characterize samples and give an insight to their energetic structure. The peak positions in an FIR spectrum correlate with an optical transitions between distinct energy levels within sample. The standard way to obtain a transmission spectrum is to acquire a spectrum with respect to some reference. A reference spectrum allows to remove features, which are related to the experimental set-up itself, and do not provide any useful information about the sample. For example, in magneto-spectroscopy, a spectrum taken at zero magnetic field ($T_{B=0T}$) can serve as a reference to a different spectrum taken at nonzero magnetic field ($T_{B\neq 0T}$). The obtained spectra are influenced by all the optical parts of the set-up, thus a formula for the measured reference spectrum can be expressed as 
\begin{equation}
\label{eq:Spectra1}
f_{B=0T}(\hbar\omega) = E_{src}(\hbar\omega)\cdot T_{s-u}(\hbar\omega)\cdot T_{sample,B=0T}(\hbar\omega)\cdot S_{det}(\hbar\omega),
\end{equation}
where $E_{src}$ is the emission spectrum of the source, $T_{s-u}$ is the transmission of the elements of the experimental set-up, $T_{sample}$ is the actual transmission through the sample, and $S_{det}$ is the sensitivity of detector. All of those parameters depend on the frequency of light. A similar formula can be written for a spectrum at nonzero magnetic field
\begin{equation}
\label{eq:Spectra2}
f_{B\neq 0T}(\hbar\omega) = E_{src}(\hbar\omega)\cdot T_{s-u}(\hbar\omega)\cdot T_{sample,B\neq 0T}(\hbar\omega)\cdot S_{det}(\hbar\omega),
\end{equation}
where all but one ($T_{sample,B\neq 0T}$) elements are the same. This is obvious since magnetic field influences only the sample. By a division of spectra at nonzero magnetic field (Equation \ref{eq:Spectra2}) by the reference spectra (Equation \ref{eq:Spectra1}), the final transmission is obtained via a relation
\begin{equation}
\label{eq:Spectra3}
T(\hbar\omega) = \frac{f_{B\neq 0T}(\hbar\omega)}{f_{B=0T}(\hbar\omega)} = \frac{E_{src}(\hbar\omega)\cdot T_{s-u}(\hbar\omega)\cdot T_{sample,B\neq 0T}(\hbar\omega)\cdot S_{det}(\hbar\omega)}{E_{src}(\hbar\omega)\cdot T_{s-u}(\hbar\omega)\cdot T_{sample,B=0T}(\hbar\omega)\cdot S_{det}(\hbar\omega)}.
\end{equation}
this division allows to explicitly remove parts depending on $E_{src}$, $T_{s-u}$, and $S_{det}$. Only the parts of the transmission related to the sample remain
\begin{equation}
\label{eq:Spectra4}
T(\hbar\omega) = \frac{T_{sample,B\neq 0T}(\hbar\omega)}{ T_{sample,B=0T}(\hbar\omega)}.
\end{equation}

To obtain a complete evolution of optical transition as a function of magnetic field, this procedure has to be repeated at different values of magnetic field, with a proper resolution. There are experimental set-ups, where the detector is situated close to the sample, where magnetic field can have an influence on the detector as well. In these systems, the parameter $S_{det}$ does not reduce itself in Equation \ref{eq:Spectra3}, thus additional reference spectra have to be obtained to eliminate the magnetic field dependence of the detector on the signal.

An example of infrared spectra, as well as the transmission obtained via the described procedure, is presented in Figure \ref{fig:Spectra}.

\begin{figure}[ht]
	\centering
	\includegraphics[width=14cm]{gfx/Spectro/Spectra.png}
	\vspace{-10pt}
	\caption{\textit{\textbf{Panel (a)}: Transmission spectra calculated from spectra from panel (b). \textbf{Panel (b)}: Spectra obtained at $B$ = 0 T (blue) and $B$ = 8 T (red). The minima present in both spectra are explained by the absorption on the parts of experimental set-up, phonons, impurities. The minima present only in red spectrum correspond to optical transitions marked as $T1$ and $T2$. Each minimum on transmission spectra was connected to a corresponding minimum of the raw spectra.}}
	\label{fig:Spectra}
\end{figure}

Panel (b) of Figure \ref{fig:Spectra} shows two curves, which are raw spectra obtained by a spectrometer. Both spectra are presented as a plot of counts versus energy, where counts correspond to the intensity of light detected at given energy of light. The blue spectrum was taken at zero magnetic field, while the red spectrum at $B$ = 8 T. The spectra are similar at the whole energy range, except at the vicinity of two energies, 340 cm$^{-1}$ and 475 cm$^{-1}$, where red spectra exhibit a visible minimum, absent in blue spectra. Those minima correspond to optical transitions marked as $T1$ and $T2$. 

Those minima are linked with corresponding minima of a transmission spectra presented on Panel (a). The transmission spectrum was obtained by a division of the red spectrum ($T_{B=8T}$) by the blue spectrum ($T_{B=0T}$), and it presents five absorption peaks (with one broad around 260-320 cm$^{-1}$). 

The minima on red and blue spectra, which are related to the sample itself have their corresponding minima on the transmission spectra:
\begin{itemize}
\item Two minima around 120 and 150 cm$^{-1}$ correspond to absorption on impurities and defects of the sample. As they are present at both magnetic fields, they potentially could be removed from transmission spectrum by division. It is not the case because the absorption is almost 100\%, which means that the intensity detected is very small, and it results in such artifacts in division of two very small numbers.
\item The broad absorption around 260-320 cm$^{-1}$ corresponds to absorption by the phonon bands. It remains visible in transmission spectra because of the same reason -- the absorption is almost 100\% leading to division artifacts.
\item The actual optical transitions $T1$ and $T2$, which resulted from an evolution of energy levels in magnetic field and transitions between them.
\end{itemize}

The rest of minima are absent in transmission spectra, which means that they are related to the absorption by the experimental set-up, rather than by the investigated sample.

\clearpage
\section{Experimental set-up}
\wciecie
The experimental work of this thesis is based on the infrared/THz magnetospectroscopy measurements of TI and NGS. The spectroscopy experiments were performed using a specially customized Oxford liquid helium cryostat coupled to a Bruker Fourier spectrometer IFS 66v/S. The schematics of the system is presented on Figure \ref{fig:Experimental_setup}. At the bottom of the cryostat a chamber was placed (white space), separated from the rest of the system by a diamond window, which is well transparent in infrared range and isolates thermally the chamber itself. A composite germanium bolometer QGEB/X was placed inside the chamber, which is cooled by liquid helium. A presence of liquid helium assures that the bolometer is kept at low temperature. Sensitivity of the bolometer is strongly related to operational temperature, and decreases drastically with increasing temperature. 

A thermal separation provided by a diamond window assures the optimal environment for the bolometer while the temperature can be varied in the vicinity of the sample in a broad range. The sample order is supplied with a temperature sensor and a heater. A combination of these two devices is used to set the temperature at the sample space on demand up to around 140 K. Higher temperatures are difficult to obtain due to the thermal radiation emitted by the sample and its surroundings, which heats up the bolometer decreasing its sensitivity. The temperature in the cryostat can be decreased to around 1.6 K, by lowering the liquid helium pressure. At ambient pressure the boiling temperature of helium is equal to 4.2 K. At the $\lambda$ point helium becomes superfluid -- at around 22 mbar the temperature reaches around 1.8 K, which is the lowest possible in this experimental set-up.

\begin{figure}[ht]
	\centering
	\includegraphics[width=10cm]{gfx/Spectro/Setup.png}
	\vspace{-10pt}
	\caption{\textit{Obsadz}}
	\label{fig:Experimental_setup}
\end{figure}

A system of two superconductive coils was embedded into the cryostat \ref{fig:Experimental_setup}. The main coil (Coil 1) is capable of creating a constant and homogeneous magnetic field of inductance up to 16 T in the vicinity of the sample. At the same time, the second coil (Coil 2) compensates the magnetic field created by the Coil 1 in the vicinity of the bolometer. This procedure is required to maintain zero magnetic field at the position of the bolometer. This system allows to perform measurements in a broad range of temperatures and magnetic fields, while preserving optimal environment for the bolometer.

In the experiment a globar lamp integrated with the spectrometer was used as a radiation source. Globar is a broadband thermal (blackbody) emitter, which is typically used for infrared spectroscopy. It is a silicon carbide rod heated electrically up to couple of hundreds degrees Celsius. It radiation is suitable for spectroscopy as it is continuous and resembles the blackbody radiation.

The emitted radiation passes through the spectrometer, where it becomes...


The radiation is delivered to the cryostat via a waveguide, which ends with a light-focusing cone. The focused light passes through and interacts with the sample and the unabsorbed light is detected by the bolometer. An electrical response of the bolometer is passed back to the spectrometer, and the final spectra are calculated.


\chapter{HgCdTe bulk systems}
\label{chpt:HgCdTe bulk systems}
\wciecie
In this work a significant attention will be paid to 3D bulk systems, 2D quantum well heterostructures, and strained HgTe layers with intermediate thickness. This variety of systems gives a unique opportunity to study different physical effects taking place in the structures. 

The bulk structures are the subject of this Chapter. A Hg$_{1-x}$Cd$_x$Te system can be either a semiconductor or a semimetal, depending on its both internal and external parameters. These systems do not exhibit a topological insulator phase, despite the fact that the band structure is inverted, because there is no energy gap separating the bands. However, it will be shown that it is possible to demonstrate a phase transition from inverted to regular band order, which makes these systems particularly interesting. In the bulk systems in a gapless state a new class of relativistic excitations arises, called Kane fermions, which can be studied using THz spectroscopy.

The description and experimental results on MCT QWs will be presented in Chapter \ref{chpt:MCT_QW}. The band structure of QWs might exhibit a topologically insulating phase, as there is a way to obtain an inverted band structure with an energy gap. That gap is provided by the quantum confinement within the QWs itself. 
So far there were numerous studies of MCT QWs with a different QW widths, showing the band gap evolution, as well as a change of topological phase, when the critical thickness is surpassed. In this thesis a different approach is used to induce a phase transition. As in the case of bulk system, instead of using an internal parameter like QW width, an external parameter is used to induce a topological phase transition -- temperature.
At the point of a phase transition, when the band gap vanishes, the QW hosts a different kind of relativistic particles -- Dirac fermions. Their presence can be investigated by THz spectroscopy as well. 

The MCT thick layers are described in Chapter \ref{chpt:MCT_TL}.  



\section{Introduction to HgCdTe systems}
\wciecie
The Mercury-Cadmium-Telluride (MCT) alloy crystal is formed of II-VI compounds which crystallize in zincblende structure, which consists of two face-centered cubic sublattices. In the zincblende structure each of Te-ions have four nearest neighbors, which in the alloy can be either Hg or Cd. The presence of a different atom on each lattice site breaks the inversion symmetry, which results in reducing the point group symmetry from cubic to tetrahedral. 

The both compounds, HgTe and CdTe, are well lattice-matched, having the lattice constant parameter equal to 6.45 Å and 6.48 Å respectively. Hg$_{1-x}$Cd$_x$Te mixed crystals have a direct band gap, which value varies from 1.6 eV for pure CdTe, a relatively large gap semiconductor to -0.3 eV for pure HgTe, a semimetal. The negative band gap is a consequence of the unusual band alignment in the crystal, where $p$-type ($\Gamma_8$) bands lie around 0.3 eV above the $s$-type ($\Gamma_6$) bands. This is caused by an extraordinary large SOC in HgTe (due to the presence of a heavy element Hg), which leads to an inverted band structure. The light-hole $\Gamma_8$ band forms the conduction band, the heavy-hole band forms the first valence band, and the $s$-type $\Gamma_6$ band is pulled below the Fermi level and lies between the heavy-hole band and the spin-orbit split-off band $\Gamma_7$. The band order of both CdTe and HgTe is presented on Figure \ref{fig:BandStructure_HgTe_CdTe}.

\begin{figure}[ht]
	\centering
	\includegraphics[width=10cm]{gfx/BandStructure_HgTe_CdTe.png}
	\vspace{-10pt}
	\caption{\textit{Inverted band order of HgTe and normal band order of CdTe. Figure comes from work \cite{Bernevig_Topology2}.}}
	\label{fig:BandStructure_HgTe_CdTe}
\end{figure} 

\clearpage
\section{Overview of MCT Bulk Crystals}
\wciecie
The earliest studies of Hg$_{1-x}$Cd$_x$Te crystals were aimed at the development of infrared detectors, especially radar applications. In 1958, Lawson \textit{et al.} syntesized for the first time a mixed crystal of HgCdTe at the Royal Radar Establishment in England. A year later his work was published \cite{Lawson_MCT}. Because of its extraordinary properties, HgCdTe was early recognized as the most versatile material for detection application over the whole infrared range, with a special attention put on the wavelength of around 10 $\mu$m. This is the range of the second wide atmospheric window, which means it is of great interest for communication applications. Moreover, it covers the range of the maximum of thermal radiation at room temperature, which opens a way for possible applications for everyday life. 

Recently, as the growth techniques have gradually advanced over the years, the quality of structures based on HgCdTe improved significantly. Especially, a Molecular Beam Epitaxy (MBE) method gives a possibility to grow structures of immense quality. Despite this, growing Hg-based structures is still a challenging task and require a lot of experience and know-how to be done properly. That is why there are only few laboratories in the world, like CEA-LETI, where the growth process was mastered.

MBE method is an epitaxial method of growing crystalline films on top of a monocrystal substrate. MBE takes place in a ultra-high vacuum environment (around $10^{-10}$ mbar). The vacuum prevents deposition of unwanted molecules, reducing the amount of non-intentional impurities. In a solid source MBE, pure elements are heated separately in effusion cells or electron beam evaporators until they start to slowly sublimate. The gaseous elements are delivered to the wafer, where they condensate. They may also react with each other, creating mixed crystals. The quality of the growth can be controlled in-situ by reflection high energy electron diffraction (RHEED), which provides information of the thickness of each layer, down to a monolayer of atoms.  

Nowadays, because of the possibilities given by high quality growing methods, the applications of HgCdTe compounds is no longer limited to the infrared detection. There has been a renewal of interest of physical community in this topic as the 2D TI was demonstrated and later, the discoveries of 2D Dirac Fermions and 3D TI followed. The attention is now centered at the phenomena relative to the fundamental science of narrow-gap semiconductors. These phenomena are, including but not limited to, the variations of effective mass of electrons, appearance of new quasi particles in semiconductor systems and new exotic phases of matter that they are related to. 

The electrical and optical properties of Hg$_{1-x}$Cd$_x$Te crystals are determined by the band structure. The shape of both electron and hole bands are described by the Kane model \cite{Kane_Model}. The band gap varies continously and almost linearly with the Cadmium content $x$ -- the crystal composition. It means that at some point there must be a special composition, where the band gap vanishes, which, according to \cite{Zawadzki_Topology}, implies that effective mass collapses as well. Crystals close to this composition are ideal materials to study various quantum  effects.

MCT bulk crystals give a special opportunity to realize and investigate a condensed matter system with particles exhibiting relativistic Dirac-like properties in all three dimensions. Three dimensional topological insulators also exhibit states with a relativistic dispersion relation, but their presence is limited to the 2D surfaces of the sample.

Recent experimental \cite{Orlita_MCT} and theoretical \cite{Malcolm_MCT} works on Hg$_{1-x}$Cd$_x$Te crystals close to the critical cadmium concentration led to a discovery of another massless Dirac-like quasiparticle, called Kane fermion \cite{Orlita_MCT}. These three dimensional particles are not equivalent to any other known relativistic particles. Kane fermions show a resemblance to the pseudospin-1 Dirac-Weyl system \cite{Malcolm_MCT} -- the band gap vanishes \cite{Weiler_MCT} and their energy dispersion relation forms a Dirac cone with an additional band crossing the vertex. These conical bands may have several spectacular properties similar to those in Dirac and Weyl semimetals (such as Klein tunnelling and suppressed backscattering) \cite{Orlita_MCT}. 

\subsection{Band Structure and the temperature}
\wciecie
Hg$_{1-x}$Cd$_x$Te band structure and electronic dispersion relation can be varied both intrinsically and externally. Intrinsically -- by changing the chemical composition. Externally -- by varying external parameters, like temperature \cite{Capper_MCT} or pressure \cite{Krishtopenko_pressure}. 

However, the method of variation of chemical composition of band structure engineering has its limitations. First of all, it can be done only once, during crystal growth, and secondly, it is technologically difficult. Even small composition fluctuations prevent the ability of fine tuning of the band gap in the vicinity of the critical value, where the band gap vanishes and a topological semimetal-to-semiconductor phase transition takes place.

There is a need to find an easy controllable external parameter, which allows to fine tune the band structure, and is usually available even in a simple experimental set-up. It turns out that temperature regulation allows to precisely control the band structure and provides a method of investigation of the relativistic properties of Kane fermions, while the system is tuned across the gapless state at the phase transition \cite{Teppe_MCT}. The only major drawback of the usage of temperature as a tuning parameter is its range. There are two general factors limiting the range of available temperatures. 

The first limit is related to the material itself. Mercury is a very volatile element, which tends to diffuse and alter the sample at high temperatures, which reduces the quality of the structure. This limitation is a reason, why processing of mercury-based compounds is a very challenging task. In general, the highest safe temperature should be around $\SI{80}{\degreeCelsius}$ \cite{Daumer_MCT_temperature}.

The second limit is related to the energy range that one wants to investigate. If a desired phenomenon, like an optical transition, is in energy range comparable with the thermal energy, it will not be observed. Moreover, an increasing temperature usually follows an increase of entropy within the sample, reducing the signal-to-noise ratio of measurements.

This renders the available temperature range narrower than possibly anticipated. That is why, the cadmium content allows to tune the band gap in a broad range, while the temperature acts like a fine tuning parameter. Depending on the phenomena that one wants to observe, it is required to choose the cadmium content, which makes the band gap close to the desired value. In case of this work, the desired value of the band gap was negative, preferably close to zero, which would allow the temperature to tune it into a  positive regime. 

For cadmium content higher than the critical value $x > x_c \approx 0.17$, the Hg$_{1-x}$Cd$_x$Te crystal is a regular semiconductor. On the other hand, if the cadmium content is lower than the critical value $x < x_c$, the structure is inverted, as schematically shown on Figure \ref{fig:MCT_bandStructure}, and the structure exhibits a semimetallic behavior. The two phases are not topologically equivalent, as characterized by the $Z_2$ topological invariant \cite{Bernevig_Topology2}.

\begin{figure}[ht]
	\centering
	\includegraphics[width=10cm]{gfx/MCT/bulk/MCT_bandStructure.png}
	\vspace{-10pt}
	\caption{\textit{A dispersion relation in Hg$_{1-x}$Cd$_x$Te system for different cadmium content $x$. Blue band represents an electron band, while red represent light- and heavy-hole bands. Heavy hole band is flat, as the effective mass of heavy-holes is huge in comparison to light-hole and electron. On the left side a semiconducting phase is shown, where the band gap is positive and conduction band is formed by a $\Gamma_6$ band. On the right side the structure is inverted. The band gap is negative and the conduction band is formed by a $\Gamma_8$ band. For both positive and negative gaps the bands are parabolic. For a critical concentration $x_c$ the band gap vanishes and the system resembles a Dirac-like dispersion relation with additional flat band (heavy-hole). Courtesy of Orlita et al \cite{Orlita_MCT}.}}
	\label{fig:MCT_bandStructure}
\end{figure} 

As was mentioned before the band structure depends on more parameters than only the cadmium content. If we consider temperature as the second parameter, the point of closing the band gap becomes a curve on the two dimensional $(x, T)$ parameter space. This means that the critical contentration $x_c \approx 0.17$ is valid only for temperatures close to the absolute zero. However, samples with a little lower cadmium content have the critical temperature elevated. Our team \cite{Teppe_MCT} for the first time used the temperature as an external parameter to induce a topological phase transition and investigate the Kane fermions arising in the gapless state. The two samples used in that experiment had the cadmium content of $x_A = 0.175$ and $x_B = 0.155$. This allowed to study the physics of Kane fermions at higher temperature, as the temperature of phase transition of the sample B was around 77 K.  

The temperature is an important factor considering the physical phenomena occurring in solid state materials, especially in narrow-gap semiconductors like MCT. Temperature increases disorder within structures -- is an important component of instability and noise of measurements.

Temperature also influences the energy structure via lattice thermal expansion. This obviously modifies the Hamiltonian and the band structure as a consequence. In the case of narrow gap semiconductors, especially if the dependence on temperature is significant, it can lead to a gap closure, as in the case of MCT. The energy gap depends on cadmium content $x$ and temperature $T$, and that dependence can be expressed as

\begin{equation}
\begin{aligned}
\label{Eq:MCT_ExT}
E_g (x,T)[eV] = -0.303(1-x) + 1.606x - 0.132x(1-x) +\\ +\frac{6.3(1-x)-3.25x-5.92(1-x)}{11(1-x)+78.7x+T}10^{-4}T^2,
\end{aligned}
\end{equation}

which is a function on two parameter space. If the left side of Equation \ref{Eq:MCT_ExT} is equal to zero, it obviously is the case for gapless state. A dependence $x_c(T, E_g = 0)$ can be derived from Equation \ref{Eq:MCT_ExT}, which gives a quantitative information about the band structure in a form of a phase diagram, presented on Figure \ref{fig:MCT_TK}. 
\begin{figure}[ht]
	\centering
	\includegraphics[width=8cm]{gfx/MCT/bulk/TK.png}
	\vspace{-10pt}
	\caption{\textit{A phase diagram of bulk Hg$_{1-x}$Cd$_x$Te as a function of temperature $T$ and cadmium content $x_c$. The red area represents a parameter range, where the system is in a semiconducting state, characterized by a positive energy gap. The blue area represents a parameter range, where the system is in a semimetallic state, characterized by a negative energy gap. The thick black line represents a parameter range where the system is gapless.}}
	\label{fig:MCT_TK}
\end{figure} 

The appearance of Kane fermions in not restricted only to the gapless state, where they are truly massless. Even it the system has a gap, the behavior of (massive) carriers can be regarded as relativistic as long as the energy range is small in terms of the energies of nearby bands, mainly the spin-split $\Gamma_7$ band, which lies $\Delta_{SO} \approx 1$ eV \cite{Novik_MCT} lower in energy. Moreover, HgCdTe systems are not the only ones, where Kane fermions can be found. Recent studies of another extraordinary material, namely cadmium arsenide, revealed that Kane fermions are indeed potentially present in that system.

Cd$_3$As$_2$ has been identified as a 3D topological Dirac semimetal, where a topological phase is stable under ambiend conditions \cite{Liu_CdAs}, which brought again a considerable interest in the electronic properties of this compound, upon which investigations started in the late sixties \cite{Roseman_CdAs}\cite{Bodnar_CdAs}. This compound was firstly expected to exhibit the Dirac-like particles. ARPES measurements confirmed those expectations, claiming that the electronic bands of Cd$_3$As$_2$ consist of a single pair of symmetry-protected 3D Dirac nodes, located close to the $\Gamma$ point of the Brillouin zone, which span over a few hundred meV \cite{Borisenko_CdAs}\cite{Neupane_CdAs}, or even eV \cite{Liu_CdAs}. However, a recent optical reflectivity experiments performed by \cite{Akrap_CdAs} shed some light on that matter. As it turns out, the electronic bands of Cd$_3$As$_2$ are quite different if considered in a high and low energy scales. At high energy scales, in the order of few hundreds of meV, there is a set of two conical bands, which results from the Kane model applied to a narrow gap semiconductor. They are not symmetry protected and they host a genuine Kane fermions. However, at low energy scales, of the order of few meV, there might be two sets of twin Dirac cones, which are protected by symmetry. This means that the excitations of higher energy behave accordingly to Kane particles, while lower energy, if present, are Dirac-like. The band structure and its similarities with Dirac and Kane systems are presented in Figure \ref{fig:CdAs_bandStructure}. 

\begin{figure}[ht]
	\centering
	\includegraphics[width=10cm]{gfx/CdAs.png}
	\vspace{-10pt}
	\caption{\textit{Idealized Landau levels of a) Dirac and b) Kane electrons as occupied in the quantum limit and their relative energy scale. The n = 0 LLs typical of Dirac electrons, dispersing linearly with the momentum, are absent for Kane electrons. The solid arrows show cyclotron resonance modes active in the quantum imit, the dashed arrows those which vanish when this limit is reached. In the central inset the band structure and respective energy scales are presented. Dirac approach is valid for low energy scales, while Kane is valid for the general picture. Courtesy of Akrap et al \cite{Akrap_CdAs}.}}
	\label{fig:CdAs_bandStructure}
\end{figure} 

\subsection{The simplified Kane model}
\wciecie
The simplified Kane model \cite{Kane_MCT}\cite{Kane_Model} can be used to describe the electronic structure near the $\Gamma$ point of the Brillouin zone, where all the interesting physics takes place. This model accounts the $k\cdot p$ interaction between the $\Gamma_6$ and $\Gamma_8$ bands, while neglects the influence of the remote spin-split $\Gamma_7$ band. The final Hamiltonian \ref{Eq:MCT_Bulk_Hamiltonian}, neglecting the small quadratic in momentum terms, takes form of

\begin{equation}
\label{Eq:MCT_Bulk_Hamiltonian}
\hat{H} = \tilde{\beta} \tilde{m}\tilde{c}^2 + \tilde{c}\tilde{\alpha}_x p_x + \tilde{c}\tilde{\alpha}_y p_y + \tilde{c}\tilde{\alpha}_z p_z ,
\end{equation}

where $\tilde{c}$ is Fermi velocity, $\tilde{m}$ is effective mass, and $p_i$ is momentum. This Hamiltonian resembles the one for true 3D Dirac fermions, presented in a Dirac equation \ref{Eq:Dirac_Hamiltonian}, 

\begin{equation}
\label{Eq:Dirac_Hamiltonian}
i\bar{h} \frac{\partial \Psi}{\partial t} = \left( \beta mc^2 + c\alpha_x p_x + c\alpha_y p_y + c\alpha_z p_z \right) \Psi.
\end{equation}

However, the matrices $\tilde{\alpha}_i$ are different from $\alpha_i$ \cite{Teppe_MCT}. There are multiple bulk condensed matter systems, which can be well described by the Dirac Hamiltonian \ref{Eq:Dirac_Hamiltonian}. Nevertheless, the Hamiltonian \ref{Eq:MCT_Bulk_Hamiltonian} does not reduce itself to the Dirac Hamiltonian nor to any other known Hamiltonian describing relativistic particles. The 6x6 matrix version of Equation \ref{Eq:MCT_Bulk_Hamiltonian} is

\begin{equation}
\label{Eq:MCT_Bulk_Hamiltonian_Matrix}
\hat{H}{p_x, p_y, p_z} = \left( \begin{array}{cccccc}
\tilde{m}\tilde{c}^2 & \frac{\sqrt{3}}{2}\tilde{c}{p_+} & -\frac{1}{2}\tilde{c}{p_{-}} & 0 & 0 & -\tilde{c}p_z \\
\frac{\sqrt{3}}{2}\tilde{c}{p_+} & -\tilde{m}\tilde{c}^2 & 0 & 0 & 0 & 0 \\
-\frac{1}{2}\tilde{c}{p_-} & 0 & -\tilde{m}\tilde{c}^2 & -\tilde{c}p_z & 0 & 0 \\
0 & 0 & -\tilde{c}p_z & \tilde{m}\tilde{c}^2 & -\frac{\sqrt{3}}{2}\tilde{c}{p_-} & \frac{1}{2}\tilde{c}{p_+} \\
0 & 0 & 0 & -\frac{\sqrt{3}}{2}\tilde{c}{p_-} & -\tilde{m}\tilde{c}^2 &0 \\
-\tilde{c}p_z & 0 & 0 & \frac{1}{2}\tilde{c}{p_+} & 0 & -\tilde{m}\tilde{c}^2 \end{array} \right) \equiv \tilde{c}\boldsymbol{\mathrm{p}} \cdot \boldsymbol{\mathrm{J}},
\end{equation}
where $p_{\pm} = p_x \pm ip_y$, $E_g = \tilde{m}\tilde{c}^2$ is the energy gap, and $\tilde{c} = \sqrt{2P^2 / 3 \hbar^2}$ is the universal velocity. The material properties are included within the model by the Kane element $P$ and $E_g$. There are three eigenvalues of Equation \ref{Eq:MCT_Bulk_Hamiltonian_Matrix}, representing the energetic structure of the sample. Each is doubly degenerated due to the Kramers theorem. The eigenvalues can be presented as

\begin{equation}
\label{Eq:MCT_eigenvalues}
E_\xi (p) = \xi^2 \tilde{m}\tilde{c}^2 + (-1)^{1-\theta (\tilde{m})} \xi \sqrt{\tilde{m}^2\tilde{c}^4 + p^2\tilde{c}^2},
\end{equation}
where the $\xi$ parameter takes values of $\xi = -1$ for the light-hole band, $\xi = 0$ for the heavy-hole band, and $\xi = 1$ for the electron band. $\theta(\tilde{m})$ is a Heaviside step function, equals to 1 for $\tilde{m} \geq 0$, and 0 if $\tilde{m}$ is negative. An eigenvalue for $\xi = 0$ means that the heavy-hole band is energetically completely flat (dispersionless), which is a consequence of an assumption that heavy-hole mass is infinite. The assumption is valid as long as the effective electron mass is significantly smaller than the effective heavy-hole mass of about $m_{hh} \approx 0.5$ $m_0$ \cite{Weiler_MCT}, which is the case for narrow gap regime \cite{Orlita_MCT}.

\subsection{Bulk MCT at magnetic field}
The energy structure of the crystals become quantized in a presence of magnetic field. The 3D dispersion relation takes form of a set of unequally spaced Landau levels (LLs), or more precisely, a form of 1D Landau bands which disperse with the momentum component along the field direction (usually $z$ axis). Moreover, these LLs are characterized by a distinct $E \sim \sqrt{B}$ behavior. The similar behavior is found for example in gapless graphene, where the LL structure can be expressed by $E_n = \mathrm{sgn}(n) \sqrt{2e\hbar \tilde{c}^2B|n|}$, where sgn($n$) takes value +1 for electronic LLs, and -1 for hole-like LLs \cite{Jiang_MCT}. This is a direct consequence of linear dispersion relation $E(p) \propto p$, as states the Equation \ref{Eq:MCT_eigenvalues}. In a case of a parabolic dispersion relation $E(p) \propto p^2$, found in most of semiconductors, the LL structure takes form of $E = \hbar \omega_c(n + 1/2) = \hbar \frac{eB}{m^*}(n + 1/2)$. In this case, the energy of a LL is proportional to applied magnetic field, and each consecutive LL is separated from the adjacent one by a constant factor of $\hbar \omega_c$. 

Magnetic field forces modification of the Hamiltonian by an inclusion of components related to magnetic vector potential $\boldsymbol{\mathrm{A}}$, through the standard Peierls substitution $\hbar \boldsymbol{\mathrm{k}} \rightarrow \hbar\boldsymbol{\mathrm{k}} - e \boldsymbol{\mathrm{A}}$. In the case of 3D material like MCT, the LL spectrum of massless and massive fermions takes a more complex form

\begin{equation}
\label{Eq:MCT_LLs}
E_{\xi,n,\sigma }(p_z) = \xi^2 \tilde{m}\tilde{c}^2 + (-1)^{1-\theta (\tilde{m})} \xi \sqrt{\tilde{m}^2\tilde{c}^4 + \frac{1}{2}e\hbar\tilde{c}^2B (4n - 2 + \sigma) + p_z^2\tilde{c}^2},
\end{equation}
where $n$ is a Landau level index, $\sigma$ accounts for the Kramers degeneracy lifted by magnetic field, and can be considered as the Zeeman  (spin) splitting of LLs \cite{Teppe_MCT}. The index $n$ takes only integer values, with respect to the parameter $\xi$. For $\xi = \pm 1$, $n$ takes value of nonzero positive integers $n = 1,2,...$ . For $\xi = 0$, $n$ takes value of zero or all positive integers except one $n = 0,2,3,...$ . The orbital parameters $\tilde{c}$, $n$, and $p_z$ fully determine the spin splitting. Moreover, when the effective mass $\tilde{m}$ vanishes, and at $p_z = 0$, the spin splitting of LLs is exactly proportional to $\sqrt{B}$. This means that the g-factor, defined in the standard way as $g_{\xi , n} = (E_{\xi , n , \uparrow} - E_{\xi , n , \downarrow})/(\mu_B B)$ diverges at $B \rightarrow 0$. This in an extraordinary situation in a solid state system, and in particular, does not exist in the case of graphene \cite{Orlita_MCT}. The uncommon case of $\sqrt{B}$ spin splitting takes place in MCT because the strength of SOC becomes effectively infinite as the band gap vanishes. 

The energy spectrum becomes quantized into a set of LLs, and the Fermi energy separates filled LLs from the empty ones. An amount of filled LLs at given magnetic field is called filling factor $\nu$. An example of LL structure created using Equation \ref{Eq:MCT_LLs} is presented on Figure \ref{fig:MCT_LL120_solo}. Electron occupying the filled LL can be excited, for example by absorbing a photon, into an empty LL. This process is called optical transition, and is governed by a set of rules, which describe whether the process is allowed or not. First rule is that the spin in a transition has to be conserved. It means that the transition can take place between two LLs only if they are characterized by the same spin orientation. Second rule is that the transition can take place only for $\Delta n = \pm 1$, which means that the electron can switch only into adjacent levels. The allowed transitions are presented as arrows on Figure \ref{fig:MCT_LL120_solo}. The transitions can be divided into two groups, intraband and interband. 
\begin{figure}[ht]
	\centering
	\includegraphics[width=8cm]{gfx/MCT/bulk/LL120_solo.png}
	\vspace{-10pt}
	\caption{\textit{Landau level graph of bulk HgCdTe system as a function of magnetic field. Colored lines represent Landau levels, characterized by a different value of $\varepsilon$, $n$, and $\sigma$, as described in indices of $L$ on the right side of the graph. The electron band Landau levels (with $\varepsilon$ = 1) have a near-$\sqrt{B}$ behavior. The heavy-hole Landau level, plotted in purple, is fully degenerated. The vertical arrows represent a possible transitions between Landau levels in this system. Dashed arrows are intraband transitions, while solid arrows are interband. Energy gap was also marked on the figure by $E_g$.}}
	\label{fig:MCT_LL120_solo}
\end{figure}

Interband transitions take place when an electron changes the entire band while executing a transition, for example from a heavy hole into an electron band. Interband transitions are marked on Figure \ref{fig:MCT_LL120_solo} as solid arrows. In case of bulk MCT the heavy hole level is independent on $k$ vector nor magnetic field. According to the Equation \ref{Eq:MCT_LLs} it has zero energy for all nonzero integer values of $n$ (except $n$ = 1), and both values of $\sigma$. As a consequence, it is not formed by a single level but contains many levels, with the same energy, which are $2n$ times degenerate. That is why a transition from a heavy-hole band into an electron LL with index $n = 1$ or $n = 3$  is possible. However, the transition into an electron LL with index $n = 2$ is forbidden, as it would require an existence of a heavy-hole LL with $n$ = 1, which is not the case. Interband transitions give a valuable information at zero magnetic field, because the energy of transition is equal to the energy gap itself. 

Intraband transitions, on the other hand, take place when an electron changes only a LL, without changing a band. Intraband transitions are marked on Figure \ref{fig:MCT_LL120_solo} as dashed arrows. Intraband transitions tend to have zero energy at zero magnetic field, as the Landau quantization disappears.

It is worth to mention, that despite the fact that a transition is allowed, it might not be possible to observe it by the means of spectroscopy. The probability of a transition depends on a parameter called oscillator strength.

\clearpage
\section{Experiment}
\subsection{Samples}
\wciecie
The MCT bulk samples were grown using a standard molecular beam epitaxy on a (013)-oriented semi-insulating GaAs substrate. The substrate was followed by a ZnTe nucleation layer and a thick CdTe buffer layer to compensate the lattice mismatch of GaAs. The actual Hg$_{1-x}$Cd$_x$Te layer was approximately 4 micron thick to assure three-dimensionality of the active region of the sample, which was further confirmed by absorption coefficent measurements (Figure \ref{fig:Samples_MCT_AbsorptCoeff}). The whole structure was capped by a CdTe layer to prevent enviromental degradation processes like oxidation. The cadmium concentration varied from $x_A = 0.175$ for sample A to $x_B = 0.155$ for sample B.  The structure scheme and the cadmium content at different depth of sample A is presented on Figure \ref{fig:Samples_MCT}. 

\begin{figure}[ht]
	\centering
	\includegraphics[width=12cm]{gfx/MCT/Bulk/Samples_MCT.png}
	\vspace{-10pt}
	\caption{\textit{\textbf{Left panel:} Layer structure of sample A. The active region of the system is 5 $\mu$m thick Hg$_{1-x}$Cd$_x$Te layer, marked on yellow on the skech. \textbf{Right panel:} Function of cadmium content $x$ as a function of sample depth.}}
	\label{fig:Samples_MCT}
\end{figure} 

Sample A, the same one which was used by Orlita in his work \cite{Orlita_MCT}, is in normal semiconducting regime at the whole temperature range. Sample B, on the other hand, at low temperature is in the inverted band order regime, and as the temperature rises it enters the normal regime. A phase diagram of both samples is presented in Figure \ref{fig:TK2}

\begin{figure}[H]
	\centering
	\includegraphics[width=8cm]{gfx/MCT/bulk/TK2.png}
	\vspace{-10pt}
	\caption{\textit{Phase diagram of bulk Hg$_{1-x}$Cd$_x$Te as a function of temperature $T$ and cadmium content $x_c$. The red area represents a parameter range, where the system is in a semiconducting state, while the blue area represents a parameter range, where the system is in a semimetallic state. The two horizontal lines mark the concentration of sample A and sample B, denoted by A and B accordingly. For the sample B, which undergoes a phase transition, the critical temperature $T_c$ is marked as well.}}
	\label{fig:TK2}
\end{figure} 

\textbf{Sample characterization}
To estimate the electrical properties of the samples, like carrier concentration, the transport measurements were performed at magnetic field as a function of temperature. Samples were contacted with indium balls in a Van der Pauw configuration and placed in perpendicular quantized magnetic field (Voigt configuration). The Hall measurements allowed to establish a carrier concentrations within the structures for temperatures in a range of 2 K -- 140 K. The result is presented on Figure \ref{fig:Samples_MCT_Transport}. The electron concentration in both samples is comparable (one order of magnitude difference) and ranges from $n_A \approx 3 \cdot 10^{14}$ cm$^{-3}$ for sample A and $n_B \approx 2 \cdot 10^{15}$ cm$^{-3}$ sample B at $T = 2$ K up to
$n_A \approx 8 \cdot 10^{15}$ cm$^{-3}$ and $n_B \approx 9 \cdot 10^{16}$ cm$^{-3}$ at $T = 140$ K for sample A and sample B, respectively. The relatively low concentration allowed to perform the transmission measurements as the absorption was moderate.
\begin{figure}[ht]
	\centering
	\includegraphics[width=8cm]{gfx/MCT/Bulk/Concentration_vs_T.png}
	\vspace{-10pt}
	\caption{\textit{Electron concentration as a function of temperature for the sample A (orange squares) and sample B (purple squares).}}
	\label{fig:Samples_MCT_Transport}
\end{figure} 

An interesting consequence of conical dispersion relation on the optical properties of genuine three-dimensionality of the massless fermions is a proportionality of the absorption coefficient $\lambda (\omega)$ to the frequency $\omega$, which resembles a behavior of 3D Weyl systems \cite{Malcolm_MCT}. The situation is qualitatively different from 2D Dirac systems like graphene, where the absorption coefficient is independent on frequency \cite{Kuzmenko_MCTBulk}. This is a direct consequence of the joint density of states, being proportional to $\omega^2$ in 3D, and $\omega$ in a 2D case.  

---------------------------------------------

add a reference to Chapter \ref{chpt:optical}

---------------------------------------------

The analysis of the absorption coefficient, presented on Figure \ref{fig:Samples_MCT_AbsorptCoeff}, provided two pieces of information. The first one -- the dependence of absorption coefficient is indeed linear as a function of photon energy (and its frequency), even up to 300 meV. Theoretical curves are in a good agreement with experimental data, excluding the influence of phonon absorption. The second one -- curves for both samples differ from one another. The point of intersection with $x$-axis gives an insight on the value of energy gap at the temperature of measurements, and is equal to 4 $\pm$ 2 meV and -20 $\pm$ 4 meV, for sample A and B, respectively. An appearance of a small energy gap does not influence the relativistic nature of Kane fermions.   

\begin{figure}[H]
	\centering
	\includegraphics[width=8cm]{gfx/MCT/Bulk/MCT_AbsorptCoeff.png}
	\vspace{-10pt}
	\caption{\textit{Optical absorption of pseudo-relativistic Kane fermions in Hg$_{1-x}$Cd$_x$Te at $T$ = 4.2 K. Zero field absorption coefficients exhibit a linear behavior reflecting the relativistic character of the 3D Kane fermions in Hg$_{1-x}$Cd$_x$Te. The band gap values of 4 $\pm$ 2 meV and -20 $\pm$ 4 meV for sample A and B, respectively, are extracted from fits (dashed lines). The inset depicts inter-band transitions that contribute to the linear optical absorption.}}	\label{fig:Samples_MCT_AbsorptCoeff}
\end{figure} 

%\clearpage
\subsection{Results}
\label{sec:MCT_bulk}
\wciecie
In this section magnetospectroscopy results of two MCT bulk samples are presented and their band structure evolution in the function of temperature is compared. One of the samples is a positive-gap semiconductor, which band gaps increases monotonically with temperature. The second one is in a semimetallic state at low temperature and undergoes a semimetal-to-semiconductor phase transition as the temperature increases. A theoretical analysis of the results based on a simplified Kane model is presented and the implications are discussed.

In order to fully investigate the Kane fermions in MCT, arising in the vicinity of a semimetal-to-semiconductor phase transition, there is a need to carry the system through the transition and make measurements along the way. The two bulk samples, which were investigated, provided an opportunity to directly see the difference between the two phases, characterized by an inverted and non-inverted band ordering, and the evolution of their properties as a function of temperature.  

The theoretical model of the LL structure of the samples, based on Equation \ref{Eq:MCT_LLs}, allowed to predict the expected optical transitions, however the parameters $\tilde{m}$ and $\tilde c$ had to be extracted directly from the experiment by fitting. In theory, the parameter $\tilde c$, describing the Fermi velocity, should be temperature independent, as it depends only on the Kane element $P$ and physical constants, as described by     
Equation \ref{Eq:MCT_Bulk_Hamiltonian_Matrix}. On the other hand, the rest mass $\tilde{m}$, depending on both the $E_g$ and $\tilde{c}$, should change as the energy gap changes.

\myparagraph{Related experimental works}
\wciecie
This chapter presents experimental results which can be considered as a direct extension and continuation of the work by Orlita \textit{et al.} on bulk MCT \cite{Orlita_MCT}. Orlita studied by infrared magnetospectroscopy a HgCdTe sample close to the point of the semiconductor-to-semimetal topological transition. However, his investigation was limited to a constant and low temperature only. He observed for the first time relativistic 3D Kane fermions, as they manifest themselves by a $E \propto \sqrt{B}$ dependence on inter and intra Landau level transitions (Figure \ref{fig:Orlita_MCT}). 

\begin{figure}[ht]
	\centering
	\includegraphics[width=12cm]{gfx/MCT/Orlita_MCT.png}
	\vspace{-10pt}
	\caption{\textit{\textbf{Panel a)} Landau levels in gapless MCT, L$_{\xi,n,\sigma}$, as a function of the magnetic field, calculated using the eight-band model. Arrows of different colors show the optically allowed transitions in undoped gapless MCT in the two circular polarizations $\sigma^{+}$ and $\sigma^{-}$. \textbf{Panel b)} Relative change of absorbance $A_B/A_{B=0}$ plotted as a false color-map. All of the observed resonances clearly follow a $\sqrt{B}$-dependence. The dashed lines are calculated positions of inter-LL resonances at $k_z$ = 0 using parameters $v_F = 1.06 \cdot 10^6$ m/s and $\Delta$ = 1 eV. The presence of the spin-orbit split band, expressed by parameter $\Delta$, does not qualitatively change the LL spectrum, but introduces a weak electron-hole asymmetry. These images come from the work \cite{Orlita_MCT}.}}
	\label{fig:Orlita_MCT}
\end{figure}

Those results proved that magnetospectroscopy can be used as a detailed probe of the energy structure of topological insulators and narrow-band semiconductors. Moreover, magnetospectroscopy is noninvasive and doesn't require any particular sample preparation, in contrast to transport measurements, which require further adaptation of a sample -- contacts deposition, etching and/or more. This is especially important in a case of MCT, where processing is a challenging task due to the volatile of the sample structure at high temperatures required by processing.

The two main findings, presented in this chapter, are a proof that temperature can be used as an easy-controllable external parameter to alter a band structure of given sample and even tune it over a topological phase transition. This is the case for HgCdTe bulk samples as well as the QWs. However, for the case of a bulk band structure, as the dependence on temperature is stronger, the results are more explicit, than in the case of QWs. 

\myparagraph{Experimental details}
\wciecie
The transmission measurements were performed by the Fourier spectrometer. Data acquisition range was 10-700 cm$^{-1}$ (1.25-87.5 meV) with spectral resolution of 4 cm$^{-1}$. Transmission is calculated by dividing the spectra at given magnetic field by the reference spectrum, taken at zero magnetic field. This allows to detect changes in relative transmission as an effect of magnetic field. The spectra were gathered at a constant and stable magnetic field, and after measurement a value of field was changed. The usual magnetic field resolution was 0.25 T, and results obtained with such resolution are presented in this chapter. Few more measurements at different temperatures were carried out with lower magnetic field resolution (0.5 T - 1.0 T). Those results are presented in the Appendix.

The data in this chapter is presented in three figures for each temperature. The first figure presents a LL structure of the sample. LLs with different number $n$ were plotted with different colors, without any distinction for the spin --  a flat, black line represents the heavy hole band, red curves represent LLs with $n = 1$, namely $L_{1,1\downarrow}$ and $L_{1,1\uparrow}$. Consequently, blue curves represent LLs with $n = 2$, and green curves represent LLs with $n = 3$. Each level is described by a set of its parameters $L_{\varepsilon,n,\sigma}$. Moreover, visible transitions at given temperature are marked with arrows with unique color and assigned capital letter.
There are two types of arrows -- solid ones and dashed ones. Solid arrows indicate interband band transitions while dashed arrows indicate intraband transitions. 

Second figure in a set is a colormap. It is a plot of every spectra in a form of 2D figure, where blue color represent a value of 1, and reddish colors represent lower values, usually around 0.5. The scale is adjusted to every colormap to make fainter transitions visible and not to saturate the scale. Colormaps for sample A are plotted with $x$-axis in a $\sqrt{B}$ scale to expose the $E \propto \sqrt{B}$ behavior, which should appear as a straight line in a $\sqrt{B}$ scale. This is not the case for the sample B, where the linear scale is preserved. The transitions are marked with white lines and an explicit description of the origin and final level of the transition. The same pattern of style (solid, dashed) of lines applies to distinguish the interband and intraband transitions, in the same way as in the LL plot. 

Due to the phonon absorption, there were two bands where the data acquisition was not possible, as the samples are completely opaque. Those bands are called Reststrahlen bands. Phonon excitations can absorb up to 100\% incoming radiation. There are two Reststrahlen bands in presented transmission results. One originates from the phonons in GaAs substrate (between 30 and 40 meV), the other from HgCdTe itself (between 15 and 20 meV). This is why, transmission result presented in this work have two grey stripes, covering some energy ranges. Those results are not shown because of the way the transmission results are obtained -- as the transmission which is almost equal to zero gets divided by the reference also almost equal to zero, the results can get very high or low -- not giving any practical information, so they are hidden from the viewer.

The magnetic field resolution on a colormap seems to be higher than 0.25 T. This is due to the way the data is processed. In order to present data in a smooth and visibly clear way the data was enhanced numerically by performing a linear extrapolation between consecutive spectra. This resulted in artificial increase of magnetic field resolution and an appearance of some unwanted artifacts in a form of oscillations visible near deep transitions. This effect is not physical, but numerical.  

Third figure, composed of two panels, shows the way how data was fitted to transitions (left side) and the raw transition spectra themselves (right side). The points on the left side represent the minima of given transition, and the color of points is consistent with a color of an arrow from the LL plot, and the respective minimum on the right panel. The obtained curves via fitting, are plotted on the colormaps as well. The fitting provided two parameters -- the effective rest mass of carriers $\tilde{m}$ and the carrier velocity $\tilde{c}$. The grey stripes represent the position of Reststrahlen bands. The spectra (right side) are plotted every 1 T and each consecutive spectra is shifted by an appropriate value for clarity. A symbol is added over every visible minimum to mark its position. The shape and color of symbols resembles the symbols used on the left side of the figure to mark data points. 

\subsubsection{Sample A}
\myparagraph{Temperature 1.8 K}
\wciecie
The temperature of 1.8 K is the lowest achievable in the experimental setup. The experimental results should resemble the results of Orlita \cite{Orlita_MCT} (Figure \ref{fig:Orlita_MCT}), as the sample is the same and the temperature is similar. The band structure of the sample A at temperature of 1.8 K is the most similar to the structure of the sample B at 87 K. The band gap is positive and equal to 5 meV. There are no intraband transitions visible, most likely due to the limits of experimental system. The band structure is presented in Figure \ref{fig:LL_110429_2K}.

\begin{figure}[ht]
	\centering
	\includegraphics[width=10cm]{gfx/wyniki/MCTbulk/110429/LL2K.png}
	\vspace{-10pt}
	\caption{\textit{Landau level graph of sample A as a function of magnetic field at $T$ = 1.8 K. Colored lines represent Landau levels, characterized by a different value of $\varepsilon$, $n$, and $\sigma$, as described in indices of $L$ on the right side of the graph. The electron band Landau levels (with $\varepsilon$ = 1) have a near-$\sqrt{B}$ behavior. The heavy-hole Landau level, plotted in black, is fully degenerated. The vertical arrows with a corresponding capital letter represent observed transitions between Landau levels in this system. Solid arrows (A, B, E, and F) are interband transitions. Right panel presents an electron distribution as a function of energy, with chemical potential plotted as a dashed line, extending into the Landau level plot with a value of around 36 meV.}}
	\label{fig:LL_110429_2K}
\end{figure}
The colormap (Figure \ref{fig:Map_110429_2K}) shows that the transitions follow an almost $\sqrt{B}$ dependence, which is normal because the band gap is small. The band structure can be compared with a band structure of sample A at 87 K, which is above the critical temperature. 

The transitions marked as A and B are the most intense ones. This is a consequence of a enormous density of states on the heavy hole band in comparison to the density of states of the electron band, which is directly related to a probability of a transition and its depth. The high energetic transitions E and F are barely visible.

\begin{figure}[ht]
	\centering
	\includegraphics[width=10cm]{gfx/wyniki/MCTbulk/110429/Map2K.png}
	\vspace{-10pt}
	\caption{\textit{False color map of transmission of Sample A as a function of energy and magnetic field at $T$ = 2 K. Blue color represents areas where transmission is equal to 1, while lightblue/yellow/red colors points where absorption takes place. White curves show fits of the energy difference of Landau levels to the experimental points. White arrow points the value of the band gap.}}
	\label{fig:Map_110429_2K}
\end{figure} 

Filled and open squares represent interband and intraband transitions, respectively. The difference in intensity between interband and intraband transition is explicitly visible, however intensity is not the only property that makes both types of transitions distinguishable. The second, equally significant, difference is the shape of the transition --  intraband transitions can be modeled by symmetrical functions like Gaussians or Lorentzians. On the other hand, interband transitions are not symmetrical at all. At higher energies than the minimum there is a noticeable "tail" at higher energies from the minimum. That behavior is explained in Chapter \ref{chpt:optical}. 

The fits of the model to the experimental data are decent. The effective rest mass obtained from the fits presented on Figure \ref{fig:Spectra_110429_2K} turned out to be equal to $\tilde m = 0.61 \pm 0.34 \cdot 10^{-3}$ $m_0$. The carrier velocity was estimated to be equal to $\tilde{c} = 1.062 \pm 0.089 \cdot 10^6$ m/s. The uncertainty of measurements, here and for all of measurements, is based on a standard deviation of values obtained from all fitted transitions. At $T$ = 2 K there were only four fits (to each transition), that is why the uncertainty of the effective mass is around 50\%.

\begin{figure}[H]
	\centering
	\includegraphics[width=14cm]{gfx/wyniki/MCTbulk/110429/Trans2K.png}
	\vspace{-10pt}
	\caption{\textit{\textbf{Left panel:} Points corresponding to the minima of transmission for sample A at $T$ = 1.8 K with fits showing the expected transition evolution as a function of magnetic field. \textbf{Right panel:} Transmission spectra plotted at magnetic fields in range from 0 to 13 T, with symbols corresponding to the transitions from left panel. Symbols represent interband transitions.}}
	\label{fig:Spectra_110429_2K}
\end{figure} 

\clearpage
\myparagraph{Temperature 57 K}
\wciecie
Temperature of 57 K slightly changed the band structure. The band gap increased to 28 meV. The transitions to a LL with $n = 3$ are not detectable anymore, so the LL with are $n = 3$ not plotted on the LL graph presented on Figure \ref{fig:LL_110429_57K}, as they are irrelevant. At this temperature an intraband transition C became visible.

\begin{figure}[ht]
	\centering
	\includegraphics[width=10cm]{gfx/wyniki/MCTbulk/110429/LL57K.png}
	\vspace{-10pt}
	\caption{\textit{Landau level graph of sample A as a function of magnetic field at $T$ = 57 K. Colored lines represent Landau levels, characterized by a different value of $\varepsilon$, $n$, and $\sigma$, as described in indices of $L$ on the right side of the graph. The electron band Landau levels (with $\varepsilon$ = 1) have a near-$\sqrt{B}$ behavior. The heavy-hole Landau level, plotted in black, is fully degenerated. The vertical arrows with a corresponding capital letter represent observed transitions between Landau levels in this system. Solid arrows (A and B) are interband transitions, while dashed arrow (C) is an intraband transition.}} 
%Right panel presents an electron distribution as a function of energy, with chemical potential plotted as a dashed line, extending into the Landau level plot with a value of around 36 meV.}}
	\label{fig:LL_110429_57K}
\end{figure}

The presence of the transition C allows to differentiate the intraband and interband transitions, which was difficult at $T$ = 2 K, as all of the transitions converged close to zero energy. At $T$ = 57 K the band gap is enlarged, so there is a clear difference between the points of convergence for interband and intraband transitions, as it is presented on the color map in Figure \ref{fig:Map_110429_57K}.
 
At high magnetic field the transitions A and B are still very strong in comparison with a faint transition C. However, they vanish at low magnetic fields as the intensity of transition C increases. This allows to estimate the Fermi level position around 45 meV. A new feature appeared on the color map at the energy of around 10 meV. 

This horizontal feature is attributed to the absorption on defects of the structure. To be more precise on the mercury vacancies, which act as a doubly-ionized acceptor centers, which got thermally activated by the elevated temperature.

\begin{figure}[ht]
	\centering
	\includegraphics[width=10cm]{gfx/wyniki/MCTbulk/110429/Map57K.png}
	\vspace{-10pt}
	\caption{\textit{False color map of transmission of Sample A as a function of energy and magnetic field at $T$ = 57 K. Blue color represents areas where transmission is equal to 1, while lightblue/yellow/red colors points where absorption takes place. White curves show fits of the energy difference of Landau levels to the experimental points. White arrow points the value of the band gap.}}
	\label{fig:Map_110429_57K}
\end{figure} 

The spectra shown on Figure \ref{fig:Spectra_110429_57K} confirms that transitions A and B are indeed intraband, as they are asymmetric, while transition C, barely visible, tends to be symmetrical. The fits allowed to estimate the effective rest mass to be equal to $\tilde m = 2.82 \pm 0.77 \cdot 10^{-3}$ $m_0$, and the carrier velocity to be equal to $\tilde{c} = 1.057 \pm 0.060 \cdot 10^6$ m/s. 

\begin{figure}[ht]
	\centering
	\includegraphics[width=14cm]{gfx/wyniki/MCTbulk/110429/Trans57K.png}
	\vspace{-10pt}
	\caption{\textit{\textbf{Left panel:} Points corresponding to the minima of transmission for sample A at $T$ = 57 K with fits showing the expected transition evolution as a function of magnetic field. \textbf{Right panel:} Transmission spectra plotted at magnetic fields in range from 0 to 13 T, with symbols corresponding to the transitions from left panel. Open (full) symbols represent intraband (interband) transitions.}}
	\label{fig:Spectra_110429_57K}
\end{figure}  

\clearpage
\myparagraph{Temperature 77 K}
\wciecie
At the temperature of 77 K the band gap increases to 36 meV. The transitions A, B, and C are still visible. A new interband transition D becomes detectable, as it is shown in Figure \ref{fig:LL_110429_77K}.
\begin{figure}[ht]
	\centering
	\includegraphics[width=10cm]{gfx/wyniki/MCTbulk/110429/LL77K.png}
	\vspace{-10pt}
	\caption{\textit{Landau level graph of sample A as a function of magnetic field at $T$ = 77 K. Colored lines represent Landau levels, characterized by a different value of $\varepsilon$, $n$, and $\sigma$, as described in indices of $L$ on the right side of the graph. The electron band Landau levels (with $\varepsilon$ = 1) have a near-$\sqrt{B}$ behavior. The heavy-hole Landau level, plotted in black, is fully degenerated. The vertical arrows with a corresponding capital letter represent observed transitions between Landau levels in this system. Solid arrows (A and B) are interband transitions, while dashed arrows (C and D) are intraband transitions.}}
	\label{fig:LL_110429_77K}
\end{figure}

\begin{figure}[ht]
	\centering
	\includegraphics[width=10cm]{gfx/wyniki/MCTbulk/110429/Map77K.png}
	\vspace{-10pt}
	\caption{\textit{False color map of transmission of Sample A as a function of energy and magnetic field at $T$ = 77 K. Blue color represents areas where transmission is equal to 1, while lightblue/yellow/red colors points where absorption takes place. White curves show fits of the energy difference of Landau levels to the experimental points. White arrow points the value of the band gap.}}
	\label{fig:Map_110429_77K}
\end{figure} 

The temperature is high enough to lift the chemical potential up to around 45 meV, which causes transitions A and B to gain intensity above this energy, while transitions C and D become fainter. The horizontal feature related to absorption on vacancies become more pronounced and at this temperature is accompanied by a second feature, at energies around 20 meV, just above the Reststrahlen band of HgTe. The proximity of the position of that new feature to the Reststrahlen band is not a coincide -- this absorption is caused by the existence of states, related to the electron-phonon interaction.

The fits allowed to estimate the effective rest mass to be equal to $\tilde m = 3.19 \pm 0.41 \cdot 10^{-3}$ $m_0$, and the carrier velocity to be equal to $\tilde{c} = 1.049 \pm 0.052 \cdot 10^6$ m/s. 

\begin{figure}[ht]
	\centering
	\includegraphics[width=14cm]{gfx/wyniki/MCTbulk/110429/Trans77K.png}
	\vspace{-10pt}
	\caption{\textit{\textbf{Left panel:} Points corresponding to the minima of transmission for sample A at $T$ = 77 K with fits showing the expected transition evolution as a function of magnetic field. \textbf{Right panel:} Transmission spectra plotted at magnetic fields in range from 0 to 13 T, with symbols corresponding to the transitions from left panel. Open (full) symbols represent intraband (interband) transitions.}}
	\label{fig:Spectra_110429_77K}
\end{figure}

\clearpage
\myparagraph{Temperature 120 K}
\wciecie
At $T$ = 120 K band gap of sample A reaches as much as 59 meV, as it is shown in Figure \ref{fig:LL_101007_120K}. The number of detectable transitions at that temperature is the same, as in the case of $T$ = 77 K. The transitions A, B, C, and D are visible.  
\begin{figure}[ht]
	\centering
	\includegraphics[width=10cm]{gfx/wyniki/MCTbulk/110429/LL120K.png}
	\vspace{-10pt}
	\caption{\textit{Landau level graph of sample A as a function of magnetic field at $T$ = 120 K. Colored lines represent Landau levels, characterized by a different value of $\varepsilon$, $n$, and $\sigma$, as described in indices of $L$ on the right side of the graph. The electron band Landau levels (with $\varepsilon$ = 1) have a near-$\sqrt{B}$ behavior. The heavy-hole Landau level, plotted in black, is fully degenerated. The vertical arrows with a corresponding capital letter represent observed transitions between Landau levels in this system. Solid arrows (A and B) are interband transitions, while dashed arrows (C and D) are intraband transitions.}}
	\label{fig:LL_110429_120K}
\end{figure}

\begin{figure}[ht]
	\centering
	\includegraphics[width=10cm]{gfx/wyniki/MCTbulk/110429/Map120K.png}
	\vspace{-10pt}
	\caption{\textit{False color map of transmission of Sample A as a function of energy and magnetic field at $T$ = 120 K. Blue color represents areas where transmission is equal to 1, while lightblue/yellow/red colors points where absorption takes place. White curves show fits of the energy difference of Landau levels to the experimental points. White arrow points the value of the band gap.}}
	\label{fig:Map_110429_120K}
\end{figure} 
Due to the chemical potential, which lies at more than 80 meV above the heavy hole band, the transitions A and B are not visible at small magnetic field. Due to the same reason the transitions C and D vanish at high magnetic fields. However, the fit to experimental points is very accurate for available data (Figure \ref{fig:Spectra_101007_120K}). The temperature related features becomes even stronger, as the thermal energy at 120 K is more than 10 meV.

The fits allowed to estimate the effective rest mass to be equal to $\tilde m = 3.38 \pm 0.46 \cdot 10^{-3}$ $m_0$, and the carrier velocity to be equal to $\tilde{c} = 1.037 \pm 0.027 \cdot 10^6$ m/s. 

\begin{figure}[ht]
	\centering
	\includegraphics[width=14cm]{gfx/wyniki/MCTbulk/110429/Trans120K.png}
	\vspace{-10pt}
	\caption{\textit{\textbf{Left panel:} Points corresponding to the minima of transmission for sample A at $T$ = 120 K with fits showing the expected transition evolution as a function of magnetic field. \textbf{Right panel:} Transmission spectra plotted at magnetic fields in range from 0 to 13 T, with symbols corresponding to the transitions from left panel. Open (full) symbols represent intraband (interband) transitions.}}
	\label{fig:Spectra_110429_120K}
\end{figure}

\clearpage
\subsubsection{Sample B}
\myparagraph{Temperature 1.8 K}
\wciecie
At the temperature 1.8 K sample B is in the inverted regime with the largest negative gap. The LL structure of the sample B is presented on Figure \ref{fig:LL_101007_2K}. All of the LLs from both electron and heavy hole bands converge as the magnetic field goes to zero, even though the sample has a negative energy gap. This is explained by the model via Equation \ref{Eq:MCT_LLs}. By assuming the energy gap to have a nonpositive value and putting $B = 0$, the whole expression equals to zero.

There are six transitions observable -- two interband, marked with solid lines, and four intraband, marked with dashed lines. A dashed horizontal line represents a position of the Fermi level. The Fermi distribution is presented on the panel on the right side of Figure \ref{fig:LL_101007_2K}. At 1.8 K the Fermi distribution resembles a step function.

\begin{figure}[ht]
	\centering
	\includegraphics[width=10cm]{gfx/wyniki/MCTbulk/101007/LL2K.png}
	\vspace{-10pt}
	\caption{\textit{Landau level graph of sample B as a function of magnetic field at $T$ = 1.8 K. Colored lines represent Landau levels, characterized by a different value of $\varepsilon$, $n$, and $\sigma$, as described in indices of $L$ on the right side of the graph. The electron band Landau levels (with $\varepsilon$ = 1) have a near-$\sqrt{B}$ behavior. The heavy-hole Landau level, plotted in black, is fully degenerated. The vertical arrows with a corresponding capital letter represent observed transitions between Landau levels in this system. Solid arrows (A, B, E, and F) are interband transitions, while dashed arrows (C, D) are intraband transitions. Energy gap was also marked on the figure by $E_g$. Right panel presents an electron distribution as a function of energy, with chemical potential plotted as a dashed line, extending into the Landau level plot with a value of around 36 meV.}}
	\label{fig:LL_101007_2K}
\end{figure}

The observed transitions are presented on Figure \ref{fig:Map_101007_2K}. A green dashed line was plotted on the figure to give an idea about the energy gap value. It was created as an extrapolation of the transition A at high magnetic field. 

One way to understand this is via invoking the Equation \ref{Eq:MCT_LLs}. There are two factors under the square root, first dependent on the energy gap, and second dependent on magnetic field. For high magnetic field the first factor is negligible, so the whole expression formally resembles an expression in a form of $E(B) = E_g/2 + \sqrt{\alpha B} $, where $\alpha$ is a constant. This equation can be plotted as a straight line in a $\sqrt{B}$ scale and it intercepts the $y$-axis exactly where $E(B = 0) = E_g/2$, which gives an indication of the value of the band gap divided by two. On Figure \ref{fig:Map_101007_2K} this value is marked with a black arrow and equals to $-12$ meV, which translates to $E_g = -24$ meV.
\begin{figure}[ht]
	\centering
	\includegraphics[width=10cm]{gfx/wyniki/MCTbulk/101007/Map2K.png}
	\vspace{-10pt}
	\caption{\textit{False color map of transmission of Sample B as a function of energy and magnetic field at $T$ = 2 K. The scale of magnetic field is presented in a $\sqrt{B}$ scale. Blue color represents areas where transmission is equal to 1, while lightblue/yellow/red colors points where absorption takes place. White curves show fits of the energy difference of Landau levels to the experimental points. Green dashed line is an extrapolation of transition $L_{0,0\downarrow}\rightarrow L_{1,1\downarrow}$, which points the value of the (negative) band gap.}}
	\label{fig:Map_101007_2K}
\end{figure} 

The intensities of the transitions are not constant at the whole range of energies. The most pronounced example is the intensity of the transition D, which decreases at $E \approx 25$ meV, while the intensity of the transition A increases radically. This is a consequence of the density of filled states and the position of the Fermi level, which is around 30 meV above the heavy hole band (dashed horizontal line on Figure \ref{fig:LL_101007_2K}). The states below the Fermi level are occupied, so the transition from an occupied state into another occupied state is not possible. In this sense the transition A is forbidden at magnetic field below $\approx 6$ T, as the energy of level $L_{1,1\downarrow}$ lies below the Fermi energy. On the other hand, the transition D is possible at low magnetic field, because the level $L_{1,1\uparrow}$ lies below the Fermi energy while the level $L_{1,2\downarrow}$ lies above it. The situation changes at around 3 T, where both levels lie above the Fermi energy, thus there are no occupied states to execute a transition.

The transitions and the transmission spectra are presented on Figure \ref{fig:Spectra_101007_2K}. On panel a) there is an unknown transition, represented by grey points, which is visible as well on Figure \ref{fig:Map_101007_2K}.

The fits allowed to estimate the effective rest mass to be equal to $\tilde m = -1.91 \pm 0.58 \cdot 10^{-3}$ $m_0$, and the carrier velocity to be equal to $\tilde{c} = 1.091 \pm 0.069 \cdot 10^6$ m/s. 

\begin{figure}[ht]
	\centering
	\includegraphics[width=14cm]{gfx/wyniki/MCTbulk/101007/Trans2K.png}
	\vspace{-10pt}
	\caption{\textit{\textbf{Left panel:} Points corresponding to the minima of transmission for sample B at $T$ = 1.8 K with fits showing the expected transition evolution as a function of magnetic field. \textbf{Right panel:} Transmission spectra plotted at magnetic fields in range from 0 to 16 T, with symbols corresponding to the transitions from left panel. Open (full) symbols represent intraband (interband) transitions.}}
	\label{fig:Spectra_101007_2K}
\end{figure} 

\newpage
\myparagraph{Temperature 37 K}
\wciecie
Temperature of 37 K slightly changed the band structure. The negative energy gap got smaller down to -14 meV, so the LLs still converge at zero magnetic field. The most noticeable change is a disappearance of the transition F, which is shown in Figure \ref{fig:LL_101007_37K}. 

\begin{figure}[ht]
	\centering
	\includegraphics[width=10cm]{gfx/wyniki/MCTbulk/101007/LL37K.png}
	\vspace{-10pt}
	\caption{\textit{Landau level graph of sample B as a function of magnetic field at $T$ = 37 K. Colored lines represent Landau levels, characterized by a different value of $\varepsilon$, $n$, and $\sigma$, as described in indices of $L$ on the right side of the graph. The electron band Landau levels (with $\varepsilon$ = 1) have a near-$\sqrt{B}$ behavior. The heavy-hole Landau level, plotted in black, is fully degenerated. The vertical arrows with a corresponding capital letter represent observed transitions between Landau levels in this system. Solid arrows (A, B, and E) are interband transitions, while dashed arrows (C and D) are intraband transitions. Energy gap was also marked on the figure by $E_g$. Right panel presents an electron distribution as a function of energy, with chemical potential plotted as a dashed line, extending into the Landau level plot with a value of around 40 meV.}}
	\label{fig:LL_101007_37K}
\end{figure}
Transitions C, D and E are faint in general, as their probability is relatively low, as is presented on the color map on Figure \ref{fig:Map_101007_37K}. The horizontal feature close to 10 meV, related to absorption on mercury vacancies, starts to be visible.

The position of the Fermi level lies close to 25 meV above the heavy hole band, as the intensity of transitions A i D switched. However, both transitions are further separated, than at $T = 1.8$ K. This can be explained by the change of Fermi distribution, which no longer resembles the step function, as the temperature is increased.

\begin{figure}[ht]
	\centering
	\includegraphics[width=10cm]{gfx/wyniki/MCTbulk/101007/Map37K.png}
	\vspace{-10pt}
	\caption{\textit{False color map of transmission of Sample B as a function of energy and magnetic field at $T$ = 37 K. The scale of magnetic field is presented in a $\sqrt{B}$ scale. Blue color represents areas where transmission is equal to 1, while lightblue/yellow/red colors points where absorption takes place. White curves show fits of the energy difference of Landau levels to the experimental points. Green dashed line is an extrapolation of transition $L_{0,0\downarrow}\rightarrow L_{1,1\downarrow}$, which points the value of the (negative) band gap.}}
	\label{fig:Map_101007_37K}
\end{figure} 

The fits allowed to estimate the effective rest mass to be equal to $\tilde m = -1.44 \pm 0.68 \cdot 10^{-3}$ $m_0$, and the carrier velocity to be equal to $\tilde{c} = 1.052 \pm 0.052 \cdot 10^6$ m/s.

\begin{figure}[ht]
	\centering
	\includegraphics[width=14cm]{gfx/wyniki/MCTbulk/101007/Trans37K.png}
	\vspace{-10pt}
	\caption{\textit{\textbf{Left panel:} Points corresponding to the minima of transmission for sample B at $T$ = 37 K with fits showing the expected transition evolution as a function of magnetic field. \textbf{Right panel:} Transmission spectra plotted at magnetic fields in range from 0 to 16 T, with symbols corresponding to the transitions from left panel. Open (full) symbols represent intraband (interband) transitions.}}
	\label{fig:Spectra_101007_37K}
\end{figure}  

\clearpage
\myparagraph{Temperature 77 K}
\wciecie
The temperature of 77 K is the critical temperature for the sample B. This is where the band gap vanishes completely and the dispersion relation is linear and formally consists of a Dirac cone. The amount of detectable transitions is the same as at $T$ = 37 K, however transition E becomes even fainter and more difficult to detect. 
\begin{figure}[ht]
	\centering
	\includegraphics[width=10cm]{gfx/wyniki/MCTbulk/101007/LL77K.png}
	\vspace{-10pt}
	\caption{\textit{Landau level graph of sample B as a function of magnetic field at $T$ = 77 K. Colored lines represent Landau levels, characterized by a different value of $\varepsilon$, $n$, and $\sigma$, as described in indices of $L$ on the right side of the graph. The electron band Landau levels (with $\varepsilon$ = 1) have a near-$\sqrt{B}$ behavior. The heavy-hole Landau level, plotted in black, is fully degenerated. The vertical arrows with a corresponding capital letter represent observed transitions between Landau levels in this system. Solid arrows (A, B, and E) are interband transitions, while dashed arrows (C and D) are intraband transitions. Right panel presents an electron distribution as a function of energy, with chemical potential plotted as a dashed line, extending into the Landau level plot with a value of around 45 meV.}}
	\label{fig:LL_101007_77K}
\end{figure}

The Figure \ref{fig:LL_101007_77K} shows that the band gap is zero and LLs follow precisely a $\sqrt{B}$ dependence on energy. This means that all of the LLs converge at zero magnetic field. As a consequence, all the transitions follow a $\sqrt{B}$ dependence on energy as well, as presented on a color map on Figure \ref{fig:Map_101007_77K}. 

\begin{figure}[ht]
	\centering
	\includegraphics[width=10cm]{gfx/wyniki/MCTbulk/101007/Map77K.png}
	\vspace{-10pt}
	\caption{\textit{False color map of transmission of Sample B as a function of energy and magnetic field at $T$ = 77 K. The scale of magnetic field is presented in a $\sqrt{B}$ scale. Blue color represents areas where transmission is equal to 1, while lightblue/yellow/red colors points where absorption takes place. White curves show fits of the energy difference of Landau levels to the experimental points. All of the transitions converge at $B$ = 0 T, when the band gap vanishes.}}
	\label{fig:Map_101007_77K}
\end{figure} 

The temperature is high enough to lift the chemical potential up to around 45 meV, which causes transitions A and B to gain intensity above this energy, while transitions C and D become fainter. Both horizontal features related to absorption on defects become more pronounced at this temperature.

The fits allowed to estimate the effective rest mass to be equal to $\tilde m = -0.49 \pm 0.53 \cdot 10^{-3}$ $m_0$, and the carrier velocity to be equal to $\tilde{c} = 1.056 \pm 0.001 \cdot 10^6$ m/s.

\begin{figure}[ht]
	\centering
	\includegraphics[width=14cm]{gfx/wyniki/MCTbulk/101007/Trans77K.png}
	\vspace{-10pt}
	\caption{\textit{\textbf{Left panel:} Points corresponding to the minima of transmission for sample B at $T$ = 77 K with fits showing the expected transition evolution as a function of magnetic field. \textbf{Right panel:} Transmission spectra plotted at magnetic fields in range from 0 to 16 T, with symbols corresponding to the transitions from left panel. Open (full) symbols represent intraband (interband) transitions.}}
	\label{fig:Spectra_101007_77K}
\end{figure}

\clearpage
\myparagraph{Temperature 120 K}
\wciecie
Above the critical temperature of 77 K sample B becomes semiconducting, as the band gap opens. At 120 K band gap reaches as much as 18 meV, as is shown on Figure \ref{fig:LL_101007_120K}. The number of detectable transitions at that temperature diminished, as only transitions A, C, and D are visible.  
\begin{figure}[ht]
	\centering
	\includegraphics[width=10cm]{gfx/wyniki/MCTbulk/101007/LL120K.png}
	\vspace{-10pt}
	\caption{\textit{Landau level graph of sample B as a function of magnetic field at $T$ = 120 K. Colored lines represent Landau levels, characterized by a different value of $\varepsilon$, $n$, and $\sigma$, as described in indices of $L$ on the right side of the graph. The electron band Landau levels (with $\varepsilon$ = 1) have a near-$\sqrt{B}$ behavior. The heavy-hole Landau level, plotted in black, is fully degenerated. The vertical arrows with a corresponding capital letter represent observed transitions between Landau levels in this system. Solid arrow (A) is an interband transition, while dashed arrows (C and D) are intraband transitions. Right panel presents an electron distribution as a function of energy, with chemical potential plotted as a dashed line, extending into the Landau level plot with a value of around 36 meV.}}
	\label{fig:LL_101007_120K}
\end{figure}

An interband transition A, for the first time for sample B, does not converge at zero energy at zero magnetic field. When the band gap is positive, at zero magnetic field it points exactly the value of the band gap, which is presented on Figure \ref{fig:Map_101007_120K}. 
\begin{figure}[ht]
	\centering
	\includegraphics[width=10cm]{gfx/wyniki/MCTbulk/101007/Map120K.png}
	\vspace{-10pt}
	\caption{\textit{False color map of transmission of Sample B as a function of energy and magnetic field at $T$ = 120 K. The scale of magnetic field is presented in a $\sqrt{B}$ scale. Blue color represents areas where transmission is equal to 1, while lightblue/yellow/red colors points where absorption takes place. White curves show fits of the energy difference of Landau levels to the experimental points. White arrow points the value of the band gap.}}
	\label{fig:Map_101007_120K}
\end{figure} 
Due to the chemical potential, which lies at more than 80 meV above the heavy hole band, the transition A is not visible at small magnetic field. Because of the same reason the transition D vanishes at high magnetic fields. However, the fit to experimental points is very accurate for available data (Figure \ref{fig:Spectra_101007_120K}). The temperature related features become even stronger, as the thermal energy at 120 K is higher than 10 meV.

The fits allowed to estimate the effective rest mass to be equal to $\tilde m = 2.15 \pm 0.46 \cdot 10^{-3}$ $m_0$, and the carrier velocity to be equal to $\tilde{c} = 1.010 \pm 0.007 \cdot 10^6$ m/s.

\begin{figure}[ht]
	\centering
	\includegraphics[width=14cm]{gfx/wyniki/MCTbulk/101007/Trans120K.png}
	\vspace{-10pt}
	\caption{\textit{\textbf{Left panel:} Points corresponding to the minima of transmission for sample B at $T$ = 120 K with fits showing the expected transition evolution as a function of magnetic field. \textbf{Right panel:} Transmission spectra plotted at magnetic fields in range from 0 to 16 T, with symbols corresponding to the transitions from left panel. Open (full) symbols represent intraband (interband) transitions.}}
	\label{fig:Spectra_101007_120K}
\end{figure}

\clearpage
\subsection{Summary}
The results presented in this chapter show in a direct and straightforward way the evolution of the band gap of bulk Hg$_{1-x}$Cd$_{x}$Te with close to the critical cadmium concentration. Results based on two samples were shown and compared to easily highlight the difference in band structure and its evolution between regular and inverted phase systems. The temperature evolution of the band gap and a semiconductor-to-semimetal phase transition was studied by the means of magnetospectroscopy. 

The critical temperature was found by an investigation of LLs evolution with magnetic field -- at the critical temperature two conditions must be met: 
\begin{itemize}
\item All transitions between LLs have to follow an $E \propto \sqrt{B}$ dependence at a broad scale of magnetic fields and energies,
\item All transitions between LLs have to converge to zero energy at zero magnetic field, showing that the band gap vanishes.
\end{itemize}

Those conditions were met by the sample B with cadmium concentration of $x$ = 0.155 at the temperature $T_c$ = 77 K. Moreover, a band gap opening was registered at higher temperature $T > T_c$, as the interband transitions converged at $B$ = 0 T at nonzero energy, and the value of the bang gap increased as the temperature got higher. This process was expected due to the study of the sample B, which band gap was positive at the whole range of temperatures, and its value increased with temperature.  

These results are the first evidence of a temperature induced phase transition measured by magnetospectroscopy in bulk MCT structures with well-chosen chemical composition in the vicinity of the semimetal-to-semiconductor phase transition. The genuine Kane fermions were observed at the critical temperature of 77 K. In order to describe the data, the simplified Kane model was used. It allowed to determine the pseudo-relativistic Dirac-like Kane fermions parameters $\tilde{m}$ and $\tilde{c}$ as a function of temperature and the cadmium concentration. The results are in agreement with theoretical predictions and also, are consistent with the data obtained previously by Orlita \cite{Orlita_MCT}.

\begin{figure}[ht]
	\centering
	\includegraphics[width=14cm]{gfx/wyniki/MCTbulk/Transitions_allT.png}
	\vspace{-10pt}
	\caption{\textit{Evolution of the transition $L_{0,0\downarrow} \rightarrow L_{1,1\downarrow}$ for different temperatures. For all temperatures $T < T_c$ the band gap is negative, thus the transition converges at $E$ = 0 meV at $B$ = 0 T and the shape of transition does not have a pure $\sqrt{B}$ dependence. At $T = T_c$ the band gap vanishes and the transition follows precisely $\sqrt{B}$ (a straight line on $\sqrt{B}$ scale). At $T > T_c$ the transition does not follow a $\sqrt{B}$ dependence again. Moreover, it converges to a finite value of the band gap at $B$ = 0 T, which is a sign of the (positive) band gap opening}}
	\label{fig:Bulk_Transitions_allT}
\end{figure}
The temperature evolution of the shape of the transitions is shown explicitly on Figure \ref{fig:Bulk_Transitions_allT}, where the transition $L_{0,0\downarrow} \rightarrow L_{1,1\downarrow}$ is shown for the broad range of temperatures. The $x$-axis is presented in a $\sqrt{B}$ scale for clarity. The band gap is negative for all the temperatures below the critical temperature, and the dispersion relation does not resemble the Dirac cone, as the transitions do not follow a true $\sqrt{B}$ dependence. The band gap opens above the critical temperature, as the transitions for 87 K and 100 K do not converge to zero energy at zero magnetic field. Finally, at the critical temperature, the band gap vanishes completely, and a pure $\sqrt{B}$ dependence is observed, which confirms that the system exhibits a true Dirac cone.

Each and every transition for all temperatures was fitted according to the model in Equation \ref{Eq:MCT_LLs}. One fit provided a set of values of $\tilde{c}$ and $\tilde{m}$. The obtained values are plotted on Figure \ref{fig:Bulk_MassVelocity}, top panel a) presents the rest mass and bottom panel b) presents the velocity. Error bars on the Figure \ref{fig:Bulk_MassVelocity} come from the standard deviation of the values taken for every transition at given temperature.

\begin{figure}[ht]
	\centering
	\includegraphics[width=14cm]{gfx/wyniki/MCTbulk/MassVelocity.png}
	\vspace{-10pt}
	\caption{\textit{Parameters $\tilde{m}$ and $\tilde{c}$ obtained from fits to the experimental data as a function of temperature. Panel a) rest mass of Kane fermions. For sample B the mass changes sign at the $T_c$, while for sample A the mass is always positive. Panel b) velocity of Kane fermions. Velocity is constant at all temperatures, for both samples, which is highlighted by a dashed blue line at $v = 1.07 \cdot 10^6$ m/s. }}
	\label{fig:Bulk_MassVelocity}
\end{figure}

Interestingly, the Dirac-like Kane fermion velocity $\tilde{c}$ is nearly constant over the whole range of temperatures for both samples with Cd contents of 0.155 and 0.175. The extracted value of $\tilde{c} = (1.07 \pm 0.05)\cdot 10^6$ m/s is in a very good agreement with the theoretical value defined by $\tilde{c} = \sqrt{2 P^2/3\hbar^2}$, which equals to $1.05 \cdot 10^6$ m/s for the well-accepted value of $E_p = 2m_0 P^2/ \hbar^2 \approx 18.8$ eV. Therefore, this universal value of $\tilde{c}$ allows to determine the particle rest-mass for bandgap values in the vicinity of the semimetal-to-semiconductor phase transition induced by temperature, Cd content, or other external parameter (e.g. pressure).

\begin{figure}[ht]
	\centering
	\includegraphics[width=14cm]{gfx/wyniki/MCTbulk/Gfactor.png}
	\vspace{-10pt}
	\caption{\textit{Obsadz}}
	\label{fig:Bulk_gfactor}
\end{figure}

There are two points limiting the applicability of the simplified Kane model, considering the $\Gamma_6$ and $\Gamma_8$ band only, for actual HgCdTe crystals. The first one, already mentioned above, is related to the existence of other bands, considered as remote and not included in the model. The energy gap between the second and the lowest conduction bands in CdTe exceeds 4 eV, while the corresponding gap in HgTe is about 3 eV. Therefore, the cut-off energies for conduction bands in the simplified model should be lower than 3 eV. For the valence band, the cut-off energy is defined by the energy difference $\Delta \approx$ 1 eV between the split-off $\Gamma_7$ band, and the heavy-hole band. The second limitation is attributed to the flat heavy-hole band, characterized by an infinite effective mass in the model. To ignore the parabolic terms in the electron dispersion of the heavy-hole band, one has to consider sufficiently low energies $E$, such that the relativistic mass of the fermions $E/\tilde{c}^2$ should be significantly lower than the heavy-hole mass $m_{hh}$. Assuming $m_{hh} \approx 0.5$ $m_0$, where $m_0$ is the free electron mass, we arrive at a cut-off energy of about 3 eV for the flat band approximation, which exceeds $\Delta$.
\begin{figure}[ht]
	\centering
	\includegraphics[width=14cm]{gfx/MCT/bulk/DiracVsKane.png}
	\vspace{-10pt}
	\caption{\textit{A comparison of band structures described by Dirac Hamiltonian (graphene as an example) and Kane Hamiltonian (bulk MCT as an example). In both cases the dispersion relation is linear -- which is a property of relativistic particles. The major difference is a presence of a HH band.}}
	\label{fig:DiracVsKane}
\end{figure}

%\begin{table}[h]
%\centering
%\begin{tabular}{ c | c | c }	
%& \textbf{Dirac fermions} & \textbf{Kane fermions} \\
%\hline\hline
%Hamiltonian $\hat{H}$ & $ \beta mc^2 + c\alpha_x p_x + c\alpha_y p_y + c\alpha_z p_z$ & $\tilde{\beta} \tilde{m}\tilde{c}^2 + \tilde{c}\tilde{\alpha}_x p_x + \tilde{c}\tilde{\alpha}_y p_y + \tilde{c}\tilde{\alpha}_z p_z$ \\ \hline
%LL energy & $E_n = \mathrm{sgn}(n) \sqrt{2e\hbar \tilde{c}^2B|n|}$ \cite{Jiang_MCT}& $E_{\xi,n,\sigma} = \xi^2 \tilde{m}\tilde{c}^2 \pm \xi \sqrt{\tilde{m}^2\tilde{c}^4 + \frac{1}{2}e\hbar\tilde{c}^2B\sigma_n}$ \\ \hline
%Example material & Graphene & Bulk HgCdTe \\ \hline
%Carrier velocity [m/s] & $(1.093 \pm 0.004) \cdot 10^6$ \cite{Graphen_velocity} & $(1.07 \pm 0.05)\cdot 10^6$ \\ \hline 
%\end{tabular}
%\end{table}




\chapter{MCT Quantum Wells}
\label{chpt:MCT_QW}

\section{Overview of MCT Quantum Wells}
\wciecie
A typical HgTe/CdTe QW is formed when a layer of HgTe is sandwiched between two layers of CdTe, which form barriers for the QW. The energy levels, for both electrons and holes, within the structure lie above the bottom of QW, and their position depends on the QW width. As the QW width varies, the relative position of the first electron-like subband ($E_1$) and the first hole-like subband ($H_1$) changes. For thin quantum wells with well thickness $d$ $<$ 6.3 nm the quantum confinement is strong and the energy levels are lifted from the bottom of their respective QW. The structure exhibits a normal semiconducting phase with conventional subbands alignment -- the level $E_1$ lies above the $H_1$ level. 

In the opposite case, for quantum wells of thickness $d$ $>$ 6.3 nm, the situation is reversed -- the quantum confinement is weaker and the $H_1$ level lies above the $E_1$, which results in a band inversion. 

Consequently, for $d_c$ $=$ 6.3 nm the band gap vanishes and the system undergoes a topological phase transition from a trivial insulator to a QSH insulator, and the QW hosts single-valley 2D massless Dirac fermions \cite{Buttner_MCT_SQW}. 

The other way to understand this is to assume that for thin QW the structure should behave similarly to CdTe and have a regular band ordering, i.e. bands with $\Gamma_6$ symmetry form the conduction subbands and the $\Gamma_8$ symmetry bands form the valence subbands. If the QW thickness $d$ is increased the structure starts to resemble more and more the properties of HgTe. When the thickness reaches the critical value $d_c$ = 6.3 nm the $\Gamma_6$ and $\Gamma_8$ subbands cross and the structure becomes inverted -- the $\Gamma_6$ bands become valence subbands and the $\Gamma_8$ bands become conduction subbands. The QW states derived from the heavy hole $\Gamma_8$ band are named $H_n$, where $n = 1, 2, 3, ...$ denotes the states existing in the QW. Similarly, those levels originated from the electron $\Gamma_6$ band are named $E_n$. The band structure and first few energy levels of a HgTe/CdTe QW as a function of QW width are shown of Figure \ref{fig:BandStructure_HgTe_CdTe2}.  

\begin{figure}[ht]
	\centering
	\includegraphics[width=10cm]{gfx/BandStructure_HgTe_CdTe2.png}
	\vspace{-10pt}
	\caption{\textit{\textbf{Panel a)} The energy of the states in the quantum well as a function of the width of the HgTe QW layer. $E_n$ represent electron-like states, while $H_n$ represent hole-like states. \textbf{Panels b), c)} Schematic of a quantum well geometry and lowest subbands for thicknesses below \textbf{(b)} and above \textbf{(c)} the critical thickness. Images come from the works of König \cite{Konig_MCT_SQW} \textbf{(a)}, and Bernevig \cite{Bernevig_Topology2} \textbf{(b,c)}.}}
	\label{fig:BandStructure_HgTe_CdTe2}
\end{figure} 

One year after the theoretical proposal of Bernevig \cite{Bernevig_Topology2} the Molenkamp's group at the University of Würzburg fabricated the devices and performed the first transport experiments showing the signature of the QSH insulator \cite{Konig_Topology}. This work showed that for thin quantum wells with well width $d$ $<$ 6.3 nm, the insulating regime show the conventional behavior of neglectable conductance at low temperature. However, for thicker quantum wells ($d$ $>$ 6.3 nm), the nominally insulating regime showed a plateau of residual conductance close to $2e^2/h$. The residual conductance is independent of the sample dimensions, indicating that it is caused by the edge states \cite{Konig_Topology}. The low temperature ballistic transport via edge states can be understood within a basic Landauer-Büttiker \cite{Landauer_MCT} framework, in which the edge states are populated adequately to the chemical potential. As a consequence, conductance is quantized and equal to $e^2/h$ for each set of edge states. Furthermore, the residual conductance is destroyed by applying a small external magnetic field. The quantum phase transition at the critical thickness, $d_c$ = 6.3 nm, was also determined independently from the insulator-to-semimetal phase transition induced by magnetic field.

\clearpage
\section{The band structure of HgTe/CdTe QW}
\wciecie
The energy dispersion of the E1 and H1 subbands of a HgTe/CdTe QW near the critical thickness can be calculated using an 8-band Kane model. It turns out that near the $\Gamma$ point of the Brillouin zone the dispersion depend linearly on momentum \textbf{\textit{k}}. Using the states (after BHZ \cite{Bernevig_Topology2}) $\ket{E_1,\frac{1}{2}}$, $\ket{H_1,\frac{3}{2}}$, $\ket{E_1,-\frac{1}{2}}$, $\ket{H_1,-\frac{3}{2}}$ as a basis, the effective Hamiltonian for the E1 and H1 subbands takes form of

\begin{equation}
\label{eq:MCT_SQW_Hamiltonian}
\begin{aligned}
H_{eff}(k_x, k_y) =  \left( \begin{array}{cc}
H(k) & 0 \\
0 & H^*(-k) \end{array} \right), \\
H = \epsilon (k) + d_i (k) \sigma_i,
\end{aligned}
\end{equation}
where $\sigma_i$ are Pauli matrices, and
\begin{equation}
\label{eq:MCT_SQW_Hamiltonian2}
\begin{aligned}
d_1 + id_2 = \mathcal{A}(k_x - ik_y) \equiv \mathcal{A}k_-, \\
d_3 = \mathcal{M} - \mathcal{B}(k_x^2 + k_y^2), \\
\epsilon = \mathcal{C} - \mathcal{D}(k_x^2 + k_y^2).
\end{aligned}
\end{equation}

The two components of the Pauli matrices $\sigma$ in equation \ref{eq:MCT_SQW_Hamiltonian} denote the E1 and H1 subbands, while the two diagonal blocks $H(k)$ and $H^*(-k)$ represent spin-up and spin-down states, connected to each other by the time reversal symmetry. 

In the gapples state, the relativistic mass $M$ in equation \ref{eq:MCT_SQW_Hamiltonian2} vanishes. By neglecting the nonlinear terms in each spin, $H(k)$ and $H^*(-k)$ can be approximated with the massless Dirac Hamiltonian describing the genuine Dirac fermions. Since HgTe QW does not have any valley degeneracy, the Dirac fermions exist only in a single valley configuration. A comparison of an approximate Dirac-like band structure and results of numerical calculations of band structure based on the Kane model are presented on Figure \ref{fig:MCT_SQW_BS_BHZvsDirac}. As mentioned before, the Dirac approximation is valid in the vicinity of \textbf{\textit{k}} = 0, where higher in momentum terms can be neglected.  

\begin{figure}[H]
	\centering
	\includegraphics[width=12cm]{gfx/MCT/sqw/MCT_SQW_BS_BHZvsDirac.png}
	\vspace{-10pt}
	\caption{\textit{Comparison of band structures obtained using the 8-band Kane model (red and blue solid curves) and the Dirac-type 2D Hamiltonian (black dotted curves) for the gapless state.}}
	\label{fig:MCT_SQW_BS_BHZvsDirac}
\end{figure} 

\subsection{The influence of magnetic field on the band structure}
\wciecie
The LL structure of HgTe QW at applied magnetic field can be described in two ways. The first way is based on an approximation -- using Dirac Hamiltonians, which gives the possibility to solve explicitly for the energy of the LLs. However, the approximation holds only for a limited range of parameters. The second (accurate) way in based on the 8-bands Kane Hamiltonian, and its solutions can be found only via numerical calculations.

The Hamiltonian takes form of
\begin{equation}
\label{eq:MCT_SQW_MagneticField}
\hat H = H_{eff} + H_{Zeeman} + H_{SIA} + H_{BIA} ,
\end{equation}
where $H_{eff}$ is the effective Hamiltonian from equation \ref{eq:MCT_SQW_Hamiltonian} with a Peierls substitution \textbf{\textit{k}} $\rightarrow$ \textbf{\textit{k}} + $\frac{e}{\hbar}$\textbf{\textit{A}} applied, $H_{Zeeman}$ includes Zeeman effects at magnetic field, $H_{SIA}$
represents the structure inversion asymmetry, and $H_{BIA}$ bulk inversion asymmetry. The expressions for $H_{Zeeman}$, $H_{SIA}$, and $H_{BIA}$, as well as the set of parameters used in the model can be found in Appendix.

Neglecting the SIA and BIA terms, the Landau level spectrum takes form of
\begin{equation}
\label{eq:MCT_SQW_LLs}
\begin{aligned}
E^{\uparrow}_{\alpha}(n) = -\frac{eB_{\perp}}{\hbar}(2\mathcal{D}n+\mathcal{B})+\frac{\mu_B B_{\perp}}{4}(g_E + g_H) \\
+ \alpha\sqrt{2n\mathcal{A}^2\frac{eB_{\perp}}{\hbar} + \left(\mathcal{M}-B_{\perp} \left(\frac{e}{\hbar}(\mathcal{D}+2\mathcal{B}n)-\frac{\mu_B}{4}(g_E + g_H)\right)\right)^2} \\
E^{\downarrow}_{\alpha}(n) = -\frac{eB_{\perp}}{\hbar}(2\mathcal{D}n-\mathcal{B})-\frac{\mu_B B_{\perp}}{4}(g_E + g_H) \\
+ \alpha\sqrt{2n\mathcal{A}^2\frac{eB_{\perp}}{\hbar} + \left(\mathcal{M}-B_{\perp} \left(\frac{e}{\hbar}(-\mathcal{D}+2\mathcal{B}n)+\frac{\mu_B}{4}(g_E + g_H)\right)\right)^2}, 
\end{aligned}
\end{equation}
where $n$ = 1, 2, ..., and $\alpha$ = +1 for conduction band and $\alpha$ = -1 for valence band. The parameter $\mathcal{C}$ is set to zero to put the Dirac point at zero energy. For the gapless sample ($\mathcal{M} = 0$) the expression \ref{eq:MCT_SQW_LLs} for conduction band ($\alpha$ = +1) at low magnetic fields reduces to
\begin{equation}
\label{eq:MCT_SQW_LLs2}
E^{\uparrow (\downarrow)}_0 = B_{\perp}\left( -\frac{e}{\hbar}(2\mathcal{D}n+\mathcal{B})+\frac{\mu_B}{4}(g_E + g_H)\right)+\alpha\sqrt{2n\mathcal{A}^2\frac{eB_{\perp}}{\hbar}}
\end{equation}
up to linear terms. This is the cause of the square-root magnetic field dependence that became the signature of Dirac fermions in graphene \cite{Castro_graphene}, with an additional linear term describing the large g-factor of the HgTe QW. This model holds for low magnetic field, where the approximations are valid. For the higher magnetic field, the 8-band Kane model has to be used and the magnetic field dependence no longer follows a square-root function.

Meanwhile, the states for $n$ = 0, called \textit{zero-mode} \cite{Bernevig_Topology2}, can be described by equations \ref{eq:MCT_SQW_ZeroMode}. It is worth to mention that these states are numbered with $n$ = 0 only in Dirac-like Hamiltonian approximation. In Kane-model approach, these states are numbered differently. 
\begin{equation}
\label{eq:MCT_SQW_ZeroMode}
\begin{aligned}
E^{\uparrow}_0 = \mathcal{M} -\frac{eB_{\perp}}{\hbar}(\mathcal{D}+\mathcal{B})+\frac{\mu_B B_{\perp}}{2}g_E \\
E^{\downarrow}_0 = \mathcal{M} +\frac{eB_{\perp}}{\hbar}(-\mathcal{D}+\mathcal{B})-\frac{\mu_B B_{\perp}}{2}g_H.
\end{aligned}
\end{equation}

The spin splitting of those LLs takes form
\begin{equation}
\label{eq:MCT_SQW_ZeroMode_SpinSplitting}
\Delta E_s = E^{\uparrow}_0 - E^{\downarrow}_0 = 2\mathcal{M} -\frac{2eB_{\perp}}{\hbar}\mathcal{B}+\frac{\mu_B B_{\perp}}{2}(g_E + g_H). 
\end{equation}

The detailed LL spectrum of HgTe/CdTe is derived from the 8-band Kane Hamiltonian and requires solving a set of eight coupled differential equations for a given LL. In general, for LLs with $n \leq 0$, there are only 7, 4, and 1 non-trivial solutions for $n = 0$, -1, and -2, respectively. For LLs with higher $n$ indices the results consists of four pairs of spin-split levels. The single LL with $n = -2$ has a pure heavy hole character and its energy decreases linearly with magnetic field. This level, along with one the levels with $n$ = 0, forms a set of zero-mode LLs, already described in Dirac-like Hamiltonian approximation.

\myparagraph{Zero-mode Landau levels}
\wciecie
A very particular property of zero-mode LLs manifests itself with a variation of applied magnetic field. For magnetic fields below a critical value $B_c$, the lower zero-mode LL has an electron-like character and arises from the valence band, while the higher zero-mode LL has a heavy hole-like character and arises from the conduction band. While the structure is in inverted band order phase, the edge channels are present. However, they are no longer protected by the time-reversal symmetry, as the magnetic field is applied. With increasing magnetic field the zero-mode LLs merge and eventually cross themselves, turning the system into a semiconducting phase. This critical magnetic field, where the crossing occurs, can be derived from equation \ref{eq:MCT_SQW_ZeroMode_SpinSplitting}. By substituting $\Delta E_s = 0$ and neglecting small g-factor terms, the critical magnetic field takes form of $B^c_{\perp} = \frac{\hbar \mathcal{M}}{e\mathcal{B}}$. In the inverted regime $\mathcal{M}/\mathcal{B} > 0$, the crossing takes place at a positive magnetic field value, while in normal regime, where $\mathcal{M}/\mathcal{B} < 0$, the crossing extrapolates to a negative value of magnetic field. For a gapless QW, the crossing occurs at zero field. Therefore, the position of the crossing is a well-defined indication of the phase of the system. The zero-mode LLs (thick red and grey curves) and their crossing point (vertical dashed line), separating the inverted phase from the semiconductor phase, is presented in Figure \ref{fig:MCT_SQW_BS_Crossing}. 

\begin{figure}[H]
	\centering
	\includegraphics[width=12cm]{gfx/MCT/sqw/MCT_SQW_BS_Crossing.png}
	\vspace{-10pt}
	\caption{\textit{Landau level structure obtained using the 8-band Kane model for the inverted state. Thick red (n = 0) and grey (n = -2) curves represent zero-mode LLs, which exhibit a crossing around $B_c = 5$ T, marked with a vertical dashed line. The crossing point separates an inverted band ordering phase ($B < B_c$) from the regular band ordering phase ($B > B_c$). Transitions from zero-mode LLs are represented with arrows marked as $\alpha$ and $\alpha'$.}}
	\label{fig:MCT_SQW_BS_Crossing}
\end{figure} 

In the case of HgTe/CdTe QWs in a gapped state, the off-diagonal terms of the massless Dirac Hamiltonian have to be completed by the diagonal massive terms. In any case, the zero-mode LLs still appear due to the off-diagonal, linear in \textbf{\textit{k}} terms. Moreover, in a gapped state, the zero-mode LLs are split in energy due to the existence of the mass term, and their position changes monotonically as a function of the temperature or the width of the QW.

The magnetic field evolution of the zero-mode LLs in HgTe QWs, and their crossing at $B_c$ can be considered as a field-driven insulator-metal-insulator phase transition. The transport data \cite{Konig_Topology} gave an indication that those LLs simply cross themselves, but also a possibility of a weak anticrossing was considered \cite{Konig_MCT_SQW}.

Indeed, several magnetospectroscopy studies \cite{Orlita_MCT_QW}\cite{Zholudev_MCT_QW} proved that those levels may exhibit an anticrossing in the vicinity of the calculated $B_c$. This effect can be observed in magnetospectroscopy as an evolution of two transitions, which are presented in Figure \ref{fig:MCT_SQW_BS_Crossing}. First of those transitions is a regular transition from LLs $n$ = 0 to $n$ = 1, designated as transition $\alpha$, following the notation of Schultz \textit{et al.} \cite{Schultz}. Second however, designated as transition $\alpha'$, is a transition from LL $n$ = -2 to $n$ = 1. This behavior does not satisfy the selection rules $\Delta n = \pm 1$, and is forbidden in the electric dipole approximation. This transition is related with a coupling between the LLs $n$ = 0 and $n$ = -2, resulting from bulk inversion asymmetry (BIA), which is inherently present in zinc-blend crystals, but often neglected in band structure calculations of HgTe QWs. BIA results in a mixing of of the states of the zero-mode LLs in the vicinity of $B_c$. This mixing activates the transition $\alpha'$ in the electric dipole approximation, rendering it detectable in magneto-optical studies.

\subsection{The influence of temperature on the band structure}
\wciecie
Apart from the variation of QW thickness, which is inherently an internal parameter of the structure -- once set (at growth) cannot be changed, external paramteres like hydrostatic pressure \cite{Krishtopenko_pressure} and temperature \cite{Wiedmann_State} (as in the case of MCT bulk systems presented in Chapter \ref{chpt:HgCdTe bulk systems}) can be used to induce a phase transition in MCT QW systems. This is a consequence of a strong temperature dependence of energy of E1 level. This dependence and a band order evolution is presented in Figure \ref{fig:SQW_EGvsT} for two systems -- a 6 nm HgTe QW with a regular band order at all temperatures (Panel a)), and a 8 nm HgTe QW with an inverted band order at low temperature (Panel b)). The second system undergoes a temperature-induced phase transition at a critical temperature $T_c$, and turns into a regular semiconductor at higher temperatures. The energy of H1 (and H2) bands does not change with temperature at all, while there is a clear dependence of energy on temperature of E1 band.

\begin{figure}[ht]
	\centering
	\includegraphics[width=12cm]{gfx/MCT/sqw/SQW_EGvsT.png}
	\vspace{-10pt}
	\caption{\textit{Temperature dependence of the electron-like E1 (blue curve) and hole-like H1, H2 (red curves) subbands at zero quasimomentum, calculated for \textbf{Panel a)} a 6 nm HgTe/CdTe quantum well, and \textbf{Panel a)} a 8 nm HgTe/CdTe quantum well. At $T = T_c$, marked with an arrow, the 8 nm quantum well undergoes a phase transition characterized by a band inversion.}}
	\label{fig:SQW_EGvsT}
\end{figure} 

The calculations of the band structure and the band nonparabolicity are based on an 8-band \textbf{\textit{k$\cdot$p}} model taking into account the temperature dependence of all relevant parameters, including but not limited to the lattice and elastic constants of Hg$_{1-x}$Cd$_x$Te. The 8-band \textbf{\textit{k$\cdot$p}} Hamiltonian takes into account the interactions between $\Gamma_6$, $\Gamma_8$, and $\Gamma_7$ bands. Despite the fact that the electronic states of HgTe QWs can be described qualitatively by the 6-band \textbf{\textit{k$\cdot$p}} model, only the inclusion of the $\Gamma_7$ band in the calculations allows to obtain the quantitative values of hydrostatic pressure or temperature of the phase transition. For other HgCdTe-based materials, like bulk systems, this is not the case -- the influence of the $\Gamma_7$ band can be neglected in the calculations \cite{Malcolm_MCT}.

The calculations for the 8 nm structure revealed that at high temperatures the band order is regular and the energy gap between E1 and H1 bands considerably decreases with decreasing temperature, which is presented on Panel c of Figure \ref{fig:3D_bandorder_HgTe}. At a critical temperature $T_c = 90$ K the band gap vanishes giving rise to massless Dirac fermions (Panel b). Further decrease of temperature induces a phase transition and renders the structure inverted with an indirect band gap, arising due to the side maxima of the valence band (Panel a).

 
\begin{figure}[ht]
	\centering
	\includegraphics[width=12cm]{gfx/MCT/sqw/3d_bandorder_T.png}
	\vspace{-10pt}
	\caption{\textit{Dispersion relation of a 8 nm HgTe/CdTe quantum well in \textbf{Panel a)} the topological insulator phase $T < T_c$, \textbf{Panel b)} the gapless state $T = T_c$, and \textbf{Panel c)} the semiconductor phase $T > T_c$. The electron-like E1 subband is represented as a blue surface, while the hole-like H1 subband as a red surface. In the inverted phase the indirect band gap is formed by the presence of side maxima of E1 subband.}}
	\label{fig:3D_bandorder_HgTe}
\end{figure} 

\clearpage
\section{Experiment}
\subsection{Samples}
\wciecie
The MCT QW samples were grown like MCT bulk samples using a molecular beam epitaxy on a (013)-oriented semi-insulating GaAs substrate followed by a relaxed CdTe buffer layer \cite{Dvoretsky_Samples}. The active part of a QW consisted of a HgTe (6 nm for sample A, 8 nm for sample B) layer sandwiched between 40 nm Cd$_x$Hg$_{1-x}$Te barriers. A cap layer of CdTe was deposited on top of the structures to prevent oxidation. Sample A remained undoped, while the barriers of sample B were doped on each side with 15 nm layers of indium with the doping concentration of $6.5\cdot10^{10}$ cm$^{-2}$. This resulted in a formation of a 2D electron gas in the QW of sample B. The QW width $d$ and cadmium concentration $x$ of investigated structures is given in Table \ref{tab:MCT_QW_properties}.   

\begin{table}[h]
\label{tab:MCT_QW_properties}
\caption{The properties of investigated HgTe/CdTe quantum wells -- quantum well thickness, Cd concentration in the barrier, and type ($p$ for holes, $n$ for electrons) and concentration of dominant carriers.}
\vspace{10pt}
\centering
\begin{tabular}{ c | c | c | c}	
\textbf{Sample name} & \textbf{QW thickness} & \textbf{Barrier Cd rate $x$} & \textbf{Type, carrier concentration (2 K)}\\
\hline\hline
Sample A & 6 nm & 0.62 & $p = 3\cdot10^{10}$ cm$^{-2}$\\ \hline
Sample B & 8 nm & 0.80 & $n = 3\cdot10^{11}$ cm$^{-2}$\\ \hline\hline 
\end{tabular}
\end{table}

The critical thickness corresponding to a phase transition at different temperatures for HgTe/Hg$_{1-x}$Cd$_{x}$Te samples with $x$ = 0.62 (the same as sample A) and $x$ = 0.80 (the same as sample B) is presented in Figure \ref{fig:SQW_DCvsT}. The 6 nm sample A (represented by a black curve) is in a semiconducting state at the whole temperature range. The 8 nm sample B (red curve) at low temperature is a topological insulator and is expected to exhibit a phase transition at $T = 90$ K followed by an opening of the band gap.

\begin{figure}[ht]
	\centering
	\includegraphics[width=12cm]{gfx/MCT/sqw/SQW_DCvsT.png}
	\vspace{-10pt}
	\caption{\textit{The critical thickness at different temperatures for HgTe/Hg$_{1-x}$Cd$_{x}$Te samples with $x$ = 0.62 (black curve) and $x$ = 0.80 (red curve).}}
	\label{fig:SQW_DCvsT}
\end{figure} 

\subsection{Results}
\label{sec:MCT_QW}
\wciecie

\myparagraph{Related experimental works}
The first results, concerning this particular behavior were reported in works of König \cite{Konig_Topology} and Zhang \cite{Zhang_Topology}. It was shown that the magnetic field evolution of those zero-mode LLs in HgTe QWs is an origin of a magnetic field-driven insulator-metal-insulator phase transition, which is characteristic for these systems. A crossing of this levels at the critical field was confirmed \cite{Konig_Topology} by magneto-transport data.

Orlita \cite{Orlita_MCT_QW} demonstrated for the first time by magnetospectroscopy measurements on two 8 nm wide HgTe QWs the evolution of zero-mode LLs. It turned out that those levels do not cross, but exhibit an effect of avoided crossing. This is most likely caused by an effect of bulk-inversion asymmetry (BIA) which is present in zinc-blend crystals and usually neglected in calculations of band structures of HgTe/CdTe QWs. His findings were confirmed by the work of Zholudev \textit{et al.} \cite{Zholudev_MCT_QW_anticrossing}, where that anticrossing was observed via magnetospectroscopy, and its presence was attributed to BIA.

The first systematic magnetospectroscopy study of HgTe QWs systems with different QW width close to the critical thickness was performed by Zholudev \cite{Zholudev_MCT_QW}. In his experiment a set of four samples, two of them were in inverted regime of thickness, while the other two in noninverted regime. This allowed to directly observe the difference between two separate topological phases and the band structure that they originate from. An anticrossing of zero-mode LLs was also observed.

The first temperature dependent study of phase transition in MCT QW were done by Ikonnikov \cite{Ikonnikov_MCT_SQW}. The study was conducted in pulsed magnetic fields up to 45 T using monochromatic radiation sources. It revealed a temperature-induced merging of the absorption lines, corresponding to the transitions from the zero-mode LLs. The results are presented in Figure \ref{fig:Ikonnikov_MCT_SQW}.

\begin{figure}[H]
	\centering
	\includegraphics[width=14cm]{gfx/MCT/sqw/Ikonnikov_MCT_SQW.png}
	\vspace{-10pt}
	\caption{\textit{Magnetoresistance and magnetoabsorption spectra obtained at different temperatures on a 8-nm HgTe QW. \textbf{Panel a)} Solid lines 4.2 K, dotted lines 20 K, dashed lines 30 K, using a 14.8-$\mu$m QCL. \textbf{Panel b)} Solid lines 80 K, dotted lines 102 K, dashed lines 174 K, using a CO$_2$ laser emitting at 10.6-$\mu$m. Images come from the work \cite{Ikonnikov_MCT_SQW}.}}
	\label{fig:Ikonnikov_MCT_SQW}
\end{figure}

All the magnetospectroscopy studies of Dirac fermions in HgTe mentioned above were conducted either at low temperatures and/or using monochromatic THz sources. In 2015, a magnetotransport study conducted by Wiedmann \textit{et al.} \cite{Wiedmann_State} showed fingerprints of a temperature-induced transition from the TI at 4.2 K to the SC phase at 300 K. However, the critical temperature for the samples used, where the phase transition occurs and the massless Dirac fermions arise, was too high to be determined by this technique. This is caused by a significant degradation of resolution between LLs observed in magnetotransport at high temperatures.


\myparagraph{Experimental details}
\wciecie
In this work a set of two samples was investigated in order to determine if a temperature induced topological phase transition takes place in HgTe/CdTe QW. Similarly to bulk system, the first sample exhibits semiconductor behavior at the whole range of temperatures, while the second one has an inverted band structure at low temperatures, undergoes a phase transition at the critical temperature and turns into a regular semiconductor at high temperatures. In contrary to bulk systems, a two-dimensional HgTe/CdTe QW in inverted state is not a semimetal but a topological insulator. 

Magnetospectroscopy measurements were performed at magnetic fields up to 16 T and in energy range 80 - 800 cm$^{-1}$ (10 - 100 meV) with a 4 cm$^{-1}$ resolution. The temperature range was between 2 K and 130 K. The infrared transmittance spectra were measured by a Fourier spectrometer with a Globar lamp as a source of radiation. The system was coupled with a liquid helium cryostat. The transmission spectra were obtained by dividing the spectra at given magnetic field by the spectra obtained at zero magnetic field.

In order to interpret the experimental results a set of temperature-dependent band structure and LL structure calculations based on eight-band \textbf{\textit{k}}$\cdot$\textbf{\textit{p}} Hamiltonian were performed. The calculations took into account a tensile strain in the layers resulting from the mismatch of lattice constants of CdTe buffer, Cd$_{x}$Hg$_{1-x}$Te barriers, and HgTe QW. The energies of LLs were obtained using axial approximation, while the calculations of dispersion relations held also nonaxial terms. On each LL graph there are three bands visible -- $E1$, $H1$, and $H2$.

On each and every spectra there is an completely opaque region due to the presence of reststrahlen band between 16 and 21 meV and 30 and 37 meV for HgTe/HgCdTe layers and GaAs substrate, respectively. Because of that the energy regions corresponding to these bands were covered by grey areas on the spectra.

\subsubsection{Sample A}
\myparagraph{Temperature 2 K}
\wciecie
Figure \ref{fig:LL_110624_2K} presents the details of LL calculations for temperature of 2 K for sample A. Only E1 and HH1 bands are present. In the Faraday configuration, optical transitions between LLs are required to follow a $\Delta n = \pm 1$ selection rules, provided by the electric dipole approximation. By taking into account the previous optical studies of HgTe QWs \cite{Schultz}\cite{Orlita_MCT_QW}\cite{Zholudev_MCT_QW}\cite{Ikonnikov_MCT_SQW}\cite{Ludwig_MCT_QW}\cite{Zholudev_MCT_QW_anticrossing}, the transmission spectra are expected to be dominated by the transitions between LLs with low indices. Those transitions are marked in Figure \ref{fig:LL_SQW_2K} by small greek letters $\alpha_1$, $\alpha_2$, $\beta$ and solid arrows, as all visible transitions are interband transitions, which is a consequence of the low concentration of holes as dominant carriers in sample A. 

The band gap is equal to 2.7 meV. The electrons from both of the zero-mode LLs, numbered by $n$ = -2 and $n$ = 0, take part in optical transitions, which allows to confirm that these LLs do not cross (for a positive value of magnetic field). This is a confirmation that the sample is in the semiconducting phase. 

\begin{figure}[ht]
	\centering
	\includegraphics[width=14cm]{gfx/wyniki/MCTSQW/110624/110624_LL2K.png}
	\vspace{-10pt}
	\caption{\textit{Landau level graph of sample A as a function of magnetic field at $T$ = 2 K. Colored lines represent Landau levels, characterized by a different value of $n$, as described as indices of $L$ on the right side of the graph. The Landau levels with $n$ = -2 and $n$ = 0 are zero-mode Landau levels. The energy of those levels at zero magnetic field is an indication of the band gap. The vertical arrows with corresponding greek letters represent observed transitions between Landau levels in this system. Solid arrows $\alpha_1$, $\alpha_2$, and $\beta$ indicate interband transitions.}}
	\label{fig:LL_110624_2K}
\end{figure}

The expected dependence of transition as a function of magnetic field can be derived from the difference of the calculated LLs. In the Panel a) of Figure \ref{fig:Spectra_110624_2K} these expected transitions are presented along with experimental points taken from the spectra. There is a strong agreement of experimental data with theoretical calculation of transitions. The strongest transitions are $\alpha_1$ and $\beta$, which is shown on waterfall plot in Panel b) of Figure \ref{fig:Spectra_110624_2K}. The spectra are plotted every 1 T and shifted vertically for clarity.

\begin{figure}[ht]
	\centering
	\includegraphics[width=14cm]{gfx/wyniki/MCTSQW/110624/110624_spectra2K.png}
	\vspace{-10pt}
	\caption{\textit{\textbf{Left Panel:} Points corresponding to the minima of transmission for sample A at $T$ = 2 K with theoretical predictions of transitions as a function of magnetic field. \textbf{Right panel:} Transmission spectra plotted for magnetic fields in range from 1 to 16 T every 1 T, with symbols corresponding to the transitions from left panel.}}
	\label{fig:Spectra_110624_2K}
\end{figure}

\clearpage
\myparagraph{Temperature 30 K}
\wciecie
At $T$ = 30 K the band structure of sample A slightly changes. The energy difference of zero-mode LLs at zero magnetic field got a little higher -- the band gap is equal to 10.2 meV. The LL structure is presented in Figure \ref{fig:LL_110624_30K}.

\begin{figure}[ht]
	\centering
	\includegraphics[width=10cm]{gfx/wyniki/MCTSQW/110624/110624_LL30K.png}
	\vspace{-10pt}
	\caption{\textit{Landau level graph of sample A as a function of magnetic field at $T$ = 30 K. Colored lines represent Landau levels, characterized by a different value of $n$, as described as indices of $L$ on the right side of the graph. The Landau levels with $n$ = -2 and $n$ = 0 are zero-mode Landau levels. The energy of those levels at zero magnetic field is an indication of the band gap. The vertical arrows with corresponding greek letters represent observed transitions between Landau levels in this system. Solid arrows $\alpha_1$, $\alpha_2$, and $\beta$ indicate interband transitions.}}
	\label{fig:LL_110624_30K}
\end{figure}

\begin{figure}[ht]
	\centering
	\includegraphics[width=12cm]{gfx/wyniki/MCTSQW/110624/110624_spectra30K.png}
	\vspace{-10pt}
	\caption{\textit{\textbf{Left Panel:} Points corresponding to the minima of transmission for sample A at $T$ = 30 K with theoretical predictions of transitions as a function of magnetic field. \textbf{Right panel:} Transmission spectra plotted for magnetic fields in range from 1 to 16 T every 1 T, with symbols corresponding to the transitions from left panel.}}
	\label{fig:Spectra_110624_30K}
\end{figure}

\clearpage
\myparagraph{Temperature 70 K}
\wciecie
At $T$ = 70 K the band structure of sample A changed even more. The energy difference of zero-mode LLs at zero magnetic field got a higher -- the band gap is equal to 23.2 meV. The LL structure is presented in Figure \ref{fig:LL_110624_70K}.

\begin{figure}[ht]
	\centering
	\includegraphics[width=10cm]{gfx/wyniki/MCTSQW/110624/110624_LL70K.png}
	\vspace{-10pt}
	\caption{\textit{Landau level graph of sample A as a function of magnetic field at $T$ = 70 K. Colored lines represent Landau levels, characterized by a different value of $n$, as described as indices of $L$ on the right side of the graph. The Landau levels with $n$ = -2 and $n$ = 0 are zero-mode Landau levels. The energy of those levels at zero magnetic field is an indication of the band gap. The vertical arrows with corresponding greek letters represent observed transitions between Landau levels in this system. Solid arrows $\alpha_1$, $\alpha_2$, and $\beta$ indicate interband transitions.}}
	\label{fig:LL_110624_70K}
\end{figure}

\begin{figure}[ht]
	\centering
	\includegraphics[width=12cm]{gfx/wyniki/MCTSQW/110624/110624_spectra70K.png}
	\vspace{-10pt}
	\caption{\textit{\textbf{Left Panel:} Points corresponding to the minima of transmission for sample A at $T$ = 70 K with theoretical predictions of transitions as a function of magnetic field. \textbf{Right panel:} Transmission spectra plotted for magnetic fields in range from 1 to 16 T every 1 T, with symbols corresponding to the transitions from left panel.}}
	\label{fig:Spectra_110624_70K}
\end{figure}

\clearpage
\subsubsection{Sample B}
\myparagraph{Temperature 2 K}
\wciecie

Those transitions are marked in Figure \ref{fig:LL_SQW_2K} by small greek letters and solid and dashed arrows to differentiate the intraband and interband transitions.

The most interesting feature of LL structure presented in Figure \ref{fig:LL_SQW_2K} is the crossing of the lower level of the conduction band (with hole-like symmetry) with index $n$ = -2, and the upper level of the valence band (with electron-like symmetry) with index $n$ = 0, which takes place at the magnetic field value of $B_c \approx 5$ T. Above this critical field, the topologically insulating phase vanishes and the structure becomes a conventional quantum Hall insulator. The magnetic field breaks the time-reversal symmetry, making the edge states no longer protected and can open an energy gap in the spectrum.

The transition designated as $\alpha$' between LLs $n$ = -2 and $n$ = 1 is not allowed by symmetry and should not be observed in an ideal, axially symmetrical system. However, if the symmetry is broken, the two levels that cross themselves at $B = B_c$ can interact and effectively mix, rendering the transition $\alpha$' allowed, thus possibly making it visible in transmission spectra.

Besides the $\alpha$' transition there are other ones. The one with the largest intensity is denoted as $\alpha$. It represents an optical transition between LLs $n$ = 0 and $n$ = 1. This is also the only interband transiton visible at 2 K. Its interpolation to zero magnetic field gives an idea about the value of energy gap in the system. Two remaining transitions are $\gamma$ and $\delta$. The $\gamma$ transition is between $n$ = -1 and $n$ = 0. The $\delta$ transition, commonly known as the cyclotron resonance, is between $n$ = 0 and $n$ = 1.

\begin{figure}[ht]
	\centering
	\includegraphics[width=14cm]{gfx/wyniki/MCTSQW/091222C/LL2K.png}
	\vspace{-10pt}
	\caption{\textit{Landau level graph of sample B as a function of magnetic field at $T$ = 2 K. Colored lines represent Landau levels, characterized by a different value of $n$, as described as indices of $L$ on the right side of the graph. The Landau levels with $n$ = -2 and $n$ = 0 are zero-mode Landau levels. The energy of those levels at zero magnetic field is an indication of the band gap. Their crossing indicates the point of field induced semiconductor-metal-semiconductor phase transition, and assures that the sample is in the inverted band order phase. The vertical arrows with corresponding greek letters represent observed transitions between Landau levels in this system. The solid arrows indicates interband transitions $\alpha$ and $\alpha'$, while dashed arrows indicate intraband transitions $\beta$, $\gamma$, and $\delta$.}}
	\label{fig:LL_SQW_2K}
\end{figure}

The expected dependence of transition as a function of magnetic field can be derived from the difference of the calculated LLs. In the left panel of Figure \ref{fig:Summary_SQW_2K} these expected transitions are presented along with experimental points taken from the spectra. Open squares, as well as dashed arrows on LL plot in Figure \ref{fig:LL_SQW_2K}, represent intraband transitions, while full points (and solid arrows) represent interband transitions. At $T$ = 2 K the only interband transition is the transition $\alpha$.

\begin{figure}[ht]
	\centering
	\includegraphics[width=10cm]{gfx/wyniki/MCTSQW/091222C/Map_HgTeSQW_091222C_2K.png}
	\vspace{-10pt}
	\caption{\textit{Obsadz}}
	\label{fig:Map_SQW_2K}
\end{figure}

The anticrossing of transitions $\alpha$ and $\alpha$' was already observed and reported \cite{Orlita_MCT_QW}\cite{Zholudev_MCT_QW}\cite{Zholudev_MCT_QW_anticrossing}. In our results, the anticrossing is visible at magnetic fields close to the calculated $B_c \approx 5$ T, as presented in Figure \ref{fig:Map_SQW_2K}. However, the anticrossing is visible for magnetic fields from around 4.5 T up to around 8 T. This evident lack of symmetry in regard to $B_c$ is caused by the position of chemical potential in the sample. Below 4.5 T there are no empty states at LL $n$ = 1 so the probability of both $\alpha$ and $\alpha$' transitions is equal to zero. For the same reason the $\beta$ transition starts to be visible from energies higher than 50 meV, which takes place at magnetic field of around 8 T.

The experimental points are in a fair agreement with the theoretical calculations. However, small discrepancies are unquestionable. The model... 

\begin{figure}[ht]
	\centering
	\includegraphics[width=14cm]{gfx/wyniki/MCTSQW/091222C/Summary2K.png}
	\vspace{-10pt}
	\caption{\textit{\textbf{Left Panel:} Points corresponding to the minima of transmission for sample B at $T$ = 2 K with theoretical predictions of transitions as a function of magnetic field. Solid symbols represent interband transitions, while open symbols represent intraband transitions. \textbf{Right panel:} Transmission spectra plotted for magnetic fields in range from 1 to 16 T every 1 T, with symbols corresponding to the transitions from left panel.}}
	\label{fig:Summary_SQW_2K}
\end{figure}

In the inset of the left panel of Figure \ref{fig:Summary_SQW_2K} there is a scheme of dispersion relation of the sample, where the band inversion is demonstrated directly. The value of the direct band gap is estimated to be $E_g \approx $ 26 meV. It is important to remember that at this temperature the $H1$ band is energetically higher that the $E1$ band, thus the band gap should be considered as negative.

The right panel of Figure \ref{fig:Summary_SQW_2K} shows the spectra taken every 1 T, shifted vertically for clarity. Every minimum was marked with a colored symbol corresponding to the scheme presented on the left panel of the figure. The intensity of $\alpha$ transition dominates the spectra for the intermediate values of magnetic field, while the cyclotron resonance ($\gamma$) dominates the low field region and $\beta$ transition has an equal intensity at high fields. The effective mass of electrons, obtained from the linear fitting of $\delta$ line, is equal to $m^* = eB/\omega_c = 0.025 \pm 0.011$ $m_0$.

\clearpage
\myparagraph{Temperature 50 K}
\wciecie
At temperature 50 K the band structure, presented on Figure \ref{fig:LL_SQW_50K}, is visibly different in comparison to the band structure at 2 K. The band gap is smaller and the crossing of LLs with indices $n$ = -2 and $n$ = 0 takes place at smaller magnetic field $\approx$ 3 T. This is the reason why the anticrossing should not be visible -- the chemical potential lies above the LL $n$ = 1, thus excitations to this level are forbidden. However, $T$ = 2 K is not the only temperature, at which the anticrossing is visible. This is the case for higher temperatures as well. Even at $T$ = 30 K, a faint sign of an anticrossing can be traced on spectra. This is presented in a proper appendix of this work.

There are four transitions visible on the spectra -- one interband $\alpha$ and three intraband $\beta$, $\gamma$, and $\delta$. The energy gap, pointed by the $\alpha$ transition at zero magnetic field, is still negative and equal to 13 meV, which is considerably smaller than the energy gap at $T$ = 2 K.

\begin{figure}[ht]
	\centering
	\includegraphics[width=14cm]{gfx/wyniki/MCTSQW/091222C/LL50K.png}
	\vspace{-10pt}
	\caption{\textit{Landau level graph of sample B as a function of magnetic field at $T$ = 50 K. Colored lines represent Landau levels, characterized by a different value of $n$, as described as indices of $L$ on the right side of the graph. The Landau levels with $n$ = -2 and $n$ = 0 are zero-mode Landau levels. The energy of those levels at zero magnetic field is an indication of the band gap. Their crossing indicates the point of field induced semiconductor-metal-semiconductor phase transition, and assures that the sample is in the inverted band order phase. The vertical arrows with corresponding greek letters represent observed transitions between Landau levels in this system. A solid arrow indicates an interband transition $\alpha$, while dashed arrows indicate intraband transitions $\beta$, $\gamma$, and $\delta$.}}
	\label{fig:LL_SQW_50K}
\end{figure}

The transitions $\alpha$ and $\beta$ are relatively close to each other, which makes them difficult to distinguish at some point, as presented on the right panel in Figure \ref{fig:Summary_SQW_50K}. It can be assumed that transition $\alpha$ starts to be detectable around $B$ = 5 T, while transition $\beta$ around $B$ = 9 T. Both of them are visible up to at least 16 T. The two lower in energy transitions, namely $\gamma$ and $\delta$, are visible in intermediate fields and low fields, respectively.
\begin{figure}[ht]
	\centering
	\includegraphics[width=14cm]{gfx/wyniki/MCTSQW/091222C/Summary50K.png}
	\vspace{-10pt}
	\caption{\textit{\textbf{Left Panel:} Points corresponding to the minima of transmission for sample B at $T$ = 50 K with theoretical predictions of transitions as a function of magnetic field. Solid symbols represent interband transitions, while open symbols represent intraband transitions. \textbf{Right panel:} Transmission spectra plotted for magnetic fields in range from 1 to 16 T every 1 T, with symbols corresponding to the transitions from left panel.}}
	\label{fig:Summary_SQW_50K}
\end{figure}

The effective mass of electrons, obtained from the linear fitting of $\delta$ line, is equal to $m^* = eB/\omega_c = 0.021 \pm 0.003$ $m_0$.

\clearpage
\myparagraph{Temperature 90 K}
The temperature of 90 K is critical for this structure. The band gap vanishes as the $HH1$ band touches the $E1$ band, as presented on Figure \ref{fig:LL_SQW_90K}. It can be thought as the anticrossing of LL $n$ = 0 and LL $n$ = 1 took place at $B$ = 0 T. Because of that the both transition $\alpha$ and $\beta$ converge at zero energy at zero magnetic field, thus it is impossible to declare if they are interband or intraband. This was marked on the Figure \ref{fig:LL_SQW_90K} by dotted arrows.
\begin{figure}[ht]
	\centering
	\includegraphics[width=14cm]{gfx/wyniki/MCTSQW/091222C/LL90K.png}
	\vspace{-10pt}
	\caption{\textit{Landau level graph of sample B as a function of magnetic field at $T$ = 90 K. Colored lines represent Landau levels, characterized by a different value of $n$, as described as indices of $L$ on the right side of the graph. The Landau levels with $n$ = -2 and $n$ = 0 are zero-mode Landau levels. Their crossing takes place at zero magnetic field, which means that the sample is gapless. The vertical arrows with corresponding greek letters represent observed transitions between Landau levels in this system. It is not possible to determine whether $\alpha$ and $\beta$ are inter- or intraband transitions, so they were marked with a dotted line. Dashed arrows indicate intraband transitions $\gamma$ and $\delta$ .}}
	\label{fig:LL_SQW_90K}
\end{figure}

The transitions $\gamma$ and $\delta$ are still visible. As usual the transition $\gamma$ shows up and vanishes at intermediate fields, while the transition $\gamma$ is present only at low magnetic fields, as shown on the left panel of Figure \ref{fig:Summary_SQW_90K}. In the inset there is presented a dispersion relation at zero energy gap, where the bands form a Dirac cone at $\vec{k}$ = 0. 
\begin{figure}[ht]
	\centering
	\includegraphics[width=14cm]{gfx/wyniki/MCTSQW/091222C/Summary90K.png}
	\vspace{-10pt}
	\caption{\textit{\textbf{Left Panel:} Points corresponding to the minima of transmission for sample B at $T$ = 90 K with theoretical predictions of transitions as a function of magnetic field. Solid symbols represent interband transitions, while open symbols represent intraband transitions. \textbf{Right panel:} Transmission spectra plotted for magnetic fields in range from 1 to 16 T every 1 T, with symbols corresponding to the transitions from left panel.}}
	\label{fig:Summary_SQW_90K}
\end{figure}
The transition $\alpha$ starts to be visible at around $B$ = 8 T, while the transition $\beta$ as low as $B$ = 4 T. $T$ = 90 K is the first temperature where the transition where the transition $\beta$ starts to be visible before the transition $\alpha$. Also at this temperature $\beta$ has higher energy at relevant magnetic fields than $\alpha$. The transition $\gamma$ is visible at intermediate fields, and the transition $\delta$ at low. The effective mass of electrons, obtained from the linear fitting of $\delta$ line, is equal to $m^* = eB/\omega_c = 0.023 \pm 0.012$ $m_0$. 

The transitions $\alpha$ and $\beta$ are further separated than at $T$ = 50 K, and their intensity is comparable, as presented on right panel of Figure \ref{fig:Summary_SQW_90K}.

\clearpage
\myparagraph{Temperature 110 K}
\begin{figure}[ht]
	\centering
	\includegraphics[width=14cm]{gfx/wyniki/MCTSQW/091222C/LL110K.png}
	\vspace{-10pt}
	\caption{\textit{Landau level graph of sample B as a function of magnetic field at $T$ = 110 K. Colored lines represent Landau levels, characterized by a different value of $n$, as described as indices of $L$ on the right side of the graph. The Landau levels with $n$ = -2 and $n$ = 0 are zero-mode Landau levels. The energy of those levels at zero magnetic field is an indication of the band gap. The vertical arrows with corresponding greek letters represent observed transitions between Landau levels in this system. A solid arrow indicates an interband transition $\beta$, while dashed arrows indicate the intraband transitions $\alpha$, $\gamma$, and $\delta$.}}
	\label{fig:LL_SQW_110K}
\end{figure}
\begin{figure}[ht]
	\centering
	\includegraphics[width=14cm]{gfx/wyniki/MCTSQW/091222C/Summary110K.png}
	\vspace{-10pt}
	\caption{\textit{\textbf{Left Panel:} Points corresponding to the minima of transmission for sample B at $T$ = 110 K with theoretical predictions of transitions as a function of magnetic field. Solid symbols represent interband transitions, while open symbols represent intraband transitions. \textbf{Right panel:} Transmission spectra plotted for magnetic fields in range from 1 to 16 T every 1 T, with symbols corresponding to the transitions from left panel.}}
	\label{fig:Summary_SQW_110K}
\end{figure}

The effective mass of electrons, obtained from the linear fitting of $\delta$ line, is equal to $m^* = eB/\omega_c = 0.022 \pm 0.010$ $m_0$.

\clearpage
\myparagraph{Temperature 130 K}
\begin{figure}[ht]
	\centering
	\includegraphics[width=14cm]{gfx/wyniki/MCTSQW/091222C/LL130K.png}
	\vspace{-10pt}
	\caption{\textit{Landau level graph of sample B as a function of magnetic field at $T$ = 130 K. Colored lines represent Landau levels, characterized by a different value of $n$, as described as indices of $L$ on the right side of the graph. The Landau levels with $n$ = -2 and $n$ = 0 are zero-mode Landau levels. The energy of those levels at zero magnetic field is an indication of the band gap. The vertical arrows with corresponding greek letters represent observed transitions between Landau levels in this system. A solid arrow indicates an interband transition $\beta$, while dashed arrows indicate the intraband transitions $\alpha$, $\gamma$, and $\delta$.}}
	\label{fig:LL_SQW_130K}
\end{figure}
\begin{figure}[ht]
	\centering
	\includegraphics[width=14cm]{gfx/wyniki/MCTSQW/091222C/Summary130K.png}
	\vspace{-10pt}
	\caption{\textit{\textbf{Left Panel:} Points corresponding to the minima of transmission for sample B at $T$ = 130 K with theoretical predictions of transitions as a function of magnetic field. Solid symbols represent interband transitions, while open symbols represent intraband transitions. \textbf{Right panel:} Transmission spectra plotted for magnetic fields in range from 1 to 16 T every 1 T, with symbols corresponding to the transitions from left panel.}}
	\label{fig:Summary_SQW_130K}
\end{figure}

The effective mass of electrons, obtained from the linear fitting of $\delta$ line, is equal to $m^* = eB/\omega_c = 0.021 \pm 0.008$ $m_0$.

\clearpage
\subsection{Summary}
\wciecie

velocity \cite{Ludwig_MCT_QW}

mass \cite{Orlita_MCT_QW}

\begin{figure}[ht]
	\centering
	\includegraphics[width=14cm]{gfx/wyniki/MCTSQW/110624/110624_AllTemp.png}
	\vspace{-10pt}
	\caption{\textit{Obsadz}}
	\label{fig:110624_AllTemp}
\end{figure}

\begin{figure}[ht]
	\centering
	\includegraphics[width=14cm]{gfx/wyniki/MCTSQW/091222C/091222C_AllTemp.png}
	\vspace{-10pt}
	\caption{\textit{Obsadz}}
	\label{fig:091222C_AllTemp}
\end{figure}

universal velocity

kane vs dirac calculations

pressure paper gamma 7 band?

\chapter{MCT thick layers}
\label{chpt:MCT_TL}
\wciecie


\chapter{InAs/GaSb Quantum Wells}
\wciecie
The first system, where QSH phase was theoretically predicted and experimentally observed, is MCT QW. It allows to investigate and understand easily the physics of topological insulators and phase transitions and is still a subject of great interest of scientists. However, despite the remarkable progress of development of the MCT growth method, there still are plenty of technological challenges, preventing from the wide use of HgTe/CdTe QW in device and industry applications. The first most important problem is the difficulty of processing of mercury-based compounds. Because Hg is highly volatile and quickly diffuses in temperatures above $\SI{80}{\degreeCelsius}$ \cite{Daumer_MCT_temperature}, it excludes the use of standard processing technologies. The second important problem is the general use of Hg and Cd elements, which are highly toxic.

There is a system based on III-V semiconductor groups, which can be used as an alternative to MCT-based systems. The growth and processing of III-V semiconductors is well known as well and their toxicity is relatively low compared to MCT. InAs/GaSb QW also exhibit a band inversion and QSH phase, however, the mechanism behind it differs from the one, responsible for band inversion in HgTe/CdTe QW.

\section{General properties}
There is a group of well lattice matched materials, so called 6.1 Å family, which consists of compounds such as InAs, GaSb, and AlSb. The name origins from a common lattice constant value, which is similar for all compounds and equals approximately 6.1 Å at the room temperature. InAs, GaSb, And AlSb crystallize in the zincblende structure. The similarity of the lattice constant allows growing lattice-matched heterostructures, without strain or other lattice-mismatch defects, like dislocations etc. The exact values of the lattice constants are 6.0584 Å for InAs, 6.0959 Å for GaSb, and 6.1355 Å for AlSb at 300 K \cite{Sze}.

Both InAs and GaSb are direct-gap semiconductor, with the exact value of energy gap measured at the $\Gamma$ point is $E_g^{InAs} = 415-0.276 T^2/(T-83)$ for InAs \cite{Fang_InAs}, and $E_g^{GaSb} = 813 - 0.108 T^2/(T-10.3)$ for GaSb \cite{Wu_GaSb}. AlSb has an indirect gap – the gap measured at the $X$ point equals to $E_g^{AlSb} = 1696 - 0.390 T^2/(T+140)$ \cite{Vurgaftman_AlSb}. The values of energy gaps at 300 K are equal to 350 meV for InAs, 727 meV for GaSb, and 1616 meV for AlSb. This wide range of gaps allows to create high energy electron confinement in quantum wells due to the high energy gap in AlSb barriers, which translates to deep quantum well with barriers as high as 1350 meV. The values of energy gap and lattice constants of 6.1 Å family are presented on Figure \ref{fig:BandStructure_InAs_GaSb}.

\begin{figure}[ht]
	\centering
	\includegraphics[width=10cm]{gfx/Introduction/InAs-GaSb/BandStructure_InAs_GaSb.png}
	\vspace{-10pt}
	\caption{\textit{Obsadz}}
	\label{fig:BandStructure_InAs_GaSb}
\end{figure} 

\subsection{Carrier concentration}
Because of an extraordinary depth of quantum well, there are multiple sources of electrons, which can influence the carrier density in the quantum well. Consequently, even not-intentionally doped structures have relatively high carrier concentration, of the order of magnitude of $1 \cdot 10^{12}$ cm$^{-2}$ \cite{Tuttle_InAs_concentration}. This particularly high concentration is mainly caused by three sources -- conventional shallow bulk donors, surface states, and deep donors at or near the interfaces. 
\begin{itemize}
\item \textbf{Conventional shallow donors and modulation doping} \newline 
Because of high barriers a modulation doping technique can be applied in those systems. It means that the donors are placed in the barrier instead of in the quantum well, and supplied electrons tunnel into the well leaving the ionized impurities away from the active part of the quantum well, and consequently reducing scattering and hence enhancing carrier mobility in comparison with bulk-doped quantum well. This method is widely used in structures, where both high concentration and mobility matters.

\item \textbf{Surface states} \newline
The states on the surface also take part in populating the quantum well. Electrons in the layers, placed outside barriers, can tunnel inside the well. The effect exists as a consequence of the extreme depth of the quantum well itself and strongly depends on the energetic location of the surface states and the chemical nature of the surface coverage \cite{Kroemer_review}. The top surface is usually made of either GaSb or InAs cap layers, which prevent against the oxidation of the structure. The cap layer can exhibit very high density of states (of the order of $10^{12}$ cm$^{-2}$), at the energies 0.85 eV below the conduction band edge of the AlSb barrier \cite{Nguyen_SurfaceDonor}. As a consequence the electrons will flow into the well until the electric field in the barrier pulls the surface states down to the same energy as the Fermi level inside the well. The barrier thickness strongly influences the tunneling, hence can suppress this effect.

\item \textbf{Interface donors and defects} \newline
Another source of electrons, responsible for such a high concentration, even in low temperature limit, might be connected with interface donors and/or defects. A study of temperature dependence of the electron concentration suggested that there exist a donor level less than 50 meV above the bottom of the bulk conduction band of InAs, which implies a level below the bottom of the quantum well, and
below the Fermi level in the well at the observed electron concentrations. But in this case only a small fraction of the donors will be ionized, calling for a donor a concentration much higher than the observed electron concentrations, on the order of about $3 \cdot 10^{12}$ donors cm$^{-2}$ per well \cite{Kroemer_review}. The nature of these donors is not clear. Kroemer proposed that they are not ordinary point defects, but are Tamm states at the InAs–AlSb interface \cite{Kroemer_TammStates}, inherent to the band structure of that interface. Alternatively, Shen et al. \cite{Shen_TammStates} proposed that the donors are very deep bulk donor states associated with AlSb antisite defects, that is, Al atoms on Sb sites.
\end{itemize}

\subsection{Carrier mobility}
One of the most interesting properties of InAs is the second highest electron mobility of all semiconductors (the first being InSb). This is a result of one of the smallest electron effective mass, only 0.023 of the free electron mass. This property makes this material very interesting, especially from a device perspective -- high mobility allows to create faster electronics (eg. HEMTs). Also, high mobility improves transport properties -- allows to  observe quantum effects such as Shubnikov-de Haas oscillations or QHE.

There are multiple factors influencing the mobility of carriers in the quantum well. The is a strong dependence on the well width and on the electron sheet concentration, as well as on the quality of growth. Especially at low temperatures, where mobility is limited by impurity and interface scattering, the growth quality plays a very important role \cite{Nguyen_Mobility}.

\begin{itemize}
\item \textbf{Quantum well width} \newline  
Both room-temperature and low-temperature mobilities are significantly reduced in narrow wells due to the dominance of interface roughness scattering. The mobility peaks for well widths around 125 Å, and then decays again, most probably due to the onset of scattering by misfit dislocations nucleated as the quantum well width exceeds the critical layer thickness imposed by the 1.3\% lattice mismatch between InAs and AlSb \cite{Bolognesi_Width}. As a result, the majority of studies have used a 15 nm InAs quantum well for transport studies of this systems [all kroemer + Li, Journal of Crystal Growth 301–302 (2007) 181–184].

\item \textbf{Layers interfaces} \newline
There is an evidence, that the mobility can depend on the interfaces between separate layers themselves. Tuttle found that because both anion and cation change across an InAs-AlSb interface, it is possible to grow such wells with two different types of interfaces, one with an InSb-like bond configuration, the other AlAs-like. Electron mobility and concentration were found to depend very strongly on the manner in which the quantum wells' interfaces were grown, indicating that high mobilities are seen only if the bottom interface is InSb-like. The AlAs type of interfaces can introduce a sheet of donor defects, which increase the scattering, hence lower the mobility \cite{Tuttle_Interface}. Jenichen examined the interfaces in InAs\slash AlSb superlattices via X-ray scattering, and claims that strong interface roughness and intermixing is definitely present at those sites \cite{Jenichen_Interfaces}.

\item \textbf{Buffer} \newline
There were studies showing that not only the active part of QW influences the mobility and transport properties. Li showed by magneto-transport, Atomic Force Microscopy and X-Ray Diffraction that the electron mobility of AlSb/InAs/AlSb quantum well with GaSb buffer is higher than that with AlSb buffer though the surface and crystal qualities of AlSb buffer are better than GaSb buffer. The crystal quality can be increased by improving the growth process. Because of InAs relaxation on AlSb buffer, mismatch dislocations will appear in the InAs layer and the interfaces of InAs QW will get rough, which is suggested as the reason leading to the lower mobility of InAs quantum well grown on AlSb buffer than on GaSb buffer \cite{Li_Buffer}. 

Thomas investigated this matter as well, and found that GaSb buffers provide atomically flat interfaces on the scale of the electron Fermi wavelength for the quantum wells. In contrast, AlSb buffers generate a very rough interface on the same scale. The low temperature mobility of their samples with GaSb buffer ($\mu = 240000$ cm$^2$/Vs) was seven times greater than of the samples with the AlSb buffer ($\mu = 35000$ cm$^2$/Vs), for concentration $n = 5.5 \cdot 10^{11}$ cm$^{-2}$. For the concentration of $n = 1.3 \cdot 10^{12}$ cm$^{-2}$ the difference was a bit smaller, but the mobilities were enormous $\mu = 944000$ cm$^2$/Vs and $\mu = 244000$ cm$^2$/Vs for GaSb and AlSb respectively \cite{Nguyen_Mobility}\cite{Thomas_Buffer}\cite{Nguyen_Buffer}.

\item \textbf{Carrier concentration} \newline
There are generally two ways to change the electron concentration of the sample -- with and without altering the sample structure. The first is doping, which was already mentioned. Doping with donors, provides additional carriers into the structure, which tunnel into the QW and increase its carrier density. On the other hand it introduces scattering centers, which gradually lower the concentration, even with a modulation doping technique \cite{Nguyen_Buffer}. 

The latter is based on gating technique or photoexcitation effects like persistent photoconductivity effect, which will be discussed later. Structure with a gate allow to continuously tune the carrier concentration in a relatively broad rage. However there is a strong relation between electron concentration and mobility. Nguyen et al investigated this dependence in gated InAs/AlSb quantum wells and found that the dependence is not monotonic -- mobility increases as the first subband gets populated, peaks, and has a minimum when the second subband is populated, then increases again \cite{Nguyen_Mobility}.

\end{itemize}

\section{Band ordering}
The title of the most bizarre lineup of the 6.1 Å family belongs to the InAs/GaSb heterojunctions. It was found already back in 1977 by Sasaki et al. \cite{Sasaki_BandOrdering}, that they exhibit a broken gap band ordering -- at the interface, the bottom of the condunction band of InAs has lower energy states than the valence band states of GaSb, with a break gap of 150 meV. This discovery most likely attracted that much attention and interest in the entire 6.1 Å family. This particular band ordering was also predicted theoretically -- in the same year Frensley \cite{Frensley_BandOrdering} had claimed that there is a possibility of such a lineup to exist. Also, Harrison using LCAO theory \cite{Harrison_BandOrdering} suggested a similar prediction. Quantum wells of InAs/GaSb sandwiched between layers of AlSb were named broken-gap quantum wells (BGQW) or composite quantum wells (CQW). The band structure of InAs/GaSb CQW is presented on Figure \ref{fig:BandStructure_InAs_GaSb2}A. The valence band of AlSb lies about 0.4 eV lower than the valence band of GaSb, while the conduction band of AlSb lies approximately 0.4 eV higher than the conduction band of AlSb. As a consequence, AlSb can be used as a quantum well barrier in CQW systems.

\begin{figure}[ht]
	\centering
	\includegraphics[width=10cm]{gfx/Introduction/InAs-GaSb/BandStructure_InAs_GaSb2.png}
	\vspace{-10pt}
	\caption{\textit{Obsadz}}
	\label{fig:BandStructure_InAs_GaSb2}
\end{figure} 

The quantum well width governs two fundamentally different regimes of the system. Similarly to MCT QW the energy position of levels in a QW depends strongly on QW width. However, in the case of InAs/GaSb CQW, where none of the compounds used has an inverted structure by itself, the relative position of $E_1$ in InAs well and $H_1$ in GaSb well decides whether the CQW is in an inverted regime or not. For thin layers $E_1$ lies higher in energy than $H_1$ and the structure is in the normal state. On the other hand, for sufficiently thick layers the band order is reversed. In the past, the inverted regime was considered as a semimetallic \cite{Chang_BandStructure}. 

Since the transition between an inverted regime and a normal regime has to be smooth and continuous, there must be a point, where the $E_1 = H_1$, so the momentum and carrier energy in the two wells are close to be equal. In this state the system is strongly coupled, and both the electron states and the hole states are mixed. This mixing leads to hybridization of the bands or, simply, to lifting of degeneracy of the levels. As a consequence, in analogy to bonding and antibonding states, a relativly small (of the order of a few of meV) hybridization gap appears. Thus, the semimetallic band dispersion relation in Figure \ref{fig:BandStructure_InAs_GaSb2}B (dashed), becomes nonmonotonic (full line), with a mini-gap $\Delta$. Because of this, a InAs/GaSb CQW in an inverted regime is not a semimetal, but has a gap \cite{Altarelli_BandStructure}. The presence of the gap means that the bulk of the sample is resisitive, as long as the Fermi energy lies within the gap, which is a case in a theoretical intrinsic case. The existence of the gap was discovered experimentally by Yang \cite{Yang_BandStructure} and Lakrimi \cite{Lakrimi_BandStructure}. The presence of the hybridization gap makes this system different from HgTe/CdTe QW. This is a consequence of inversion asymmetry present in InAs/GaSb CQW -- the QW is not symmetrical in the growth direction, which is not the case for HgTe/CdTe systems.

Because of the broken gap alignment, the CQW in the inverted regime should have equal intrinsic density of both two-dimensional electron gas (2DEG) and two-dimensional hole gas (2DHG). However, in the actual samples there are localized states at interfaces, defects or inhomogeneities, which potentially lead to unbalanced concentrations of carriers. Magnetotransport studies indicate that the electron density dominates over the hole density \cite{Petchsingh_BandOrdering}\cite{Munekata_BandOrdering}, which is partially caused by surplus of electrons, tunneled from the high surface states. As a consequence, the Fermi level is set around 130 meV above the bottom of the conduction band of InAs \cite{Nguyen_Mobility}, which is different from its bulk value.

\subsection{Magnetic field tunability}
Magnetic field can be used to induce a electron and hole recombination in GaSb/InAs QW in inverted regime. Energy spectrum of 2DEG and 2DHG in magnetic field tends to quantize into a discrete levels called Landau levels. Energy of these levels increases for electrons and decreases for holes as the magnetic field gets stronger, which is a consequence of the opposite sign of carriers. Works of Lo \cite{Lo_MagneticField} and Smith \cite{Smith_MagneticField} show that it is possible, by applying external magnetic field, to bring the hole Landau level and the electron Landau level to the same value of energy, thus effectively crossing them and force electrons to recombine with holes, which leads to semiconductor to semimetal phase transition.

\subsection{Electric field tunability}
There were studies showing that band relative position can be altered also with electric field. Naveh et al. Found that applying small electric fields in a growth direction of AlSb-InAs-GaSb-AlSb QW, any value can be achieved for such parameters as the energy gaps, effective masses, and carrier types and densities in the material, which is a consequence of a strong electron-hole coupling in this system \cite{Naveh_ElectricField}. The electron and hole levels in this system, when subjected to an external electrical field, usually by a metallic gate on the top (and/or the bottom) of the structure, shift in opposite directions, approaching each other and eventually crossing. In the intrinsic case, thus when $n_H$ = $n_E$, Fermi energy lies somewhere between the first electron level ($E_1$) and the first hole level ($H_1$), and with increasing electric field $H_1$ crosses it first, removing the holes from the system, and $E_1$ follows at higher voltages applied between the gate and the channel, depleting completely the region of carriers. It gives a possibility to control the concentrations of carriers by relatively small electric field and in consequence, induce a semimetal to semiconductor phase transition \cite{Mendez_ElectricField}.

However, to control both the band structure and the Fermi level position, there is a need to use two gate configuration. A top gate and a back gate are used to tune the concentration of electrons and holes, respectively. Two gate configuration is required because of the screening of top gate induced electric field by electrons in InAs layer. Also, a very important issue is resolved in this way – electron and hole densities can be tuned almost indepentently, with respect to the total charge in the system. This idea was used by Cooper \cite{Cooper_ElectricField} to study a coupling of electrons and holes in InAs/GaSb/AlSb systems as a function of temperature and concentration. The two gate configuration was used as a concept in a theoretical work of Liu, in which the idea of a topological insulator based on a InAs/GaSb/AlSb QW was introduced \cite{Liu_Topology}.

\section{Trilayer QW}
Since the prediction of QSH effect in InAs/GaSb CQW most of the investigation considered only two layered (InAs layer and GaSb) structure. This introduced an inversion asymmetry into the structure, which does not exist in the case of HgTe/CdTe QW (which is symmetrical). As a consequence, the crossing of $E_1$ and $H_1$ subbands did not occur at the $\Gamma$ point of the Brillouin zone \cite{Murakami_Trilayer} (\ref{fig:BandStructure_InAs_GaSb2}), which had further implications on the shape of Landau levels.

In order to eliminate the problem of asymmetry in a growth direction, Kristophenko [] proposed to attach an additional layer of InAs or GaSb to the structure, which restores the inversion symmetry in the system and brings the crossing of the $E_1$ and $H_1$ subbands at the $\Gamma$ point, as in the case of HgTe/CdTe QW. The layer order and the band structure for InAs-designed (with an additional InAs layer) and GaSb-designed (with an additional GaSb layer) quantum wells are presented on Figure \ref{fig:TrilayerBandStructure1}.

\begin{figure}[ht]
	\centering
	\includegraphics[width=12cm]{gfx/Introduction/InAs-GaSb/TrilayerBandStructure1.png}
	\vspace{-10pt}
	\caption{\textit{Obsadz}}
	\label{fig:TrilayerBandStructure1}
\end{figure} 

This new class of structures, based on InAs/GaSb, which differ from the broken-gap QWs by the band-gap inversion at the $\Gamma$ point. As hybridization between electron-like and heavy-hole-like subbands vanishes at \textbf{\textit{k}} = 0, the crossing of $E_1$ and $H_1$ subbands at the $\Gamma$ point results in the gapless state with 2D massless DFs.

These materials have zincblende lattice structure and direct energy gaps in the vicinity of $\Gamma$ point, thus the system can be well described by 8-band Kane model \cite{Kane_Model}, like in the case of a BGQW \cite{Liu_Topology}. However, trilayer structure can be considered as a double QW with a middle barrier. In the case of InAs-designed QW, a GaSb layer in the middle plays a role of barrier separating two QW for electrons, and at the same time is a single QW for holes. On the other hand, GaSb-designed QW has a barrier of InAs in the middle, separating two QW for holes, being a single QW for electron itself at the same time. The energy levels in the wells are sensitive to the QW width, like in the case of QW mentioned before. Since the relative position of $E_1$ and $H_1$ subbands decide whether the whole structure is inverted or not, it means that there is a possibility to drive the system from inverted regime to normal, simply by changing the QW width.  

\section{Topological properties}
The band structure of InAs/GaSb QW is inverted because of the unique relative band alignment of InAs and GaSb layers. As a consequence, the valence band of GaSb is higher energetically than the conduction band of InAs. A smooth connection of bands inside and outside of the sample with a matter of different topology (a regular one, e.g. vacuum or trivial insulator) leads to a rise of gapless states at the boundary with a linear dispersion relation. The mechanism of a band inversion differs from the one, which causes the inversion of the band structure of HgTe/CdTe QW, where the SOC was responsible for pushing the $\Gamma_8$ bands above the $\Gamma_6$ bands. 


%From the topological insulator point of view a InAs/GaSb CQW, for a particular space of parameters, exhibits an inverted band order phase with gapless boundary states described by a linear dispersion relation. The bulk of the sample is insulating, because there is a small gap. Those boundary states are protected by the time-reversal symmetry from elastic backscattering and are topologically robust to small perturbations. This means that InAs/GaSb QW should be in a topologically nontrivial QSH phase.



\begin{thebibliography}{99}
%Introduction topology
\bibitem{Anderson_Topology}
Anderson P. W. (1997), Basic Notions of Condensed Matter Physics (Westview Press).
\bibitem{Landau_Topology}
Landau L. D., and E. M. Lifshitz (1980), Statistical Physics (Pergamon Press, Oxford).
\bibitem{Josephson_Topology}
B. D. Josephson, Possible new effects in superconductive tunneling, \textit{Phys. Lett.} \textbf{1} pp. 251–253 
\bibitem{Klitzing_Topology}
K. v. Klitzing, G. Dorda and M. Pepper, New Method for High-Accuracy Determination of the Fine-Structure Constant Based on Quantized Hall Resistance, \textit{Phys. Rev. Lett.} \textbf{45}, 494 (1980).
\bibitem{Tsui_Topology}
D. C. Tsui, H. L. Stormer and A. C. Gossard, Two-Dimensional Magnetotransport in the Extreme Quantum Limit, \textit{Phys. Rev. Lett.} \textbf{48}, 1559 (1982).
\bibitem{Laughlin_Topology}
R. B. Laughlin, Quantized Hall conductivity in two dimensions, \textit{Phys. Rev. B.} \textbf{23}, 5632 (1981).
\bibitem{Thouless_Topology}
Thouless D. J., M. Kohmoto, M. P. Nightingale, and M. den Nijs, \textit{Phys. Rev. Lett.} \textbf{49}, 405 (1982).
\bibitem{Zawadzki_Topology}
W. Zawadzki, Narrow Gap Semiconductors Physics and Application, (1979).
\bibitem{Kane_Topology}
C. L. Kane and E. J. Mele. Quantum Spin Hall Effect in Graphene, \textit{Phys. Rev. Lett.} \textbf{95}, 226801 (2005).
\bibitem{Kane_Topology2}
C. L. Kane and E. J. Mele. $\mathbb{Z_2}$ Topological Order and the Quantum Spin Hall Effect. \textit{Phys. Rev. Lett.} \textbf{95}, 146802 (2005).
\bibitem{Bernevig_Topology2}
B. A. Bernevig, T. L. Hughes, S.-C. Zhang, Quantum Spin Hall Effect and Topological Phase Transition in HgTe Quantum Wells, \textit{Science} \textbf{314}, 5806, pp. 1757-1761 (2006).
\bibitem{Bernevig_Topology1}
B. A. Bernevig, S.-C. Zhang, Quantum Spin Hall Effect, \textit{Phys. Rev. Lett.} \textbf{96}, 106802 (2006).
\bibitem{Yao_Topology}
Y. Yao, F. Ye, X.-L. Qi, S.-C. Zhang, and Z. Fang, Spin-orbit gap of graphene: First-principles calculations, \textit{Phys. Rev. B.} \textbf{75}, 041401(R) (2007).
\bibitem{Min_Topology}
H. Min, J. E. Hill, N. A. Sinitsyn, B. R. Sahu, L. Kleinman, and A. H. MacDonald, Intrinsic and Rashba spin-orbit interactions in graphene sheets, \textit{Phys. Rev. B.} \textbf{74}, 165310 (2006).
\bibitem{Konig_Topology}
M. König, \textit{et al.}, Quantum Spin Hall Insulator State in HgTe Quantum Wells, \textit{Science} \textbf{318}, 5851, pp. 766-770 (2007).
\bibitem{Konig_MCT_SQW}
M. König, \textit{et al.} The Quantum Spin Hall Effect: Theory and Experiment. \textit{J. Phys. Soc. Jpn} \textbf{77}, 031007 (2008).
\bibitem{Liu_Topology}
C. Liu, T. L. Hughes, X.-L. Qi, K. Wang and S.-C. Zhang, \textit{Phys. Rev. Lett.} \textbf{100}, 236601 (2008).
\bibitem{Fu_Topology}
L. Fu, C. L. Kane and E. J. Mele, Topological Insulators in Three Dimensions, \textit{Phys. Rev. Lett.} \textbf{98}, 106803 (2007).
\bibitem{Hsieh}
D. Hsieh \textit{et al.}, A topological Dirac insulator in a quantum spin Hall phase, \textit{Nature} \textbf{452}, pp.970-974 (2008).
\bibitem{Zhang_Topology}
H. Zhang \textit{et al.}, Topological insulators in Bi$_2$Se$_3$, Bi$_2$Te$_3$ and Sb$_2$Te$_3$ with a single Dirac cone on the surface, \textit{Nature Physics} \textbf{5}, pp.438-442 (2009).
\bibitem{Xia_Topology}
Y. Xia \textit{et al.}, Observation of a large-gap topological-insulator class with a single Dirac cone on the surface, \textit{Nature Physics} \textbf{5}, pp.398-402 (2009).
\bibitem{Chen_Topology}
Y. L. Chen \textit{et al.}, Experimental Realization of a Three-Dimensional Topological Insulator, Bi$_2$Te$_3$,  \textit{Science} \textbf{325}, 5937 pp.178-181 (2009).


%state of art
\bibitem{Halperin_State}
B. I. Halperin, Quantized Hall conductance, current-carrying edge states, and the existence of extended states in a two-dimensional disordered potential. \textit{Phys. Rev. B.} \textbf{25}, 2185 (1982).
\bibitem{Murakami_State}
S. Murakami, N. Nagaosa, and S.-C. Zhang, Spin-Hall Insulator. \textit{Phys. Rev. Lett.} \textbf{93}, 156804 (2006).
\bibitem{Roth_State}
A. Roth  \textit{et al.}, Nonlocal Transport in the Quantum Spin Hall State. \textit{Science} \textbf{17}, 5938, pp. 294-297 (2009).
\bibitem{Buttiker_State}
M. Büttiker, Edge-State Physics Without Magnetic Fields. \textit{Science} \textbf{325}, 5938, pp. 278-279 (2009).
\bibitem{Brune_State}
C. Brüne \textit{et al.}, Spin polarization of the quantum spin Hall edge states. \textit{Nature Physics} \textit{8}, pp. 485-490 (2012).
\bibitem{Hankiewicz_State}
E. M. Hankiewicz, L. W. Molenkamp, T. Jungwirth, and J. Sinova, Manifestation of the spin Hall effect through charge-transport in the mesoscopic regime. \textit{Phys. Rev. B.} \textbf{70}, 241301(R) (2004).  
\bibitem{Valenzuela_State}
S. O. Valenzuela, and M. Tinkham, Direct electronic measurement of the spin Hall effect. \textit{Nature} \textit{442}, pp. 176-179 (2006).
\bibitem{Huber_State}
M. E. Huber \textit{et al.}, Gradiometric micro-SQUID susceptometer for scanning measurements of mesoscopic samples. \textit{Rev. Sci. Instrum.} \textbf{79}, 053704 (2008).
\bibitem{Nowack_State}
K. C. Nowack \textit{et al.}, Imaging currents in HgTe quantum wells in the quantum spin Hall regime. \textit{Nature Materials} \textbf{12}, pp. 787-791 (2013).
\bibitem{Spanton_State}
E. M. Spanton \textit{et al.}, Images of Edge Current in InAs/GaSb Quantum Wells. \textit{Phys. Rev. Lett.} \textbf{113}, 026804 (2014).


%MCT
\bibitem{Landauer_MCT}
M. Büttiker, Absence of backscattering in the quantum Hall effect in multiprobe conductors, \textit{Phys. Rev. B} \textbf{38}, 9375 (1988).
\bibitem{Zawadzki_MCT}
W. Zawadzki, Electron transport phenomena in small-gap semiconductors. \textit{Advances in Physics} \textbf{23}, pp. 435–522 (1974).
\bibitem{Novoselow_MCT}
K. S. Novoselov, \textit{et al.}, Electric Field Effect in Atomically Thin Carbon Films. \textit{Science} \textbf{306}, 5696 pp. 666-669 (2004).
\bibitem{Wallace_MCT}
P.R. Wallace, The Band Theory of Graphite. \textit{Phys. Rev.} \textbf{71} 622 (1947).

%MCT bulk
\bibitem{Orlita_MCT}
M. Orlita, \textit{et al.}, Observation of three-dimensional massless Kane fermions in a zinc-blende crystal. \textit{Nature Physics} \textbf{10}, pp. 233-238 (2014).
\bibitem{Malcolm_MCT}
J. D. Malcolm and E. J. Nicol, Magneto-optics of massless Kane fermions: Role of the flat band and unusual Berry phase, \textit{Phys. Rev. B} \textbf{92} 035118 (2015).
\bibitem{Weiler_MCT}
M. H. Weiler, \textit{Semiconductors and Semimetals}. Vol. 16, pp. 119–191, Elsevier (1981).
\bibitem{Capper_MCT}
P. Capper and J. W. Garland, \textit{Mercury Cadmium Telluride: Growth, Properties and Applications}. Wiley Series in Materials for Electronic and Optoelectronic Applications (2010).
\bibitem{Teppe_MCT}
F. Teppe, \textit{et al.}, Tempeature-driven massless Kane fermions in HgCdTe crystals. \textit{Nature Communications} \textbf{7}, 12576 (2016).
\bibitem{Novik_MCT}
E. G. Novik, \textit{et al.}, Band structure of semimagnetic Hg$_{1-y}$Mn$_y$Te quantum wells. \textit{Phys. Rev. B} \textbf{72} 035321 (2005).
\bibitem{Kane_MCT}
E. O. Kane, \textit{Band structure of narrow gap semiconductors.} (1980).
\bibitem{Kane_Model}
E. O. Kane, Band structure of indium antimonide. \textit{J. Phys. Chem. Solids} \textbf{1}, pp. 249-261 (1957).
\bibitem{Jiang_MCT}
Z. Jiang, \textit{et al.}, Infrared spectroscopy of Landau levels of graphene. \textit{Phys. Rev. Lett.} \textbf{98} 197403 (2007).


%GaSb/InAs
\bibitem{Sze}
S. M. Sze, \textit{Physics of Semiconductor Devices.} John Wiley and Sons, New York, second edition (1981).
\bibitem{Fang_InAs}
Z. M. Fang \textit{et al.}, Photoluminescence of InSb, InAs, and InAsSb grown by organometallic vapor phase epitaxy, \textit{J. Appl. Phys.} \textbf{67}, 7034 (1990).
\bibitem{Wu_GaSb}
M. Wu, C. Chen, Photoluminescence of high quality GaSb grown from Ga and Sb rich solutions by liquidphase epitaxy, \textit{J. Appl. Phys.} \textbf{72}, 4275 (1992).
\bibitem{Vurgaftman_AlSb} 
I. Vurgaftman, J. R. Meyer and L. R. Ram-Mohan, Band parameters for III–V compound semiconductors and their alloys, \textit{J. Appl. Phys.} \textbf{89}, 5815 (2001).
\bibitem{Tuttle_InAs_concentration}
G. Tuttle, H. Kroemer and J. H. English, Electron concentrations and mobilities in AlSb/InAs/AlSb quantum wells, \textit{J. Appl. Phys.} \textbf{65}, 5239 (1989).
\bibitem{Kroemer_review}
H. Kroemer, The 6.1 Å family (InAs, GaSb, AlSb) and its heterostructures: a selective review, \textit{Physica E} \textbf{20}, pp. 196-203 (2004). 
\bibitem{Nguyen_SurfaceDonor}
C. Ngyuen, B. Brar, H. Kroemer and J. H. English, Surface donor contribution to electron sheet concentrations in not-intentionally doped InAs-AlSb quantum wells, \textit{Appl. Phys.Lett.} \textbf{60}, 1854 (1992).
\bibitem{Kroemer_TammStates}
H. Kroemer, C. Ngyuen and B. Brar, Are there Tamm-state donors at the InAs–AlSb quantum well interface, \textit{J. Vac. Sci. Technol. B} \textbf{10}, 1769 (1992). 
\bibitem{Shen_TammStates}
J. Shen, H. Goronkin, J. D. Dow and S. Y. Ren, Tamm states and donors at InAs/AlSb interfaces, \textit{J. Appl. Phys.} \textbf{77}, 1576 (1995).
%Mobility
\bibitem{Nguyen_Mobility}
C. Nguyen, \textit{et al.}, Growth of InAs-AlSb Quantum Wells Having Both High Mobilities and High Concentrations, \textit{Journal of Electronic Materials} \textbf{22}, 255-258 (1993).
\bibitem{Bolognesi_Width}
C. R. Bolognesi, H. Kroemer and J. H. English, Well width dependence of electron transport in molecular-beam epitaxially grown InAs/AlSb quantum wells, \textit{J. Vac. Sci. Technol. B} \textbf{10}, 877 (1992).
\bibitem{Tuttle_Interface}
G. Tuttle, H. Kroemer and J. H. English, Effects of interface layer sequencing on the transport properties of InAs/AlSb quantum wells: Evidence for antisite donors at the InAs/AlSb interface, \textit{J. Appl. Phys.} \textbf{67}, 3032 (1990).
\bibitem{Jenichen_Interfaces}
B. Jenichen, S. A. Stepanov, B. Brar and H. Kroemer, Interface roughness of InAs/AlSb superlattices investigated by X-ray scattering, \textit{J. Appl. Phys.} \textbf{79}, 120 (1996).
\bibitem{Li_Buffer}
Z. H. Li, \textit{et al.}, Buffer influence on AlSb/InAs/AlSb quantum wells, \textit{Journal of Crystal Growth}, 181–184 301–302 (2007).
\bibitem{Thomas_Buffer}
M. Thomas, H.-R. Blank, K. C. Wong and H. Kroemer, Buffer-dependent mobility and morphology of InAs/(Al,Ga)Sb quantum wells, \textit{Journal of Crystal Growth}, 175/176 (1997) 849-897.
\bibitem{Nguyen_Buffer}
C. Ngyuen, K. Ensslin and H. Kroemer, Magneto-transport in InAs/AlSb quantum wells with large electron concentration modulation, \textit{Surface Science} \textbf{267}, pp. 549-552 (2007).
%Band Ordering
\bibitem{Sasaki_BandOrdering} 
H. Sasaki, \textit{et al.}, In$_{1-x}$Ga$_x$As-GaSb$_{1-y}$As$_y$ heterojunctions by molecular beam epitaxy, \textit{Appl. Phys. Lett.} \textbf{31}, 211 (1977).
\bibitem{Frensley_BandOrdering}
W. R. Frensley and H. Kroemer, Theory of the energy-band lineup at an abrupt semiconductor heterojunction, \textit{Phys. Rev. B} \textbf{16}, 2642 (1977).
\bibitem{Harrison_BandOrdering}
W. A. Harrison, Elementary theory of heterojunctions, \textit{J. Vac. Sci. Technol.} \textbf{14}, 1016 (1977).
\bibitem{Chang_BandStructure}
L. L. Chang and L. Esaki, Electronic properties of InAs-GaSb superlattices \textit{Surf. Sci.} \textbf{98}, 70 (1980).
\bibitem{Altarelli_BandStructure}
M. Altarelli, Electronic structure and semiconductor-semimetal transition in InAs-GaSb superlattices, \textit{Phys. Rev. B} \textbf{28}, 842 (1983).
\bibitem{Yang_BandStructure} 
M. J. Yang, C. H. Yang, B. R. Bennett and B. V. Shanabrook, Evidence of a Hybridization Gap in "Semimetallic" InAs/GaSb Systems, \textit{Phys. Rev. Lett.} \textbf{78}, 4613 (1997).
\bibitem{Lakrimi_BandStructure}
M. Lakrimi, Minigaps and Novel Giant Negative Magnetoresistance in InAs/GaSb Semimetallic Superlattices, \textit{et al.}, \textit{Phys. Rev. Lett.} \textbf{79}, 3034 (1997).
\bibitem{Petchsingh_BandOrdering}
C. Petchsingh, Cyclotron Resonance Studies on InAs/GaSb heterostructures, Wolfson College, University of Oxford, 2002.
\bibitem{Munekata_BandOrdering}
H. Munekata, E. E. Mendez, Y. Iye and L. Esaki, Densities and mobilities of coexisting electrons and holes in GaSb/InAs/GaSb Quantum Wells, \textit{Surface Science} \textbf{174}, pp. 449-453 (1986).
\bibitem{Lo_MagneticField}
I. Lo, W. C. Mitchel and J.-P. Cheng, Magnetic-field-induced free electron and hole recombination in semimetallic Al$_x$Ga$_{1-x}$Sb/InAs quantum wells, \textit{Phys. Rev. B} \textbf{48}, 9118 (1993).
\bibitem{Smith_MagneticField}
T. P. Smith, H. Munekata, L. L. Chang, F. F. Fang and L. Esaki, Magnetic-Field-Induced Transitions in InAs/Ga$_{1-x}$Al$_x$Sb Heterostructures, \textit{Surface Science} \textbf{196}, pp. 687-693 (1988).
\bibitem{Naveh_ElectricField}
Y. Naveh and B. Laikhtman, Band-structure tailoring by electric field in a weakly coupled electron-hole system, \textit{Appl. Phys. Lett.} \textbf{66} 1980 (1995).
\bibitem{Mendez_ElectricField}
E. E. Mendez, L. L. Chang and L. Esaki, Two-Dimensional Quantum States in Multi-Heterostructures of Three Constituents, \textit{Surface Science} \textbf{13} 474-478 (1982).
\bibitem{Cooper_ElectricField}
L. J. Cooper, \textit{et al.}, Resistance resonance induced by electron-hole hybridization in a strongly coupled InAs/GaSb/AlSb heterostructure \textit{Phys. Rev. B} \textbf{57} 11915 (1998).
%TRILAYER
\bibitem{Murakami_Trilayer}
S. Murakami, \textit{et al.}, Tuning phase transition between quantum spin Hall and ordinary insulating phases. \textit{Phys. Rev. B} \textbf{76} 205304 (2007).
%SAMPLES
\bibitem{Dvoretsky_Samples}
S. Dvoretsky, \textit{et al.}, Growth of HgTe Quantum Wells for IR to THz Detectors. \textit{J. Electron. Mater.} \textbf{39}, 918 (2010). 


\bibitem{Knez1_State}
I. Knez, R. R. Du and G. Sullivan, Finite conductivity in mesoscopic Hall bars of inverted InAs/GaSb quantum wells. \textit{Phys. Rev. B} \textbf{81} 201301(R) (2010).
\bibitem{Naveh_State}
Y. Naveh and B. Laikhthman, Magnetotransport of coupled electron-holes. \textit{Europhys. Lett.} \textbf{55}, pp. 545-551 (2001).
\bibitem{Suzuki_State}
K. Suzuki, Y. Harada, K. Onomitsu, and K. Muraki, Edge channel transport in the InAs/GaSb topological insulating phase. \textit{Phys. Rev. B} \textbf{87}  235311 (2013).
\bibitem{Knez2_State}
I. Knez, R. R. Du and G. Sullivan, Evidence for Helical Edge Modes in Inverted InAs/GaSb Quantum Wells. \textit{Phys. Rev. Lett.} \textbf{107} 136603 (2011).
\bibitem{Zhou_State}
B. Zhou, H.-Z. Lu, R.-L. Chu, S.-Q. Shen, and Q. Niu, Finite Size Effects on Helical Edge States in a Quantum Spin-Hall System. \textit{Phys. Rev. Lett.} \textbf{101} 246807 (2008).
\bibitem{Charpentier_State}
C. Charpentier, \textit{et al.}, Suppression of bulk conductivity in InAs/GaSb broken gap composite quantum wells. \textit{Appl. Phys. Lett.} \textbf{103}, 112102 (2013).
\bibitem{Du_State}
L. Du, I. Knez, G. Sullivan, and R.-R. Du, Observation of Quantum Spin Hall States in InAs/GaSb Bilayers under Broken Time-Reversal Symmetry. \textit{arXiv:1306.1925 [cond-mat.mes-hall]}
\bibitem{Knez3_State}
 I. Knez, \textit{et al.}, Observation of Edge Transport in the Disordered Regime of Topologically Insulating InAs/GaSb Quantum Wells. \textit{Phys. Rev. Lett.} \textbf{112}, 026602 (2014).
\bibitem{Kuzmenko_MCTBulk}
A. B. Kuzmenko, \textit{et al.}, Universal optical conductance of graphite. \textit{Phys. Rev. Lett.} \textbf{100}, 117401 (2008).
\bibitem{Smith_State}
D. L. Smith, and C. Mailhiot, Proposal for strained type II superlattice infrared detectors. \textit{J. Appl. Phys.} \textbf{62}, pp. 2545-2548 (1987). 
\bibitem{Akiho_State}
T. Akiho, \textit{et al.}, Engineering quantum spin Hall insulators by strained-layer heterostructures. \textit{arXiv:1608.06751} (2016).
\bibitem{Du_State2}
L. Du, \textit{et al.}, Strain-Engineering of InAs/GaInSb Topological Insulator Towards Majorana Platform. \textit{arXiv:1608.06588} (2016.)
\bibitem{Brune_State2}
C. Brüne, \textit{et al.}, Quantum Hall Effect from the Topological Surface States of Strained Bulk HgTe. \textit{Phys. Rev. Lett.} \textbf{106}, 126803 (2011).
\bibitem{Pankratov_State}
O. A. Pankratov, Electronic properties of band-inverted heterojunctions: supersymmetry in narrow-gap semiconductors. \textit{Semicond. Sci. Technol.} \textbf{5}, S204 (1990).
\bibitem{Pfeuffer_State}
A. Pfeuffer-Jeschke, \textit{Bandstruktur und Landau-Niveaus quecksilberhaltiger II-IV-Heterostrukturen}, Ph.D. thesis, Universität Würzburg (2000).
\bibitem{Leubner_State}
Leubner
\bibitem{Wiedmann_State}
S. Wiedmann, \textit{et al.}, Temperature-driven transition from a semiconductor to a topological insulator. \textit{Phys. Rev. B}\textbf{91} 205311 (2015).
\bibitem{Orlita_MCT_QW}
M. Orlita, \textit{et al.}, Fine structure of "zero-mode" Landau levels in HgTe/CdTe quantum wells. \textit{Phys. Rev. B} \textbf{83} 115307 (2011).
\bibitem{Zholudev_MCT_QW}
M. Zholudev, \textit{et al.}, Magnetospectroscopy of two-dimensional HgTe-based topological insulators around the critical thickness. \textit{Phys. Rev. B} \textbf{86} 205420 (2012).
\bibitem{Lawson_MCT}
W. D. Lawson, S. Nielson, E. H. Putley, and A. S Young, Preparation and properties of HgTe and mixed crystals of HgTe-CdTe. \textit{J. Phys. Chem. Solids} \textbf{9} 325-329 (1959).
\bibitem{Krishtopenko_pressure}
S. S. Krishtopenko, \textit{et al.}, Pressure- and temperature-driven phase transitions in HgTe quantum wells. \textit{Phys. Rev. B} \textbf{B}, 245402 (2016).
\bibitem{Daumer_MCT_temperature}
V. Daumer, \textit{et al.}, Quasiballistic transport in HgTe quantum-well nanostructures. \textit{Appl. Phys. Lett.} \textbf{83}, 1376-1378 (2003).
\bibitem{Liu_CdAs}
Z. K. Liu, \textit{et al.}, A stable three-dimensional topological Dirac semimetal Cd$_3$As$_2$. \textit{Nature Materials} \textbf{13}, 677-681 (2014).  
\bibitem{Roseman_CdAs}
I. Roseman, Effet Shubnikov de Haas dans Cd$_3$As$_2$: Forme de la surface de Fermi et modele non parabolique de la bande de conduction. \textit{J. Phys. Chem. Solids} \textbf{30}, 1385 (1969).
\bibitem{Bodnar_CdAs}
J. Bodnar, in Proc. III Conf. Narrow-Gap Semiconductors, Warsaw, edited by J. Raułszkiewicz, M. Górska, and E. Kaczmarek (Elsevier, 1977) p. 311.
\bibitem{Borisenko_CdAs}
S. Borisenko, \textit{et al.}, Experimental Realization of a Three-Dimensional Dirac Semimetal. \textit{Phys. Rev. Lett.} \textbf{113}, 027603 (2014).
\bibitem{Neupane_CdAs} 
M. Neupane, \textit{et al.}, Observation of a three-dimensional topological Dirac semimetal phase in high-mobility Cd$_3$As$_2$. \textit{Nature Communication} \textbf{5}, 3786 (2014).
\bibitem{Akrap_CdAs}
A. Akrap, \textit{et al.}. Magneto-Optical Signature of Massless Kane Electrons in Cd$_3$As$_2$. \textit{Phys. Rev. Lett.} \textbf{117}, 136401 (2016). 
\bibitem{Graphen_velocity}
R. S. Deacon, K.-C. Chuang, R. J. Nicolas, K. S. Novoselov, and A. K. Geim, Cyclotron resonance study of the electron and hole velocity in graphene monolayers. \textit{Phys. Rev. B} \textit{76}, 081406(R) (2007).
\bibitem{Berry_phase}
M. V. Berry. Quantal Phase Factors Accompanying Adiabatic Changes. \textit{Proceedings of the Royal Society A.} \textbf{392}, 1802, pp. 45-57 (1984).
\bibitem{Berezinski1}
V. L. Berezinski. Destruction of Long-range Order in One-dimensional and Two-dimensional Systems Having a Continuous Symmetry Group I. Classical Systems. \textit{Zh. Eksp. Teor. Fiz} \textbf{59}, pp. 907-920 (1970) [\textit{Sov. Phys.-JETP} \textbf{32}, 493 (1971)].
\bibitem{Berezinski2}
V. L. Berezinski. Destruction of Long-range Order in One-dimensional and Two-dimensional Systems Possessing a Continuous Symmetry Group II. Classical Systems. \textit{Zh. Eksp. Teor. Fiz} \textbf{61}, pp. 1144-1156 (1971) [\textit{Sov. Phys.-JETP} \textbf{34}, 610 (1972)].
\bibitem{Thouless_nobel}
J. M. Kosterlitz and D. J. Thouless. Ordering, metastability and phase transitions in two-dimensional systems. \textit{Journal of Physics C: Solid State Physics} \textbf{6}, pp. 1181-1203 (1973).
\bibitem{Bernevig_book}
B. A. Bernevig and T. L. Hughes. \textit{Topological insulators and topological superconductors}. Princeton University Press, 1st edition (2013).
\bibitem{Xu_CTI}
S.-Y. Xu, \textit{et al.} Observation of a topological crystalline insulator phase and topological phase transition in Pb$_{1-x}$Sn$_{x}$Te. \textit{Nature Communications} \textbf{3}, 1192 (2012).
\bibitem{Ikonnikov_MCT_SQW}
A. V. Ikonnikov, \textit{et al.} Temperature-dependent magnetospectroscopy of HgTe quantum wells. \textit{Phys. Rev. B} \textbf{94}, 155412 (2016).
\bibitem{Buttner_MCT_SQW}
B. Büttner, \textit{et al.} Single valley Dirac fermions in zero-gap HgTe quantum wells. \textit{Nature Physics} \textbf{7}, 418-422 (2011).
\bibitem{Ludwig_MCT_QW}
J. Ludwig, \textit{et al.} Cyclotron resonance of single-valley Dirac fermions in nearly gapless HgTe quantum wells. \textit{Phys. Rev. B} \textbf{89}, 241406 (2014).
\bibitem{Castro_graphene}
N. Castro, \textit{et al.} The electronic properties of graphene. \textit{Reviews of Modern Physics} \textbf{81}, 109 (2009).
\bibitem{Schultz}
M. Schultz, \textit{et al.} Crossing of conduction- and valence-subband Landau levels in an inverted HgTe/CdTe quantum well. \textit{Phys. Rev. B} \textbf{57}, 14772 (1998).
\bibitem{Zholudev_MCT_QW_anticrossing}
M. S. Zholudev, \textit{et al.} Anticrossing of Landau levels in HgTe/CdHgTe (013) quantum wells with an inverted band structure. \textit{Jetp Lett.} \textbf{100}, 790 (2015).
\bibitem{Martinez_Nanotechnology}
J. M. Martínez-Duart, R. J. Martín-Palma, and F. Agulló-Rueda. \textit{Nanotechnology for microelectronics and optoelectronics}. ISBN-13: 978 0080 445533. Elsevier (2006).

\end{thebibliography}

\appendix
\chapter{Appendix A}

\begin{equation}
\label{1}
\Psi = \left( \ket{\Gamma_6, \frac{1}{2}}, \ket{\Gamma_6, -\frac{1}{2}}, \ket{\Gamma_8, \frac{3}{2}}, \ket{\Gamma_8, \frac{1}{2}}, \ket{\Gamma_8, -\frac{1}{2}}, \ket{\Gamma_8, -\frac{3}{2}}  \right).
\end{equation}


\begin{equation}
\label{2}
H_{eff}(k_x, k_y) =  \left( \begin{array}{cc}
H(k) & 0 \\
0 & H^*(-k) \end{array} \right),
\end{equation}

where 

\begin{equation}
\label{3}
H = \epsilon (k) + d_i (k) \sigma_i,
\end{equation}

and $\sigma_i$ are Pauli matrices.

\begin{equation}
\begin{aligned}
\label{4}
d_1 + id_2 = A(k_x + ik_y) \equiv Ak_+ \\
d_3 = M - B(k_x^2 + k_y^2) \\
\epsilon_k = C - D(k_x^2 + k_y^2).
\end{aligned}
\end{equation}

\begin{equation}
\label{asd}
\hat{H}_{k_x, k_y} = \epsilon_k(k_x, k_y) I_{4 \times 4} + \left( \begin{array}{cccc}
\mathcal{M}(k) & Ak_- & 0 & 0 \\
Ak_+ & -\mathcal{M}(k) & 0 & 0 \\
0 & 0 & \mathcal{M}(k) & -Ak_+ \\
0 & 0 & -Ak_- & -\mathcal{M}(k) \end{array} \right),
\end{equation}

where $\epsilon_k(k_x, k_y) = C - D(k_x^2, k_y^2)$, $\mathcal{M}(k) = M - B(k_x^2, k_y^2)$, $k_pm = k_x \pm ik_y$, and \textit{A, B, C, D, M} depend on the specified QW width.

%\appendix
\chapter{Appendix B}


\end{document}
