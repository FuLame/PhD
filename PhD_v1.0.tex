\documentclass[titlepage,a4paper]{book}
\usepackage[english]{babel}
\usepackage[utf8]{inputenc} 
\usepackage{wrapfig}
\usepackage{graphicx}
\usepackage{color}
\usepackage{colortbl}
\usepackage{amsfonts}
\usepackage{amssymb}
\usepackage{amsmath}
\usepackage{amsthm}
\usepackage{subfigure}
\usepackage[papersize={21cm,29.7cm},text={15.5cm,24.5cm},left=2.7cm,top=2.5cm]{geometry}
\usepackage{enumerate}
\usepackage{rotfloat}
\usepackage[T1]{fontenc}

\newcommand{\wciecie}{\quad\phantom{v}}
\newcommand{\refe}[1]{(\ref{#1})}
%\fancyhf{}
%\fancyhead[R]{\small\bfseries\thepage}
%\fancyhead[L]{\small\bfseries\rightmark}
%\fancyfoot[C]{\small\bfseries\thepage}
%\renewcommand{\headrulewidth}{0.5pt}
%\renewcommand{\footrulewidth}{0pt}
%\renewcommand*{\tablename}{Tab.}
%\addtolength{\headheight}{0.5pt}
%\fancypagestyle{plain}

\begin{document}
\newcommand{\HRule}{\rule{\linewidth}{0.5mm}}
\newcommand{\angstrom}{\text{\normalfont\AA}}

%\begin{titlepage}
%\begin{center}
%\quad\phantom{.}\\[1.5cm]
%{\Huge Uniwersytet Warszawski}\\[0.5cm]
%{\huge Wydzial Fizyki}\\[2cm]
%{\Large Michał Marcinkiewicz}\\[0.1cm]
%{\large Nr albumu: 290813}\\[1.8cm]
%{\Huge Spektroskopia studni kwantowych GaAs/GaInAs}\\[2.3cm]
%{\large Praca magisterska}\\
%{\large na kierunku Fizyka}\\
%{\large specjalność Fizyka materii skondensowanej}\\
%{\large i nanostruktur półprzewodnikowych}\\[2.4cm]
%	\begin{flushright}
%	{\large Praca wykonana pod kierunkiem} \\[0.4cm]
%	dr. hab. Jerzego Łusakowskiego\\
%	Instytut Fizyki Doświadczalnej\\
%	Zakład Fizyki Ciała Stałego
%	\end{flushright}
%\vfill
%{\large Warszawa, Wrzesień 2014}
%\end{center}
%\end{titlepage}

%\newpage
%\thispagestyle{empty}
%{\large
%\quad\phantom{.}\\[2cm]
%\textit{Oświadczenie kierującego pracą}\\[0.6cm]
%Oświadczam, że niniejsza praca została przygotowana pod moim kierunkiem i stwierdzam, że spełnia ona warunki do przedstawienia jej w postępowaniu o %nadanie tytułu zawodowego.\\[1.2cm]
%Data \qquad \qquad \qquad \qquad \qquad \qquad \qquad Podpis kierującego\\[3.2cm]
%\textit{Oświadczenie autora pracy}\\[0.6cm]
%Świadom odpowiedzialności prawnej oświadczam, że niniejsza praca dyplomowa została napisana przeze mnie samodzielnie i nie zawiera treści uzyskanych w sposób niezgodny z obowiązującymi przepisami.\\[0.3cm]
%Oświadczam również, że przedstawiona praca nie była wcześniej przedmiotem procedur związanych z uzyskaniem tytułu zawodowego wyższej uczelni.\\[0.3cm]
%Oświadczam ponadto, że niniejsza wersja pracy jest identyczna z załączoną wersją elektroniczną.\\[1.2cm]
%Data \qquad \qquad \qquad \qquad \qquad \qquad \qquad Podpis autora pracy\\
%}

%\newpage
%\thispagestyle{empty}
%\begin{center}
%{\large
%\quad\phantom{.}\\[1cm]
%\textbf{Streszczenie}\\[0.5cm]}
%\end{center}{\large

%W pracy zostały scharakteryzowane pod względem elektrycznym i optycznym studnie kwantowe GaAs/GaInAs. Zbadano zjawiska fizyczne zachodzące w dwuwymiarowym gazie elektronowym, który był umieszczony w niskich temperaturach i polu magnetycznym, takie jak oscylacje Shubnikova -- de Haasa i rezonans cyklotronowy indukowany promieniowaniem z zakresu dalekiej podczerwieni. Analiza wyników pozwoliła na wyznaczenie koncentracji i ruchliwości elektronów. Przeprowadzono także pomiary fotoluminescencji, których wyniki dały informacje o energii rekombinacji 2DEG z dziurami z pasma walencyjnego w studni i zależności tej energii od wartości indukcji pola magnetycznego.\\[2cm]}
%\begin{center}{\large
%\textbf{Słowa kluczowe}\\[0.5cm]
%Studnia kwantowa, dwuwymiarowy gaz elektronowy, spektroskopia THz, fotoluminescencja\\[2cm]
%\textbf{Dziedzina pracy (kody według programu Socrates-Erasmus)}\\[0.5cm]
%13.2 Fizyka\\[2cm]
%\textbf{Tytuł pracy w języku angielskim}\\[0.5cm]
%Spectroscopy of GaAs/GaInAs quantum wells
%}
%\end{center}


\newpage
\large
\thispagestyle{empty}
\tableofcontents


\newpage

\chapter{Introduction}
\wciecie
Since the dawn of science, people always wanted to encounter new frontiers and cross them, expanding the boundaries of the universal knowledge. Even in the modern times, it is still our duty to explore the matter around us and the laws governing it. Although throughout the 19th  and 20th century, chemistry succeed in finding and classifying most of the elements -- building blocks of matter, there is still plenty of work to do in discovery and classification of the distinct states of matter, called phases. Matter in the quantum approach can form different phases, such as crystalline solids, magnets and superconductors. 

\section{Topological states of matter}
\wciecie
One of the most remarkable achievements of condensed matter physics in the recent times is the classification of quantum states of matter by the principle of spontaneous symmetry breaking \cite{Anderson_Topology}. The pattern of symmetry breaking led to an unique concept of order parameter, which can be understood in terms of the famous work of Landau-Ginzburg \cite{Landau_Topology}, where the notion of the effective field theory is described. The effective field theory is determined by the general properties like dimensionality and symmetry of the order parameter, and can be used to give an universal description of quantum states of matter. The following examples can be given to better understand the concept of states of matter and symmetry breaking: a crystalline solid breaks translation symmetry, despite the fact that the interaction between its atomic cells is translationally invariant. A rotational symmetry in magnetic systems is spontaneously broken, even though the fundamental interactions are isotropic. A superconductor breaks the more subtle gauge symmetry, which as a consequence leads to phenomena such as flux quantization and Josephson effect \cite{Josephson_Topology}.

The concept of symmetry breaking and local order parameter used to describe the phase transition is well accepted as in general. However, it fails to explain some interesting phenomena such as the integer quantum Hall effect \cite{Klitzing_Topology} and many body phases of the fractional quantum Hall effect \cite{Tsui_Topology}. The study of these effects ultimately led to a new paradigm in the classification of condensed matter systems -- the concept of topological order \cite{Laughlin_Topology} \cite{Thouless_Topology}.

The sample in quantum Hall state has a very particular properties. Its bulk is insulating (there is an energy gap between the highest occupied band and the lowest empty band) and the electric current is carried only along the edges of the sample, forming discrete channels of conductance. The current in the channels avoids dissipation and has a very precise value of resistance, giving rise to a quantized Hall effect. The quantum hall state turned out to be the first example of a quantum state being topologically different from all other states of matter known before. The state in which the quantum Hall occurs, does not break any symmetries but it defines a topological phase, meaning that particular fundamental properties are insensitive to smooth changes in general parameters of the system, and cannot change unless the system undergoes a quantum phase transition. The very fundamental reason for such a quantization is the existence of topological invariants, in case of the quantum Hall effect such an invariant is the conductance. The conductance takes values only in integer units of $e^2/h$, is independent of the type of material investigated and does not change (is invariant) for smooth variations of material parameters – it can be considered as non-local order parameter of the system.  

Topological invariant was introduced as a mathematical concept used to classify different geometrical objects into broad classes. One of such invariants can be the number of holes on the surface. The most famous example is a coffee mug and a torus. Both of them are classify as the same topological object because both of them posses exactly one hole and one can be deformed, via a smooth transformation, into the other, and vice-versa. A sphere can be smoothly deformed into an ellipsoid, because they share the same number of holes (zero), etc. Topology tends to disregard the small differences of objects and focus on their general properties. In this context, quantum Hall effect and quantized conductance, which can be found in a variety of materials of different shape, remains unchanged -- it is an invariant. 
 
The link between geometrical classes of objects and condensed matter physics is not direct -- topology in solid state physics has very little to do with shape of considered material, and definately the subject is much broader than just the quantum Hall effect. In topological geometry there are surfaces, holes and smooth transformations of surfaces, which does not require tearing the surface or making holes in it. In condense matter Hamiltonians are used to describe a given system, providing information about energy gaps (which play a role of holes in geometry) and its band structure. There is a possibility to transform given Hamiltonian into a different one, and if the transition is smooth in a sense that it does not require closing a gap (an equivalent of creating a hole in geometry) an any point, then the transition preserves the topology of the system.    

This concept, where number of gap closing is a topological invariant, can be applied only to systems, which posses the band gap -- semiconductors and insulators, it cannot be applied to metals or superconductors. Following this definition it is clear that two quantum states, having the same topology, put together do not have to host gapless states. However if the topology of two quantum states is different (for example a nontrivial state and vacuum), the interface must host gapless states. In this sense, a topological insulator is related to the two-dimensional integer quantum Hall state, which also posseses  unique edge states. These states, existing at the bonduary (egde in 2D, surface in 3D) of the topological insulator, lead to the existence of conducting states with properties unlike any other electronic systems, and are predicted to have special properties, like a vanishing effective mass and a relativistic (linear) dispersion relation. Effective mass in electronic band in inverted regime is negative. It is nothing unusuall, considering that the effective hole mass in a normal semiconductor is negative as well. This is a consequence of the shape of a band, but can be also understood in a relativistic approach. Einstein in his famous equation stated that energy is proportional to mass, thus in systems which a negative energy gap the mass should be negative. When the bands tend to continously join with a normal insulator at the boundary, the energy gap as well as the effective mass switches to positivie. The transition has to be smooth, so going from negative to positive value at some point the system has to have a gap closure, when the effective mass vanishes as well. At this point the particles have to be described by the relativistic equation with a linear dispersion relation \cite{Zawadzki_Topology}.    
	
The 2D topological insulator, called quantum spin Hall insulator (QSHI), was first predicted to occur in graphene \cite{Kane_Topology} and in strained GaAs \cite{Bernevig_Topology2}. One year later, in 2006. it was predicted by Bernevig \cite{Bernevig_Topology2}\cite{Bernevig_Topology1} and experimetally observed in 2007 by König \cite{Konig_Topology} in HgTe/CdTe quantum well system. Two years after the prediction of topological insulating phase in HgTe/CdTe quantum well, in 2008, Liu predicted a similar phase in a different kind of material -- a quantum well made of layers of InAs/GaSb sandwiched between AlSb walls -- a structure called composite quantum well. The quantum well exhibits an inverted phase similar to HgTe/CdTe quantum wells, which is a QSH state when the Fermi level lies inside the gap \cite{Liu_Topology}. 

The 3D topological insulating phase was predicted in the Bi$_{1-x}$Sb$_x$ alloy for a specyfic compositions $x$ \cite{Fu_Topology}, and shortly after the topologically nontrivial surface states were observed by angle-resolved photoemission spectroscopy (ARPES) by Hsieh \cite{Hsieh}. Similarly, the topological insulators in 3D were predicted in Bi$_2$Te$_3$, Sb$_2$Te$_3$ \cite{Zhang_Topology} and Bi$_2$Se$_3$ \cite{Zhang_Topology}\cite{Xia_Topology}. There compounds exhibit a large bulk band gap and a gapless surface states consisting of a single Dirac cone. Xia \cite{Xia_Topology} and Chen \cite{Chen_Topology} observed a linear disperion relation of this states using ARPES.


\section{This work}

\chapter{HgCdTe systems}
\wciecie
The Mercury-Cadmium-Telluride (MCT) alloy crystal is formed by II-VI compounds which crystallize in zincblende structure, which consists of two face-centered cubic sublattices. In the zincblende structure each of Te-ions have four nearest neighbours, which in the alloy can be either Hg or Cd. The presence of a different atom on each lattice site breaks the inversion symmetry, which results in reducing the point group symmetry from cubic to tetrahedral. Despite the fact that the inversion symmetry is explicitly broken, this has a small influence on the physics of the QSH effect.

The both compounds, HgTe and CdTe, are well lattice-matched, having the lattice constant parameter equal to 6.45 Å and 6.48 Å respectively. The band gap in Hg$_{1-x}$Cd$_x$Te mixed crystals varies from 1.6 eV for pure CdTe, a relatively large gap semiconductor to -0.3 eV for pure HgTe, a semimetal. The negative band gap is a consequence of the unusual band alignment in the crystal, where $p$-type ($\Gamma_8$) bands lie around 0.3 eV above the $s$-type ($\Gamma_6$) bands. In this work the attention is paid to both 3D (bulk) and 2D (quantum well) structures, based on MCT compounds, because both classes exhibit unusuall properties. HgTe/CdTe quantum well, as mentioned already in the introduction, is a topological insulator and a significant attention is going to be paid to this system in this chapter.  Hg$_{1-x}$Cd$_x$Te bulk system however, can be either a semiconductor or a semimetal, depending on its both internal and external parameters. This system does not exhibit a topological insulator phase, however it will be shown that it is possible to demonstrate a phase transition from negative to positive band gap, which makes this system particularly interesting.

\section{MCT Quantum Wells}
\subsection{Band structure}
A typical HgTe/CdTe QW is formed when a layer of HgTe is sandwiched between two layers of CdTe, which form barriers for the QW. In CdTe the conduction band edge states have a $p$-like  symmetry, while the valence band edge states have a $s$-like symmetry, as other semiconductor. However, because of an extraordinary large Spin-Orbit Coupling (SOC) in this material due to the presence of a heavy element Hg, in HgTe the $p$-like states lie above the $s$-like states, which leads to an inverted band regime. The light-hole $\Gamma_8$ band forms the conduction band, the heavy-hole band forms the first valence band, and the $s$-type $\Gamma_6$ band is pulled below the Fermi level and lies between the heavy-hole band and the spin-orbit split-off band $\Gamma_7$. The band order of both CdTe and HgTe is presented on Figure \ref{fig:BandStructure_HgTe_CdTe}.

\begin{figure}[ht]
	\centering
	\includegraphics[width=10cm]{gfx/BandStructure_HgTe_CdTe.png}
	\vspace{-10pt}
	\caption{\textit{Obsadz}}
	\label{fig:BandStructure_HgTe_CdTe}
\end{figure} 

The energy levels, for both electrons and holes, within the structure lie above the bottom of QW, and their position depends on the QW width. For thin quantum wells with well thickness $d_c$ $<$ 6.3 nanometers, the structure exhibit a normal semiconducting behavior, because the quantum confinement is strong and the energy levels are lifted from their respective QW bottom, and the first electron level ($E_1$) lies above the first hole level ($H_1$). In a different case, for quantum wells of thickness $>$ 6.3 nm, the situation is reversed. The $H_1$ level lies above the $E_1$, which results in semimetalic behavior. This means that for $d_c$ $=$ 6.3 nm the band gap vanishes and the system undergoes a topological phase transition from a trivial insulator to a quantum spin Hall insulator. The other way to understand this is to assume that for thin QW the structure should behave similarly to CdTe and have a regular band ordering, i.e. bands with $\Gamma_6$ symmetry form the conduction subbands and the $\Gamma_8$ symmetry bands form the valence subbands. If the QW thickness $d$ is increased the structure starts to more and more resemlbe the properties of HgTe. When the thickness reaches the critical value $d_c$ = 6.3 nm the $\Gamma_6$ and $\Gamma_8$ subbands cross and the structure becomes inverted -- the $\Gamma_6$ bands become valence subbands and the $\Gamma_8$ bands become conduction subbands. The QW states derived from the heavy hole $\Gamma_8$ band are named $H_n$, where $n = 1, 2, 3, ...$ denotes the states existing in the QW. Similarly, those levels originated from the electron $\Gamma_6$ band are named $E_n$. The band structure and energy levels of a HgTe/CdTe QW as a function of QW width are shown of Figure \ref{fig:BandStructure_HgTe_CdTe2}.  

\begin{figure}[ht]
	\centering
	\includegraphics[width=10cm]{gfx/BandStructure_HgTe_CdTe2.png}
	\vspace{-10pt}
	\caption{\textit{Obsadz}}
	\label{fig:BandStructure_HgTe_CdTe2}
\end{figure} 

One year after the theoretical proposal of Bernevig \cite{Bernevig_Topology2} the Molenkamp group at the Unversity of Würzburg fabricated the devices and performed the first transport experiments showing the signature of the quantum spin Hall insulator \cite{Konig_Topology}. This work showed that for thin quantum wells with well width $d$ $<$ 6.3 nm, the insulating regime show the conventional behavior of neglectable conductance at low temperature. However, for thicker quantum wells ($d$ $>$ 6.3 nm), the nominally insulating regime showed a plateau of residual conductance close to $2e^2/h$. The residual conductance is independent of the sample dimensions, indicating that it is caused by the edge states \cite{Konig_Topology}. The low temperature ballistic transport via edge states can be understood within a basic Landauer-Büttiker \cite{Landauer_MCT} framework, in which the egde states are populated adequatly to the chemical potential. As a consequence, conductance is quantized and equal to $e^2/h$ for each set of egde states.



Furthermore, the residual conductance is destroyed by applying a small external magnetic field. The quantum phase transition at the critical thickness, $d_c$ = 6.3 nm, was also determined independently from the insulator-to-semimetal phase transition induced by magnetic field

\section{MCT Bulk Crystals}
The earliest studies of Hg$_{1-x}$Cd$_x$Te crystals were aimed at the development of infrared detectors, especially for wavelenghts around 10 um, which can be tuned via changing composition. This is the range of the second wide atmospheric window, which means it is of great interest for communication. Morover it covers the range of the maximum of thermal radiation at room temperature, which opens possible applications for every day life. Nowadays, besides the infrared detection application, the interest of community is centered at the phenomena relative to the variations of effective mass of electrons and effective g-factor, coupled with the variation of the band gap. The band gap in Hg$_{1-x}$Cd$_x$Te mixed crystals ranges from positive 1.6 eV for pure CdTe to negative -0.3 eV for pure HgTe. The band gap varies continously and almost linearly with the Cadmium content $x$ -- the crystal composition. It means that at some point there must be a very special composition, where the band gap vanishes which, according to [], implies that effective mass vanishes too and g-factor goes to infinity. Crystals close to this composition are ideal materials for studying various quantum and spin effects, especially at the presence of magnetic field, as those effect are usually more pronounced as the g-factor increases, and far more other phenomena.

MCT bulk crystals give a special opportunity to realize and investigate a condensed matter system with particles exi

Special properties of the band structure of MCT has been known since the first works in early sixties. In seventies \cite{Zawadzki_MCT}



\subsection{Band Structure}
\wciecie
HgCdTe band structure depends on multiple properties. During the study only the chemical composition (Cadmium concentration) and temperature were considered. Temperature

\begin{figure}[ht]
	\centering
	\includegraphics[width=13cm]{gfx/MCT_bandStructure.png}
	\vspace{-10pt}
	\caption{\textit{Obsadz}}
	\label{asdfg}
\end{figure} 

\subsection{Physical properties}
%\label{section:domieszkowanie}
\wciecie
Wprowadzeniw:

\subsection{Effective electron mass}
\wciecie
Donadzenie  
\begin{figure}[ht]
	\centering
	\includegraphics[width=13cm]{gfx/wyniki/MCTbulk/110429-77K-6T_3.png}
	\vspace{-10pt}
	\caption{\textit{Pa}}
	\label{asd}
\end{figure} 

\begin{figure}[ht]
	\centering
	\includegraphics[width=13cm]{gfx/wyniki/MCTbulk/ThreeGraphs2.png}
	\vspace{-10pt}
	\caption{\textit{Obat).}}
	\label{asdf}
\end{figure} 


\chapter{InAs/GaSb Quantum Wells}
\wciecie
asd
\section{General properties}
There is a group of well lattice matched materials, so called 6.1 Å family, which consists of compounds such as InAs, GaSb, and AlSb. The name origins from a common lattice constant value, which is similar for all compounds and equals approximately 6.1 Å at the room temperature. InAs, GaSb, And AlSb crystallize in the zincblende structure. The similarity of the lattice constant allows growing lattice-matched heterostructures, without strain or other lattice-mismatch defects, like dislocations etc. The exact values of the lattice constants are 6.0584 Å for InAs, 6.0959 Å for GaSb, and 6.1355 Å for AlSb at 300 K \cite{Sze}.

Both InAs and GaSb are direct-gap semiconductor, with the exact value of energy gap measured at the $\Gamma$ point is $E_g^{InAs} = 415-0.276 T^2/(T-83)$ for InAs \cite{Fang_InAs}, and $E_g^{GaSb} = 813 - 0.108 T^2/(T-10.3)$ for GaSb \cite{Wu_GaSb}. AlSb has an indirect gap – the gap measured at the $X$ point equals to $E_g^{AlSb} = 1696 - 0.390 T^2/(T+140)$ \cite{Vurgaftman_AlSb}. The values of energy gaps at 300 K are equal to 350 meV for InAs, 727 meV for GaSb, and 1616 meV for AlSb. This wide range of gaps allows to create high energy electron confinement in quantum wells due to the high energy gap in AlSb barriers, which translates to deep quantum well with barriers as high as 1350 meV. The values of energy gap and lattice constants of 6.1 Å family are presented on Figure \ref{fig:BandStructure_InAs_GaSb}.

\begin{figure}[ht]
	\centering
	\includegraphics[width=10cm]{gfx/BandStructure_InAs_GaSb.png}
	\vspace{-10pt}
	\caption{\textit{Obsadz}}
	\label{fig:BandStructure_InAs_GaSb}
\end{figure} 

\subsection{Carrier concentration}
Because of an extraordinary depth of quantum well, there are multiple sources of electrons, which can influence the carrier density in the quantum well. Consequently, even not-intentionally doped structures have relatively high carrier concentration, of the order of magnitude of $1 \cdot 10^{12}$ cm$^{-2}$ \cite{Tuttle_InAs_concentration}. This particularly high concentration is mainly caused by three sources -- conventional shallow bulk donors, surface states, and deep donors at or near the interfaces. 
\begin{itemize}
\item \textbf{Conventional shallow donors and modulation doping} \newline 
Because of high barriers a modulation doping technique can be applied in those systems. It means that the donors are placed in the barrier instead of in the quantum well, and supplied electrons tunnel into the well leaving the ionized impurities away from the active part of the quantum well, and consequently reducing scattering and hence enhancing carrier mobility in comparison with bulk-doped quantum well. This method is widely used in structures, where both high concentration and mobility matters.

\item \textbf{Surface states} \newline
The states on the surface also take part in populating the quantum well. Electrons in the layers, placed outside barriers, can tunnel inside the well. The effect exists as a consequence of the extreme depth of the quantum well itself and strongly depends on the energetic location of the surface states and the chemical nature of the surface coverage \cite{Kroemer_review}. The top surface is usually made of either GaSb or InAs cap layers, which prevent against the oxidation of the structure. The cap layer can exhibit very high density of states (of the order of $10^{12}$ cm$^{-2}$), at the energies 0.85 eV below the conduction band edge of the AlSb barrier \cite{Nguyen_SurfaceDonor}. As a consequence the electrons will flow into the well until the electric field in the barrier pulls the surface states down to the same energy as the Fermi level inside the well. The barrier thickness strongly influences the tunneling, hence can suppress this effect.

\item \textbf{Interface donors and defects} \newline
Another source of electrons, responsible for such a high concentration, even in low temperature limit, might be connected with interface donors and/or defects. A study of temperature dependence of the electron concentration suggested that there exist a donor level less than 50 meV above the bottom of the bulk conduction band of InAs, which implies a level below the bottom of the quantum well, and
below the Fermi level in the well at the observed electron concentrations. But in this case only a small fraction of the donors will be ionized, calling for a donor a concentration much higher than the observed electron concentrations, on the order of about $3 \cdot 10^{12}$ donors cm$^{-2}$ per well \cite{Kroemer_review}. The nature of these donors is not clear. Kroemer proposed that they are not ordinary point defects, but are Tamm states at the InAs–AlSb interface \cite{Kroemer_TammStates}, inherent to the band structure of that interface. Alternatively, Shen et al. \cite{Shen_TammStates} proposed that the donors are very deep bulk donor states associated with AlSb antisite defects, that is, Al atoms on Sb sites.
\end{itemize}

\subsection{Carrier mobility}
One of the most interesting properties of InAs is the second highest electron mobility of all semiconductors (the first being InSb). This is a result of one of the smallest electron effective mass, only 0.023 of the free electron mass. This property makes this material very interesting, especially from a device perspective – high mobility allows to create faster electronics (eg. HEMTs). Also, high mobility improves transport properties – allows to  observe quantum effects such as Shubnikov -- de Haas oscillations or quantum hall effect.

There are multiple factors influencing the mobility of carriers in the quantum well. The is a strong dependence on the well width and on the electron sheet concentration, as well as on the quality of growth. Especially at low temperatures, where mobility is limited by impurity and interface scattering, the growth quality plays a very important role \cite{Nguyen_Mobility}.

\begin{itemize}
\item \textbf{Quantum well width} \newline  
Both room-temperature and low-temperature mobilities are significantly reduced in narrow wells due to the dominance of interface roughness scattering. The mobility peaks for well widths around 125 Å, and then decays again, most probably due to the onset of scattering by misfit dislocations nucleated as the quantum well width exceeds the critical layer thickness imposed by the 1.3\% lattice mismatch between InAs and AlSb \cite{Bolognesi_Width}. As a result, the majority of studies have used a 15 nm InAs quantum well for transport studies of this systems [all kroemer + Li, Journal of Crystal Growth 301–302 (2007) 181–184].

\item \textbf{Layers interfaces} \newline
There is an evidence, that the mobility can depend on the interfaces between separate layers themselves. Tuttle found that because both anion and cation change across an InAs-AlSb interface, it is possible to grow such wells with two different types of interfaces, one with an InSb-like bond configuration, the other AlAs-like. Electron mobility and concentration were found to depend very strongly on the manner in which the quantum wells' interfaces were grown, indicating that high mobilities are seen only if the bottom interface is InSb-like. The AlAs type of interfaces can introduce a sheet of donor defects, which increase the scattering, hence lower the mobility \cite{Tuttle_Interface}. Jenichen examined the interfaces in InAs\slash AlSb superlattices via X-ray scattering, and claims that strong interface roughness and intermixing is definitely present at those sites \cite{Jenichen_Interfaces}.

\item \textbf{Buffer} \newline
There were studies showing that not only the active part of QW influences the mobility and transport properties. Li showed by magneto-transport, Atomic Force Microscopy and X-Ray Diffraction that the electron mobility of AlSb/InAs/AlSb quantum well with GaSb buffer is higher than that with AlSb buffer though the surface and crystal qualities of AlSb buffer are better than GaSb buffer. The crystal quality can be increased by improving the growth process. Because of InAs relaxation on AlSb buffer, mismatch dislocations will appear in the InAs layer and the interfaces of InAs QW will get rough, which is suggested as the reason leading to the lower mobility of InAs quantum well grown on AlSb buffer than on GaSb buffer \cite{Li_Buffer}. 

Thomas investigated this matter as well, and found that GaSb buffers provide atomically flat interfaces on the scale of the electron Fermi wavelength for the quantum wells. In contrast, AlSb buffers generate a very rough interface on the same scale. The low temperature mobility of their samples with GaSb buffer ($\mu = 240000$ cm$^2$/Vs) was seven times greater than of the samples with the AlSb buffer ($\mu = 35000$ cm$^2$/Vs), for concentration $n = 5.5 \cdot 10^{11}$ cm$^{-2}$. For the concentration of $n = 1.3 \cdot 10^{12}$ cm$^{-2}$ the difference was a bit smaller, but the mobilities were enormous $\mu = 944000$ cm$^2$/Vs and $\mu = 244000$ cm$^2$/Vs for GaSb and AlSb respectively \cite{Nguyen_Mobility}\cite{Thomas_Buffer}\cite{Nguyen_Buffer}.

\item \textbf{Carrier concentration} \newline
There are generally two ways to change the electron concentration of the sample -- with and without altering the sample structure. The first is doping, which was already mentioned. Doping with donors, provides additional carriers into the structure, which tunnel into the QW and increase its carrier density. On the other hand it introduces scattering centers, which gradually lower the concentration, even with a modulation doping technique \cite{Nguyen_Buffer}. 

The latter is based on gating technique or photoexcitation effects like persistent photoconductivity effect, which will be discussed later. Structure with a gate allow to continuously tune the carrier concentration in a relatively broad rage. However there is a strong relation between electron concentration and mobility. Nguyen et al investigated this dependence in gated InAs/AlSb quantum wells and found that the dependence is not monotonic -- mobility increases as the first subband gets populated, peaks, and has a minimum when the second subband is populated, then increases again \cite{Nguyen_Mobility}.

\end{itemize}

\section{Band ordering}
The title of the most bizarre lineup of the 6.1 Å family belongs to the InAs/GaSb heterojunctions. It was found already back in 1977 by Sasaki et al. \cite{Sasaki_BandOrdering}, that they exhibit a broken gap band ordering -- at the interface, the bottom of the condunction band of InAs has lower energy states than the valence band states of GaSb, with a break gap of 150 meV. This discovery most likely attracted that much attention and interest in the entire 6.1 Å family. This particular band ordering was also predicted theoretically -- in the same year Frensley \cite{Frensley_BandOrdering} had claimed that there is a possibility of such a lineup to exist. Also, Harrison using LCAO theory \cite{Harrison_BandOrdering} suggested a similar prediction. Quantum wells of InAs/GaSb sandwiched between layers of AlSb were named broken-gap quantum wells (BGQW) or composite quantum wells (CQW). The band structure of InAs/GaSb CQW is presented on Figure \ref{fig:BandStructure_InAs_GaSb2}A. The valence band of AlSb lies about 0.4 eV lower than the valence band of GaSb, while the conduction band of AlSb lies approximately 0.4 eV higher than the conduction band of AlSb. As a consequence, AlSb can be used as a quantum well barrier in CQW systems.

\begin{figure}[ht]
	\centering
	\includegraphics[width=10cm]{gfx/BandStructure_InAs_GaSb2.png}
	\vspace{-10pt}
	\caption{\textit{Obsadz}}
	\label{fig:BandStructure_InAs_GaSb2}
\end{figure} 

The quantum well width governs two fundamentally different regimes of the system. Similarly to MCT QW the energy position of levels in a QW depends strongly on QW width. However, in the case of InAs/GaSb CQW, where none of the compounds used has an inverted structure by itself, the relative position of $E_1$ in InAs well and $H_1$ in GaSb well decides whether the CQW is in an inverted regime or not. For thin layers $E_1$ lies higher in energy than $H_1$ and the structure is in the normal state. On the other hand, for sufficiently thick layers the band order is reversed. In the past, the inverted regime was considered as a semimetallic \cite{Chang_BandStructure}. 

Since the transition between an inverted regime and a normal regime has to be smooth and continuous, there must be a point, where the $E_1 = H_1$, so the momentum and carrier energy in the two wells are close to be equal. In this state the system is strongly coupled, and both the electron states and the hole states are mixed. This mixing leads to hybridization of the bands or, simply, to lifting of degeneracy of the levels. As a consequence, in analogy to bonding and antibonding states, a relativly small (of the order of a few of meV) hybridization gap appears. Thus, the semimetallic band dispersion relation in Figure \ref{fig:BandStructure_InAs_GaSb2}B (dashed), becomes nonmonotonic (full line), with a mini-gap $\Delta$. Because of this, a InAs/GaSb CQW in an inverted regime is not a semimetal, but has a gap \cite{Altarelli_BandStructure}. The presence of the gap means that the bulk of the sample is resisitive, as long as the Fermi energy lies within the gap, which is a case in a theoretical intrinsic case. The existence of the gap was discovered experimentally by Yang \cite{Yang_BandStructure} and Lakrimi \cite{Lakrimi_BandStructure}.

Because of the broken gap alignment, the CQW in the inverted regime should have equal intrinsic density of both two-dimensional electron gas (2DEG) and two-dimensional hole gas (2DHG). However, in the actual samples there are localized states at interfaces, defects or inhomogeneities, which potentially lead to unbalanced concentrations of carriers. Magnetotransport studies indicate that the electron density dominates over the hole density \cite{Petchsingh_BandOrdering}\cite{Munekata_BandOrdering}, which is partially caused by surplus of electrons, tunneled from the high surface states. As a consequence, the Fermi level is set around 130 meV above the bottom of the conduction band of InAs \cite{Nguyen_Mobility}, which is different from its bulk value.

\subsection{Magnetic field tunability}
Magnetic field can be used to induce a electron and hole recombination in GaSb/InAs QW in inverted regime. Energy spectrum of 2DEG and 2DHG in magnetic field tends to quantize into a discrete levels called Landau levels. Energy of these levels increases for electrons and decreases for holes as the magnetic field gets stronger, which is a consequence of the opposite sign of carriers. Works of Lo \cite{Lo_MagneticField} and Smith \cite{Smith_MagneticField} show that it is possible, by applying external magnetic field, to bring the hole Landau level and the electron Landau level to the same value of energy, thus effectively crossing them and force electrons to recombine with holes, which leads to semiconductor to semimetal phase transition.

\subsection{Electric field tunability}
There were studies showing that band relative position can be altered also with electric field. Naveh et al. Found that applying small electric fields in a growth direction of AlSb-InAs-GaSb-AlSb QW, any value can be achieved for such parameters as the energy gaps, effective masses, and carrier types and densities in the material, which is a consequence of a strong electron-hole coupling in this system \cite{Naveh_ElectricField}. The electron and hole levels in this system, when subjected to an external electrical field, usually by a metallic gate on the top (and/or the bottom) of the structure, shift in opposite directions, approaching each other and eventually crossing. In the intrinsic case, thus when n$_H$ = n$_E$, Fermi energy lies somewhere between the first electron level ($E_1$) and the first hole level ($H_1$), and with increasing electric field $H_1$ crosses it first, removing the holes from the system, and $E_1$ follows at higher voltages applied between the gate and the channel, depleting completely the region of carriers. It gives a possibility to control the concentrations of carriers by relatively small electric field and in consequence, induce a semimetal to semiconductor phase transition \cite{Mendez_ElectricField}.

However, to control both the band structure and the Fermi level position, there is a need to use two gate configuration. A top gate and a back gate are used to tune the concentration of electrons and holes, respectively. Two gate configuration is required because of the screening of top gate induced electric field by electrons in InAs layer. Also, a very important issue is resolved in this way – electron and hole densities can be tuned almost indepentently, with respect to the total charge in the system. This idea was used by Cooper \cite{Cooper_ElectricField} to study a coupling of electrons and holes in InAs/GaSb/AlSb systems as a function of temperature and concentration. The two gate configuration was used as a concept in a theoretical work of Liu, in which the idea of a topological insulator based on a InAs/GaSb/AlSb QW was introduced \cite{Liu_Topology}.

\section{Topological properties}
The band structure of InAs/GaSb QW is inverted because of the unique relative band alignment of InAs and GaSb layers. As a consequence, the valence band of GaSb is higher energetically than the conduction band of InAs. A smooth connection of bands inside and outside of the sample with a matter of different topology (a regular one, e.g. vacum or trivial insulator) leads to a gapless states at the boundary with a linear dispersion relation. The mechanism of a band inversion differs from the one, which causes the inversion of the band structure of HgTe/CdTe QW, where the SOC was responsible for pushing the $\Gamma_8$ bands above the $\Gamma_6$ bands. SOC is however, still relevant in

From the topological insulator point of view a InAs/GaSb CQW, for a particular space of parameters, exhibits an inverted band order phase with gapless boundary states described by a linear dispersion relation. The bulk of the sample is nonconductuve, because there is a small gap. Those boundary states are protected by the time-reversal symmetry from elastic backscattering and are topologically robust to small perturbations. This means that InAs/GaSb QW shoudl be in a topologically nontrivial QSH phase.





\begin{thebibliography}{99}
%Introduction topology
\bibitem{Anderson_Topology}
Anderson P. W. (1997), Basic Notions of Condensed Matter Physics (Westview Press).
\bibitem{Landau_Topology}
Landau L. D., and E. M. Lifshitz (1980), Statistical Physics (Pergamon Press, Oxford).
\bibitem{Josephson_Topology}
B. D. Josephson, Possible new effects in superconductive tunneling, \textit{Phys. Lett.} \textbf{1} pp. 251–253 
\bibitem{Klitzing_Topology}
K. v. Klitzing, G. Dorda and M. Pepper, New Method for High-Accuracy Determination of the Fine-Structure Constant Based on Quantized Hall Resistance, \textit{Phys. Rev. Lett.} \textbf{45}, 494 (1980).
\bibitem{Tsui_Topology}
D. C. Tsui, H. L. Stormer and A. C. Gossard, Two-Dimensional Magnetotransport in the Extreme Quantum Limit, \textit{Phys. Rev. Lett.} \textbf{48}, 1559 (1982).
\bibitem{Laughlin_Topology}
R. B. Laughlin, Quantized Hall conductivity in two dimensions, \textit{Phys. Rev. B.} \textbf{23}, 5632 (1981).
\bibitem{Thouless_Topology}
Thouless D. J., M. Kohmoto, M. P. Nightingale, and M. den Nijs, \textit{Phys. Rev. Lett.} \textbf{49}, 405 (1982).
\bibitem{Zawadzki_Topology}
W. Zawadzki, Narrow Gap Semiconductors Physics and Application, (1979).
\bibitem{Kane_Topology}
C. L. Kane and E. J. Mele, Quantum Spin Hall Effect in Graphene, \textit{Phys. Rev. Lett.} \textbf{95}, 226801 (2005).
\bibitem{Bernevig_Topology2}
B. A. Bernevig, T. L. Hughes, S.-C. Zhang, Quantum Spin Hall Effect and Topological Phase Transition in HgTe Quantum Wells, \textit{Science} \textbf{314}, 5806, pp. 1757-1761 (2006).
\bibitem{Bernevig_Topology1}
B. A. Bernevig, S.-C. Zhang, Quantum Spin Hall Effect, \textit{Phys. Rev. Lett.} \textbf{96}, 106802 (2006).
\bibitem{Konig_Topology}
M. König, \textit{et al.}, Quantum Spin Hall Insulator State in HgTe Quantum Wells, \textit{Science} \textbf{318}, 5851, pp. 766-770 (2007).
\bibitem{Liu_Topology}
C. Liu, T. L. Hughes, X.-L. Qi, K. Wang and S.-C. Zhang, \textit{Phys. Rev. Lett.} \textbf{100}, 236601 (2008).
\bibitem{Fu_Topology}
L. Fu, C. L. Kane and E. J. Mele, Topological Insulators in Three Dimensions, \textit{Phys. Rev. Lett.} \textbf{98}, 106803 (2007).
\bibitem{Hsieh}
D. Hsieh \textit{et al.}, A topological Dirac insulator in a quantum spin Hall phase, \textit{Nature} \textbf{452}, pp.970-974 (2008).
\bibitem{Zhang_Topology}
H. Zhang \textit{et al.}, Topological insulators in Bi$_2$Se$_3$, Bi$_2$Te$_3$ and Sb$_2$Te$_3$ with a single Dirac cone on the surface, \textit{Nature Physics} \textbf{5}, pp.438-442 (2009).
\bibitem{Xia_Topology}
Y. Xia \textit{et al.}, Observation of a large-gap topological-insulator class with a single Dirac cone on the surface, \textit{Nature Physics} \textbf{5}, pp.398-402 (2009).
\bibitem{Chen_Topology}
Y. L. Chen \textit{et al.}, Experimental Realization of a Three-Dimensional Topological Insulator, Bi$_2$Te$_3$,  \textit{Science} \textbf{325}, 5937 pp.178-181 (2009).

%MCT
\bibitem{Landauer_MCT}
M. Büttiker, Absence of backscattering in the quantum Hall effect in multiprobe conductors, \textit{Phys. Rev. B} \textbf{38}, 9375 (1988).
\bibitem{Zawadzki_MCT}
W. Zawadzki, Electron transport phenomena in small-gap semiconductors. \textit{Advances in Physics} \textbf{23}, pp. 435–522 (1974).
\bibitem{Novoselow_MCT}
K. S. Novoselov, \textit{et al.}, Electric Field Effect in Atomically Thin Carbon Films. \textit{Science} \textbf{306}, 5696 pp. 666-669 (2004).
\bibitem{Wallace_MCT}
P.R. Wallace, The Band Theory of Graphite. \textit{Phys. Rev.} \textbf{71} 622 (1947).


%GaSb/InAs
\bibitem{Sze}
S. M. Sze, \textit{Physics of Semiconductor Devices.} John Wiley and Sons, New York, second edition (1981).
\bibitem{Fang_InAs}
Z. M. Fang \textit{et al.}, Photoluminescence of InSb, InAs, and InAsSb grown by organometallic vapor phase epitaxy, \textit{J. Appl. Phys.} \textbf{67}, 7034 (1990).
\bibitem{Wu_GaSb}
M. Wu, C. Chen, Photoluminescence of high quality GaSb grown from Ga and Sb rich solutions by liquidphase epitaxy, \textit{J. Appl. Phys.} \textbf{72}, 4275 (1992).
\bibitem{Vurgaftman_AlSb} 
I. Vurgaftman, J. R. Meyer and L. R. Ram-Mohan, Band parameters for III–V compound semiconductors and their alloys, \textit{J. Appl. Phys.} \textbf{89}, 5815 (2001).
\bibitem{Tuttle_InAs_concentration}
G. Tuttle, H. Kroemer and J. H. English, Electron concentrations and mobilities in AlSb/InAs/AlSb quantum wells, \textit{J. Appl. Phys.} \textbf{65}, 5239 (1989).
\bibitem{Kroemer_review}
H. Kroemer, The 6.1 Å family (InAs, GaSb, AlSb) and its heterostructures: a selective review, \textit{Physica E} \textbf{20}, 196-203 (2004). 
\bibitem{Nguyen_SurfaceDonor}
C. Ngyuen, B. Brar, H. Kroemer and J. H. English, Surface donor contribution to electron sheet concentrations in not-intentionally doped InAs-AlSb quantum wells, \textit{Appl. Phys.Lett.} \textbf{60}, 1854 (1992).
\bibitem{Kroemer_TammStates}
H. Kroemer, C. Ngyuen and B. Brar, Are there Tamm-state donors at the InAs–AlSb quantum well interface, \textit{J. Vac. Sci. Technol. B} \textbf{10}, 1769 (1992). 
\bibitem{Shen_TammStates}
J. Shen, H. Goronkin, J. D. Dow and S. Y. Ren, Tamm states and donors at InAs/AlSb interfaces, \textit{J. Appl. Phys.} \textbf{77}, 1576 (1995).
%Mobility
\bibitem{Nguyen_Mobility}
C. Nguyen, \textit{et al.}, Growth of InAs-AlSb Quantum Wells Having Both High Mobilities and High Concentrations, \textit{Journal of Electronic Materials} \textbf{22}, 255-258 (1993).
\bibitem{Bolognesi_Width}
C. R. Bolognesi, H. Kroemer and J. H. English, Well width dependence of electron transport in molecular-beam epitaxially grown InAs/AlSb quantum wells, \textit{J. Vac. Sci. Technol. B} \textbf{10}, 877 (1992).
\bibitem{Tuttle_Interface}
G. Tuttle, H. Kroemer and J. H. English, Effects of interface layer sequencing on the transport properties of InAs/AlSb quantum wells: Evidence for antisite donors at the InAs/AlSb interface, \textit{J. Appl. Phys.} \textbf{67}, 3032 (1990).
\bibitem{Jenichen_Interfaces}
B. Jenichen, S. A. Stepanov, B. Brar and H. Kroemer, Interface roughness of InAs/AlSb superlattices investigated by X-ray scattering, \textit{J. Appl. Phys.} \textbf{79}, 120 (1996).
\bibitem{Li_Buffer}
Z. H. Li, \textit{et al.}, Buffer influence on AlSb/InAs/AlSb quantum wells, \textit{Journal of Crystal Growth}, 181–184 301–302 (2007).
\bibitem{Thomas_Buffer}
M. Thomas, H.-R. Blank, K. C. Wong and H. Kroemer, Buffer-dependent mobility and morphology of InAs/(Al,Ga)Sb quantum wells, \textit{Journal of Crystal Growth}, 175/176 (1997) 849-897.
\bibitem{Nguyen_Buffer}
C. Ngyuen, K. Ensslin and H. Kroemer, Magneto-transport in InAs/AlSb quantum wells with large electron concentration modulation, \textit{Surface Science} \textbf{267}, pp. 549-552 (2007).
%Band Ordering
\bibitem{Sasaki_BandOrdering} 
H. Sasaki, \textit{et al.}, In$_{1-x}$Ga$_x$As-GaSb$_{1-y}$As$_y$ heterojunctions by molecular beam epitaxy, \textit{Appl. Phys. Lett.} \textbf{31}, 211 (1977).
\bibitem{Frensley_BandOrdering}
W. R. Frensley and H. Kroemer, Theory of the energy-band lineup at an abrupt semiconductor heterojunction, \textit{Phys. Rev. B} \textbf{16}, 2642 (1977).
\bibitem{Harrison_BandOrdering}
W. A. Harrison, Elementary theory of heterojunctions, \textit{J. Vac. Sci. Technol.} \textbf{14}, 1016 (1977).
\bibitem{Chang_BandStructure}
L. L. Chang and L. Esaki, Electronic properties of InAs-GaSb superlattices \textit{Surf. Sci.} \textbf{98}, 70 (1980).
\bibitem{Altarelli_BandStructure}
M. Altarelli, Electronic structure and semiconductor-semimetal transition in InAs-GaSb superlattices, \textit{Phys. Rev. B} \textbf{28}, 842 (1983).
\bibitem{Yang_BandStructure} 
M. J. Yang, C. H. Yang, B. R. Bennett and B. V. Shanabrook, Evidence of a Hybridization Gap in "Semimetallic" InAs/GaSb Systems, \textit{Phys. Rev. Lett.} \textbf{78}, 4613 (1997).
\bibitem{Lakrimi_BandStructure}
M. Lakrimi, Minigaps and Novel Giant Negative Magnetoresistance in InAs/GaSb Semimetallic Superlattices, \textit{et al.}, \textit{Phys. Rev. Lett.} \textbf{79}, 3034 (1997).
\bibitem{Petchsingh_BandOrdering}
C. Petchsingh, Cyclotron Resonance Studies on InAs/GaSb heterostructures, Wolfson College, University of Oxford, 2002.
\bibitem{Munekata_BandOrdering}
H. Munekata, E. E. Mendez, Y. Iye and L. Esaki, Densities and mobilities of coexisting electrons and holes in GaSb/InAs/GaSb Quantum Wells, \textit{Surface Science} \textbf{174}, pp. 449-453 (1986).
\bibitem{Lo_MagneticField}
I. Lo, W. C. Mitchel and J.-P. Cheng, Magnetic-field-induced free electron and hole recombination in semimetallic Al$_x$Ga$_{1-x}$Sb/InAs quantum wells, \textit{Phys. Rev. B} \textbf{48}, 9118 (1993).
\bibitem{Smith_MagneticField}
T. P. Smith, H. Munekata, L. L. Chang, F. F. Fang and L. Esaki, Magnetic-Field-Induced Transitions in InAs/Ga$_{1-x}$Al$_x$Sb Heterostructures, \textit{Surface Science} \textbf{196}, pp. 687-693 (1988).
\bibitem{Naveh_ElectricField}
Y. Naveh and B. Laikhtman, Band-structure tailoring by electric field in a weakly coupled electron-hole system, \textit{Appl. Phys. Lett.} \textbf{66} 1980 (1995).
\bibitem{Mendez_ElectricField}
E. E. Mendez, L. L. Chang and L. Esaki, Two-Dimensional Quantum States in Multi-Heterostructures of Three Constituents, \textit{Surface Science} \textbf{13} 474-478 (1982).
\bibitem{Cooper_ElectricField}
L. J. Cooper, \textit{et al.}, Resistance resonance induced by electron-hole hybridization in a strongly coupled InAs/GaSb/AlSb heterostructure \textit{Phys. Rev. B} \textbf{57} 11915 (1998).

\end{thebibliography}
\end{document}
